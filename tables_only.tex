\documentclass{thesisby}

%\usepackage[cp1251]{inputenc}
\usepackage[utf8]{inputenc}
\usepackage[T2A]{fontenc}
\usepackage[russian]{babel}

%\usepackage{pscyr}
%\renewcommand{\rmdefault}{ftm}

%для борьбы с переполнениями за счет разреженных слов в абзаце
\emergencystretch=25pt
%math
\usepackage{mathtools}
\usepackage{tabularx}
\usepackage{pdflscape}

\usepackage{afterpage}
\usepackage{geometry}
\usepackage{multirow}
\usepackage{enumitem}

\usepackage{amsmath,amssymb,amsfonts}
\usepackage{longtable,array}
\usepackage{graphicx,epsfig}
\usepackage[unicode,colorlinks=false,pagebackref=false]{hyperref}
%\usepackage{refcheck}% checks lost and useless labels, shows `keys' of \label in the margins
%фиксить картинки
%\usepackage{float}

\usepackage{textcomp}% For Celsium sign only

% Листинги с исходным кодом программ
\usepackage{fancyvrb}
\usepackage{listings}
\usepackage{totcount}
%\usepackage{totpages}

% Плавающие окружения. во многом лучше пакета float
\usepackage{floatrow}

\usepackage{tikz}
\newcolumntype{P}[1]{>{\centering\arraybackslash}p{#1}}
\newcolumntype{M}[1]{>{\centering\arraybackslash}m{#1}}

%for lists
\usepackage[ampersand]{easylist}
\ListProperties(Hide=100, Hang=false, Margin=0mm, Indent1=10.5mm, Indent2=15mm, Style*=-- ,
Style2*=$\bullet$ ,Style3*=$\circ$ ,Style4*=\tiny$\blacksquare$ )

\usepackage[ruled, vlined]{algorithm2e}
\makeatletter
\newenvironment{algo}[1][]
  {\renewcommand{\algorithmcfname}{Алгоритм}%
   \begin{algorithm}[#1]
   \long\def\@caption##1[##2]##3{%
     \par
     \begingroup\@parboxrestore
     \if@minipage\@setminipage\fi
     \normalsize \@makecaption{\AlCapSty{\AlCapFnt\algorithmcfname}}{\ignorespaces ##3}%
     \par\endgroup
   }}
  {\end{algorithm}}
\makeatother

% \SetKwData{KwDat}{Данные}
\SetKwInput{KwIn}{Исходные данные}
\SetKwInput{KwRes}{Результат}%
% \SetKwIF{Si}{SinonSi}{Sinon}{si}{alors}{sinon si}{sinon}{fin si}%
% \SetKwFor{Tq}{tant que}{faire}{fin tq}%

\newenvironment{easylistNum}{\begin{easylist}
\ListProperties(Hide1=0, Hang=false, Margin=0mm, Indent1=10.5mm, Indent2=15mm, Start1=1, Style*=, FinalMark={)})}{\ListProperties(Hide=100, Hang=false, Margin=0mm, Indent1=10.5mm, Indent2=15mm, Style*=-- ,
Style2*=$\bullet$ ,Style3*=$\circ$ ,Style4*=\tiny$\blacksquare$ )\end{easylist}}

\renewcommand\labelitemi{\textbf{--}}
\floatsetup[table]{capposition=top}

    % \lstset{
    %     language=Python,
    %     basicstyle=\ttfamily,
    %     keywordstyle=\bfseries,
    %     showspaces=false,
    %     showstringspaces=false,
    %     mathescape=true,
    %     aboveskip=0pt,
    %     belowskip=6pt
    % }
\lstdefinestyle{PythonStyle}{
  keywordstyle=\bf,
  belowcaptionskip=1\baselineskip,
  breaklines=true,
  language=Python,
  showstringspaces=false,
  basicstyle=\small\ttfamily,
  commentstyle=\itshape
}
% \lstdefinestyle{PythonStyle}{
% basicstyle=\footnotesize\ttfamily,
% language=Python,
% keywordstyle=\bfseries,
% showstringspaces=false,
% commentstyle={},
% texcl=true
% }

\graphicspath{{../}}

%\renewcommand{\cftchapleader}{\cftdotfill{\cftdotsep}}


\begin{document}

\begin{table} [H]
  \small
  \caption{Средняя продолжительность разработки агентов}\label{TimeNormTable}
\begin{tabularx}{\hsize}{| X | X | X | X |}
  \hline
 \multirow{2}{*}{Тип агента} &\multicolumn{2}{c|}{Длительность операции, ч } & \multirow{2}{*}{Всего, ч}\\
 \cline{2-3}
  & Реализация агента & Отладка агента & \\
\hline
 
Поисковый агент	& 1,5	& 0,5	& 2\\
\hline
Усложненный поисковый агент	& 8	& 3	& 11\\
\hline
Агент логического вывода	& 12	& 4	& 16\\
\hline
Агент расчета операций	& 6	& 2	& 8\\
\hline
\end{tabularx}
\end{table}


\begin{table} [H]
  \small
  \caption{Количество агентов в решателях}\label{AgentsCountTable}
\begin{tabularx}{\hsize}{| p{2.5cm} | X | X | X | X | X | X | X |}
  \hline
 \multirow{2}{*}{Тип агента} &\multicolumn{7}{c|}{Количество агентов, шт.}\\
 \cline{2-8}
 & ИСС по теории множеств & ИСС по числовым моделям	& ИСС по истории	& ИСС по химии	& ИСС по географии	& Система автоматизации предприятия	& Система обслуживания розничной торговли\\
\hline
Поисковые агенты,  & 9 & 15 & 12 & 10 & 11 & 12 & 9\\
\cline{2-8}
в том числе заимствованных & 7 & 8 & 7 & 6 & 7 & 10 & 5\\
\hline
Усложнен-ные поисковые агенты, & 5 & 11 & 13 & 5 & 5 & 6 & 6\\
\cline{2-8}
в том числе заимствованных & 3 & 5 & 5 & 3 & 3 & 3 & 3\\
\hline
Агенты \mbox{логического} вывода, & 1 & 3 & 4 & 3 & 1 & 0 & 0\\
\cline{2-8}
в том числе заимствованных & 0 & 0 & 1 & 1 & 0 & 0 & 0\\
\hline
Агенты расчета операций, & 8 & 20 & 3 & 10 & 6 & 4 & 2\\
\cline{2-8}
в том числе заимствованных & 5 & 5 & 1 & 4 & 0 & 1 & 0\\
\hline
Итого  & 23 & 49 & 32 & 28 & 23 & 22 & 17\\
\hline
Итого заимствованных & 15 & 18 & 14 & 14 & 10 & 14 & 8\\
\hline

\end{tabularx}
\end{table}


\begin{table} [H]
  \small
  \caption{Расчет временных затрат на разработку агентов}\label{AgentsTimeTable}
\begin{tabularx}{\hsize}{| p{2.5cm} | X | X | X | X | X | X | X |}
  \hline
 \multirow{2}{*}{Тип агента} &\multicolumn{7}{c|}{Продолжительность разработки агентов, ч}\\
 \cline{2-8}
 & ИСС по теории множеств & ИСС по числовым моделям	& ИСС по истории	& ИСС по химии	& ИСС по географии	& Система автоматизации предприятия	& Система обслуживания розничной торговли\\
\hline
Поисковые агенты,  & 18 & 30 & 24 & 20 & 22 & 24 & 18\\
\cline{2-8}
в том числе заимствованных & 14 & 16 & 14 & 12 & 14 & 20 & 10\\
\hline
Усложнен-ные поисковые агенты, & 55 & 121 & 143 & 55 & 55 & 66 & 66\\
\cline{2-8}
в том числе заимствованных & 33 & 55 & 55 & 33 & 33 & 33 & 33\\
\hline
Агенты \mbox{логического} вывода, & 16 & 48 & 64 & 48 & 16 & 0 & 0\\
\cline{2-8}
в том числе заимствованных & 0 & 0 & 16 & 16 & 0 & 0 & 0\\
\hline
Агенты расчета операций, & 64 & 160 & 24 & 80 & 48 & 32 & 16\\
\cline{2-8}
в том числе заимствованных & 40 & 40 & 8 & 32 & 0 & 8 & 0\\
\hline
Итого  & 153 & 359 & 255 & 203 & 141 & 122 & 100\\
\hline
Итого заимствованных & 87 & 111 & 93 & 93 & 47 & 61 & 43\\
\hline

\end{tabularx}
\end{table}


\begin{table} [H]
  \small
  \caption{\hangindent=33mm \hangafter=1 Удельный вес продолжительности разработки заимствованных агентов в общем времени разработки общего числа агентов для \textit{j}-го решателя}\label{AgentsPercentTable}
\begin{tabularx}{\hsize}{| p{3.6cm} | X | X | X | X | X | X | X |}
  \hline
 \multirow{2}{*}{Показатель} &\multicolumn{7}{c|}{Система}\\
 \cline{2-8}
 & ИСС по теории множеств & ИСС по числовым моделям	& ИСС по истории	& ИСС по химии	& ИСС по географии	& Система автоматизации предприятия	& Система обслуживания розничной торговли\\
\hline
Продолжительность разработки агентов для \textit{j}-системы, ч & 153 & 359 & 255 & 203 & 141 & 122 & 100\\
\hline
Продолжительность разработки заимствованных агентов для \textit{j}-системы, ч & 87 & 111 & 93 & 93 & 47 & 61 & 43\\
\hline
Удельный вес продолжительности разработки заимствованных агентов в общем времени разработки агентов для \textit{j}-системы, \% & 56,86 & 30,92 & 36,47 & 45,81 & 33,33 & 50 & 43\\
\hline

\end{tabularx}
\end{table}


\begin{table} [H]
  \small
  \caption{Анализ заимствования многократно используемых агентов}\label{LibraryUseTable}
\begin{tabularx}{\hsize}{| X | X | X | X |}
  \hline
 Прототип системы & Общее число агентов & Число заимствованных агентов & Процент заимствования\\
\hline
ИСС по теории множеств & 23 & 15 & 65~\%\\
\hline
ИСС по числовым моделям & 49 & 18 & 37~\%\\
\hline
ИСС по истории & 32 & 14 & 44~\%\\ 
\hline
ИСС по химии & 28 & 14 & 50~\%\\
\hline
ИСС по географии & 23 & 10 & 44~\%\\
\hline
Система автоматизации предприятия & 22 & 14 & 64~\% \\
\hline
Система обслуживания розничной торговли & 17 & 8 & 47~\% \\
\hline
\end{tabularx}
\end{table}


\end{document}