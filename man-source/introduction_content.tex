% В настоящее время все более актуальным становится использование интеллектуальных систем в самых различных областях. Одним из ключевых компонентов интеллектуальной системы, обеспечивающим возможность решать широкий круг задач, является решатель задач. Особенностью решателей задач интеллектуальных систем по сравнению с другими современными программными системами является необходимость решать задачи в условиях, когда сведения, необходимые для решения задачи, не локализованы явно в базе знаний интеллектуальной системы и должны быть найдены в процессе решения задачи на основании каких-либо критериев.

% \ifx\isabstract\undefined 
% Состав решателя задач каждой конкретной системы зависит от ее назначения, классов решаемых задач, предметной области и ряда других факторов.
% \fi

% \ifx\isabstract\undefined 
% В общем случае решатель задач обеспечивает возможность решения задач, связанных как с непосредственно основной функциональностью системы, так и с обеспечением эффективности работы такой системы, а также с обеспечением автоматизации развития самой этой системы. Решатель задач, обеспечивающий выполнение всех перечисленных функций, будем называть \textit{объединенным решателем задач} указанной интеллектуальной системы.
% \fi

% %Ключевой отличительной особенностью решателей задач от современных программных систем вообще является необходимость решать задачи в условиях, когда исходные данные для задачи не локализованы явно
% %Отличительными особенностями решателей задач является их ориентация на обработку знаний, хранящихся в базе знаний интеллектуальной системы, а также 

% Расширение областей применения интеллектуальных систем требует от таких систем возможности решения комплексных задач, решение каждой из которых предполагает совместное использование целого ряда различных моделей представления знаний и различных моделей решения задач. Кроме того, решение комплексных задач предполагает использование общих информационных ресурсов (в предельном случае -- всей базы знаний интеллектуальной системы) различными компонентами решателя, ориентированными на решение различных подзадач. Поскольку решатель комплексных задач осуществляет интеграцию различных моделей решения задач, будем называть его \textit{гибридным решателем задач}.

% Примерами комплексных задач являются:

% \begin{easylist}
% &	задачи понимания текстов естественного языка (как печатных, так и рукописных), понимания речевых сообщений, изображений\ifx\isabstract\undefined. В каждом из перечисленных случаев необходимо осуществить синтаксический анализ обрабатываемого файла (сигнала), устранить незначимые фрагменты, классифицировать значимые фрагменты, соотнести их с понятиями, известными системе, выявить те фрагменты, которые система распознать не в состоянии, устранить дублирование информации и т. д.\fi;
% &	задачи автоматизации адаптивного обучения школьников и студентов\ifx\isabstract\undefined, предполагающие, что система может самостоятельно решать различные задачи из некоторой предметной области, а также управлять процессом обучения, самостоятельно формировать задания для учащихся и контролировать их выполнение\fi;
% &	задачи планирования поведения интеллектуальных роботов\ifx\isabstract\undefined, предполагающие как понимание различного рода внешней информации, так и принятие различных решений с использованием как достоверных методов, так и правдоподобных\fi;
% &	задачи комплексной и гибкой автоматизации различных предприятий;
% &	и другие.
% \end{easylist}

% Использование различных моделей решения задач в рамках
% интеллектуальной системы предполагает декомпозицию комплексной задачи на подзадачи, которые могут быть решены с помощью одной из известных интеллектуальной системе моделей решения задач. Благодаря комбинации различных моделей решения задач, множество задач, решаемых гибридным решателем, будет значительно шире, чем объединение множеств задач, решаемых по отдельности всеми решателями задач, входящими в его состав\ifx\isabstract\undefined ~\cite{Tarasov2007}\fi.

% \ifx\isabstract\undefined
% Значительный вклад в разработку и исследование моделей, методов и средств решения задач в интеллектуальных системах внесли такие ученые, как E. W. Dijkstra, C. A. Hoare, P. Jackson, J. McCarthy, A. Newell, P. Norvig, G. Polya, R. Reiter, S. Russel, H. A. Simon, D. Waterman, M. Wooldridge, А. Н. Аверкин, И. З. Батыршин, А. Н. Борисов, В. Б. Борщев, В. Н. Вагин, Т. А. Гаврилова, Л. А. Гладков, В. А. Головко, А. Н. Горбань, В. И. Городецкий, В. В. Грибова, А. П. Еремеев, Ю. А. Загорулько, А. С. Клещев, А. В. Колесников, В. Е. Котов, О. П. Кузнецов, В. М. Курейчик, Д. В. Ландэ, Л. В. Массель, А. С. Нариньяни, Г. С. Осипов, Г. С. Плесневич, Э. В. Попов, Д. А. Поспелов, Г. В. Рыбина, П. О. Скобелев, В. Б. Тарасов, Э. Х. Тыугу, В. К. Финн, И. Б. Фоминых, В. Ф. Хорошевский, А. Е. Янковская и др.
% \fi

% Постоянная эволюция интеллектуальных систем и технологий их разработки делает актуальной не только проблему снижения сроков разработки гибридных решателей задач, но и проблему снижения трудоемкости внесения изменений в состав уже разработанных решателей без необходимости изменения архитектуры всей системы в целом. 

% Современные подходы к построению гибридных решателей задач, как правило, предполагают совмещение разнородных моделей решения задач без какой-либо единой основы, например, посредством специализированных программных интерфейсов между разными компонентами системы, что приводит к существенным накладным расходам при разработке такой системы и в особенности при ее модификации, в том числе при добавлении в систему новой модели решения задач. 

% Таким образом, несмотря на успехи в области разработки решателей задач интеллектуальных систем, остаются нерешенными проблемы, связанные с обеспечением:
% \begin{easylist}
% & совместимости различных частных решателей задач, т. е. возможности их согласованного использования при решении одной и той же комплексной задачи;
% & возможности без существенных накладных расходов модифицировать гибридный решатель непосредственно в процессе эксплуатации интеллектуальной системы, в том числе расширять число используемых моделей решения задач без каких-либо ограничений на вид этих моделей. Такое требование обусловлено тем, что при решении комплексной задачи априори может оказаться неизвестным, какие именно модели решения задач и виды знаний могут потребоваться.
% \end{easylist}

% Важным способом снижения трудоемкости процесса изменения функциональности интеллектуальных систем является накопление библиотек совместимых компонентов решателей, которые позволят значительно снизить как сроки разработки и модификации решателей, так и уровень профессиональных требований к их разработчикам.

% Кроме того, актуальной является проблема создания самих средств разработки гибридных решателей задач, обеспечивающих информационную поддержку и автоматизацию деятельности разработчиков.

Практические приложения компьютерного зрения с каждым годом становятся все более разнообразными. В современном мире компьютерное зрение используется повсеместно -- в сложных производственных и медицинских системах, в интеллектуальных системах интернета вещей и развлекательных приложениях для мобильных устройств.

В качестве основы при разработке таких систем все чаще находят применение глубокие нейросетевые модели. Данные модели показывают впечатляющие результаты при решении самых разнообразных задач компьютерного зрения -- распознавания, детекции и сегментации объектов на фото- и видеоизображениях, получения аннотаций для фотографий и генерации изображений по текстовому описанию. Глубокие нейронные сети, применяемые для решения подобных задач, содержат миллионы настраиваемых параметров и, для некоторых архитектур, десятки слоев нейронных элементов.

Обучение подобных <<тяжелых>> моделей с нуля является нетривиальной задачей. Оно часто сопряжено с риском переобучения, результатом которого является отличная приспособленность сети к данным из обучающей выборки, но плохая обобщающая способность, то есть неэффективность модели для данных, не использовавшихся при обучении. Чаще всего переобученность возникает при применении малой обучающей выборки. Другой проблемой является ресурсоемкость процесса обучения таких моделей, даже при применении современных технических средств.

Проблемы обучения глубоких нейронных сетей активно изучаются в зарубежных научных школах. В нашей стране такие исследования также проводятся. Однако, нужно отметить, что такие исследования часто носят эмпирический характер, поэтому разработка строгих математически обоснованных методов обучения остается важной задачей теории нейронных сетей.

В диссертационной работе разработаны подходы для неконтролируемого предобучения глубоких нейронных сетей и редуцирования параметров моделей. Предложены алгоритмы для решения практических задач теории компьютерного зрения -- обнаружения и локализации солнечных панелей на аэрофотоснимках, обнаружения и распознавания маркировки на поточных производственных линиях. Предложенные методы и алгоритмы позволят улучшить работу интеллектуальных систем, использующих полносвязные и сверточные нейросетевые модели.
