{\actuality}
\vspace{3mm}

Тема диссертации соответствует приоритетному направлению научно-технической деятельности согласно пункту 1 перечня приоритетных направлений научной, научно-технической и инновационной деятельности на 2021-2025 годы  (Указ Президента Республики Беларусь от 07 мая 2020 г. № 156).

Исследования по теме диссертационной работы проводились в рамках научных программ:
\begin{enumerate}[wide, labelindent=10mm]
\item НИР МОРБ <<Алгоритмы интеллектуального анализа и обработки больших объемов данных на основе нейронных сетей глубокого доверия>> (ГБ 15/203, № госрегистрации 20150743),
\item ГПНИ <<Информатика и космос, научное обеспечение безопасности и защиты от чрезвычайных ситуаций>> по заданию <<Нейросетевые методы обработки комплексной информации и принятия решений на основе интеллектуальных многоагентных систем>> (№ госрегистрации 20140547),
\item ГПНИ <<Информатика и космос, научное обеспечение безопасности и защиты от чрезвычайных ситуаций>> по заданию <<Методы и алгоритмы интеллектуальной обработки и анализа большого объема данных на основе нейронных сетей глубокого доверия>> (задание 1.6.05, № госрегистрации 20163595),
\item НИР <<Методы и алгоритмы построения интеллектуальных систем анализа и обработки данных>>, этап <<Разработка гибридных интеллектуальных систем на основе нейросимволического подхода>> (решение НТС УО <<Брестский государственный технический университет>> от 12.11.2021, протокол № 6, № 22202052022070),
\item НИР БРФФИ <<Модели и исследование 3-D оцифровки на основе фактических данных и анализа гетерогенных данных>> (№ Ф22КИ-046 от 05.11.2021 г., № госрегистрации 20220090).
\end{enumerate}

% Тема диссертации соответствует приоритетному направлению <<Информатика и космические исследования>> согласно пункту 5 перечня приоритетных направлений научных исследований Республики Беларусь на 2016–2020 гг. (Постановление Совета Министров Республики Беларусь от 12 марта 2015 г. № 190).

% Диссертационное исследование выполнено в рамках следующих НИР: <<Семантическая технология компонентного проектирования интеллектуальных решателей задач>> (грант Министерства образования № 12-3134 от 27.12.2011, № ГР 20121587); <<Методы и средства онтологического моделирования для семантических технологий проектирования интеллектуальных систем>> (БРФФИ № Ф15РМ-074 от 04.05.2015 г., № ГР 20151081); <<Формализация темпоральных рассуждений в интеллектуальных системах>> (\mbox{БРФФИ № Ф16Р-102} от 20.05.2016 г., № ГР 20164340); <<Разработка методов и средств поддержки принятия решений при выявлении информационных операций>> (БРФФИ № Ф16К-068 от 21.10.2016 г., № ГР 20164717).

\vspace{3mm}
\aim
\vspace{3mm}

\textit{Целью исследования} является разработка эффективных методов и алгоритмов для обучения глубоких нейронных сетей, используемых для решения задач компьютерного зрения, включающих распознавание маркировки продукта на конвейерной линии и обнаружение солнечных панелей на аэрофотоснимках.

Указанная цель определяет следующие \textit{задачи исследования}:
\begin{easylistNum}
	& Разработать метод неконтролируемого предобучения глубоких нейронных сетей, позволяющий повысить эффективность обучения моделей;
	& Разработать алгоритм редуцирования параметров глубоких нейронных сетей, позволяющий упростить структуру моделей;
	& Провести сравнительный анализ эффективности разработанных метода обучения и алгоритма редуцирования;
	& Разработать прикладные нейросетевые системы компьютерного зрения (обнаружения солнечных панелей на аэрофотоснимках и распознавания маркировки продукта на конвейерной линии) с применением предлагаемого метода обучения.
\end{easylistNum}

% \textit{Целью исследования} является разработка комплекса моделей, методики и средств построения и модификации гибридных решателей задач интеллектуальных систем.

% Указанная цель определяет следующие {\tasks}:
% \begin{enumerate}[wide, labelindent=10mm]
%   \item	Проанализировать существующие модели, методики и средства построения решателей задач, а также модели, методы и средства повышения эффективности и модифицируемости таких решателей, в частности, в направлении расширения множества решаемых ими задач.
%   \item	Разработать агентно-ориентированную модель гибридного решателя задач, обеспечивающую интеграцию различных моделей решения задач в рамках одной интеллектуальной системы при решении комплексных задач.
%   \item	Разработать формальную модель взаимодействия информационных процессов в общей семантической памяти, включающую классификацию таких процессов, средства их синхронизации, а также принципы взаимодействия агентов, выполняющих указанные процессы.
%   \item Разработать методику построения и модификации гибридных решателей задач, ориентированных на решение задач в семантической памяти, в основе которой лежит формальное описание деятельности разработчиков таких решателей.
%   \item Разработать средства автоматизации и информационной поддержки процесса построения и модификации решателей задач.
% \end{enumerate}

\textit{Объектом исследования} являются нейросетевые системы компьютерного зрения. \textit{Предметом исследования} выступают методы и алгоритмы обучения глубоких нейронных сетей и их применение к задачам компьютерного зрения.

\vspace{3mm}
\novelty
\vspace{3mm}

Научная новизна состоит в доказательстве эквивалентности задач максимизации функции правдоподобия распределения входных данных, минимизации кросс-энтропийной функции ошибки и суммарной квадратичной ошибки при использовании линейных нейронов в пространстве синаптических связей ограниченной машины Больцмана.

Разработан метод обучения ограниченной машины Больцмана на основе доказанных эквивалентностей, применение которого для предобучения ГНС позволяет расширить класс обучаемых моделей и повысить обобщающую способность ГНС.

Разработан алгоритм редуцирования параметров нейросетевой модели, основывающийся на неконтролируемом предобучении, что позволяет уменьшить количество настраиваемых параметров модели без потери обобщающей способности.

Разработаны нейросетевые системы компьютерного зрения (система обнаружения солнечных панелей на аэрофотоснимках, система распознавания маркировки продуктов на производственной линии), которые основываются на нейросетевых моделях, предобученных предложенным методом, что позволяет повысить качество решения задач.

\vspace{3mm}
\defpositions
\vspace{3mm}

\begin{enumerate}[wide, labelindent=10mm]
\item Метод обучения ограниченной машины Больцмана, базирующийся на эквивалентности задач максимизации функции правдоподобия распределения входных данных и минимизации суммарной квадратичной ошибки при использовании линейных нейронов в пространстве синаптических связей сети, что позволяет расширить класс обучаемых моделей и повысить обобщающую способность глубоких нейронных сетей.
\item Алгоритм редуцирования параметров глубокой нейронной сети, базирующийся на неконтролируемом предобучении сети, что позволяет упростить ее архитектуру (путем сокращения числа настраиваемых параметров модели) без потери обобщающей способности.
\item Нейросетевые системы компьютерного зрения, базирующиеся на использовании предобученных моделей -- система обнаружения солнечных панелей на аэрофотоснимках (включая снимки низкого разрешения) с точностью до 87,46\% и система распознавания (в реальном режиме времени) маркировки продукта на производственной линии, что позволяет повысить качество решения задач. 
% \item Агентно-ориентированная модель гибридного решателя задач, рассматривающая каждый такой решатель как иерархическую систему агентов, управляемых ситуациями и событиями в общей семантической памяти, обеспечивающая модифицируемость таких решателей задач, а также возможность решения задач, требующих совместного использования различных методов решения задач.
% \item Формальная модель взаимодействия параллельных асинхронных информационных процессов в общей семантической памяти, определяющая принцип коммуникации агентов, выполняющих указанные процессы, включающая средства синхронизации процессов на основе механизма блокировок элементов семантической памяти и обеспечивающая возможность спецификации планируемых блокировок, а также выявления и устранения взаимоблокировок.
% \item Методика построения и модификации гибридных решателей задач, основанная на формальной онтологии деятельности разработчиков таких решателей и ориентированная на применение многократно используемых компонентов решателей, позволяющая снизить сроки разработки решателей как минимум на 31~\% за счет использования разработанных ранее компонентов.
% \item Средства автоматизации и информационной поддержки процесса построения и модификации гибридных решателей задач, включающие средства автоматизации процесса построения агентов обработки знаний и библиотеку многократно используемых компонентов таких решателей, которая в текущий момент позволяет сократить число агентов, разрабатываемых для каждого решателя, как минимум на 37~\%.
\end{enumerate}

%Личный вклад
\vspace{3mm}
\contribution
\vspace{3mm}

Основные положения диссертации получены соискателем лично. Соавтором основных публикаций автора является научный руководитель д.т.н., профессор В.А. Головко, который осуществлял определение целей и постановку задач исследований, выбор методов исследований, принимал участие в планировании работ и обсуждении результатов. В диссертационную работу не включены результаты, которые были получены другими соавторами или с другими соавторами. Материалы совместных публикаций использованы соискателем в объеме авторского вклада.

\vspace{3mm}
\probation
\vspace{3mm}

Основные положения и результаты диссертационной работы докладывались и обсуждались на следующих международных конференциях: <<International Conference on Neural Networks and Artificial Intelligence>> (Брест, 2014); <<Информационное, программное и техническое обеспечение систем управления организационно-технологическими комплексами>> (Луцк, 2015); <<8th IEEE International Conference on Intelligent Data Acquisition and Advanced Computing Systems: Technology and Applications>> (Варшава, 2015); <<International Scientific-Practical Conference Problems of Infocommunications. Science and Technology (PIC S\&T)>> (2018); <<International Conference on Pattern Recognition and Information Processing (PRIP)>> (2019); <<Вычислительные методы, модели и образовательные технологии>> (Брест, 2013, 2014, 2015, 2016, 2019); <<Современные проблемы математики и вычислительной техники>> (Брест, 2015); <<Открытые семантические технологии проектирования интеллектуальных систем>> (Минск, 2015, 2018, 2019, 2020, 2021, 2023); <<11th IEEE International Conference on Intelligent Data Acquisition and Advanced Computing Systems: Technology and Applications (IDAACS)>> (Краков, 2021).

%Научная и практическая значимость
%\influence\ \ldots

%Степень достоверности
%\reliability\ полученных результатов обеспечивается \ldots \ Результаты находятся в соответствии с результатами, полученными другими авторами.

%Публикации
\vspace{3mm}
\publications
\vspace{3mm}

По теме диссертационного исследования опубликовано 30 научных работ, среди которых: 14 статей в научных изданиях в соответствии с пунктом 19 Положения о присуждении ученых степеней и присвоении ученых званий (9,39 авторских листа), 3 статей в других научных изданиях, 9 статей в сборниках материалов научных конференций и 4 тезиса.

% По материалам выполненных исследований опубликовано 37 научных ра-
% бот, в том числе 7 статей в рецензируемых изданиях. Без соавторства опубликовано 10 работ, из них 1 статья в рецензируемых изданиях.
% Общий объем публикаций по теме диссертации, соответствующий пункту
% 18 Положения о присуждении ученых степеней и присвоении ученых званий
% в Республике Беларусь, составляет около 18,88 авторского листа.

%По материалам диссертационного исследования опубликовано 37 научных работ, в том числе 7 статей в рецензируемых научных изданиях, соответствующих пункту 18 Положения о присуждении ученых степеней и присвоении ученых званий в Республике Беларусь, общим объемом около 3,5 авт. листа, 3 статьи в научных журналах, 27 статей в материалах республиканских и международных конференций. Без соавторства опубликовано 10 работ, из них 1 статья в рецензируемых изданиях.