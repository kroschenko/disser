\chapter*{Перечень сокращений и обозначений}
\addcontentsline{toc}{chapter}{Перечень сокращений и обозначений}
%\sectioncentered*{Перечень условных обозначений}
%\addcontentsline{toc}{section}{Перечень условных обозначений}
%\label{sec:reduction}

\noindent\textit{ГНС} -- глубокая нейронная сеть\\
\textit{ИНС} -- искусственная нейронная сеть\\
\textit{КЗ} -- компьютерное зрение\\
\textit{НС} -- нейронная сеть\\
\textit{СНС} -- сверточная нейронная сеть\\
\textit{AP (англ. Average Precision metric)} -- метрика, используемая для оценки методов детекции объектов на изображениях, применяется для одноклассовых случаев\\
\textit{CD (англ. Contrastive Divergence)} -- контрастное расхождение\\
\textit{CE (англ. Cross-Entropy error)} -- кросс-энтропийная функция ошибок\\
\textit{CIFAR (англ. Canadian Institute For Advanced Research)} -- Канадский институт перспективных исследований\\
\textit{COCO (англ. Common Objects in Context)} -- выборка размеченных изображений, применяемая для решения задач обнаружения, сегментации и аннотирования объектов на изображениях\\
\textit{CV (англ. Computer Vision)} -- компьютерное зрение\\
\textit{C-RBM (англ. Classic Restricted Boltzmann Machine Training)} -- классический метод обучения ограниченной машины Больцмана\\
\textit{DNN (англ. Deep Neural Network)} -- глубокая нейронная сеть\\
\textit{IoU (англ. Intersection over Union, мера Жаккара)} -- метрика, применяемая для оценки эквивалентности двух множеств, в задачах компьютерного зрения применяется для оценки методов детекции (обнаружения) объектов как величина, описывающая степень перекрытия двух прямоугольных областей, используемых для определения эталонной и предсказываемой моделью локализации объекта \\
\textit{HREBA (англ. Hybrid Reconstruction Error-Based Approach)} -- гибридный вариант предобучения ГНС, сочетающий в себе использование классического (C-RBM) и предлагаемого в данной работе (REBA) методов обучения ограниченной машины Больцмана\\
\textit{mAP (англ. mean-Average Precision metric)} -- метрика, применяемая для оценки качества алгоритмов детекции объектов на изображениях, применяется для многоклассовых случаев детекции\\
\textit{MNIST (англ. Mixed National Institute of Standards and Technology database)} -- смешанная база Национального института стандартов и технологии\\
\textit{MSE (англ. Mean Squared Error)} -- среднеквадратичная ошибка\\
\textit{NLP (англ. Natural Language Processing)} -- обработка естественного языка\\
\textit{Faster-RCNN (англ. Faster Region-based Convolutional Neural Network)} -- архитектура глубокой сверточной нейронной сети, применяемая для решения задачи детекции (обнаружения) объектов на изображениях\\
\textit{PCA (англ. Principal Component Analysis)} -- метод главных компонент\\
\textit{RBM (англ. Restricted Boltzmann Machine)} -- ограниченная машина Больцмана\\
\textit{REBA (англ. Reconstruction Error-Based Approach)} -- метод обучения ограниченной машины Больцмана, базирующийся на минимизации ошибки восстановления образов на видимом и скрытом слоях\\
\textit{ReLU (англ. Rectified Linear Unit)} -- исправленный линейный элемент\\
\textit{ResNet-50 (англ. Residual Neural Network)} -- архитектура глубокой нейронной сети, которая характеризуется передачей сигналов между отдельными слоями модели, применяется в качестве классификатора\\
\textit{SVM (англ. Support Vector Machines)} -- метод опорных векторов\\
\textit{SVHN (англ. The Street View House Numbers)} -- выборка изображений номеров домов

% ACL (англ. Agent Communication Language) -- язык взаимодействия агентов, предложенный FIPA в качестве стандарта

% FIPA (англ. Foundation for Intelligent Physical Agents) -- организация, осуществляющая разработку и продвижение стандартов в области многоагентных систем

% GPS (англ. General Problem Solver) -- компьютерная программа, созданная в 1959 г. и предназначенная для работы в качестве универсальной машины для решения задач, сформулированных на языке хорновских дизъюнктов

% IACPaaS (англ. Intelligent Applications, Control and Platform as a Service) -- исследовательская облачная платформа, объединяющая различные модели парадигмы облачных вычислений

% IMS (англ. Intelligent MetaSystem) -- интеллектуальная метасистема поддержки проектирования интеллектуальных систем

% KIF (англ. Knowledge Interchange Language) -- компьютерно-ориентированный язык для обмена знаниями между различными компьютерными программами

% KQML (англ. Knowledge Query and Manipulation Language) -- язык взаимодействия между программными агентами и системами, основанными на знаниях

% OSTIS (англ. Open Semantic Technology for Intelligent Systems) -- открытая семантическая технология проектирования интеллектуальных систем

% OWL (англ. Web Ontology Language) -- язык описания онтологий для семантической паутины

% QA3 (англ. Question Answer system ver. 3) -- вопросно-ответная дедуктивная система, созданная в 1969 г. на языке LISP

% RDF (англ. Resource Description Framework) -- разработанная Консорциумом Всемирной паутины модель для представления данных

% SCg-код (англ. Semantic Code graphic) -- графический нелинейный вариант визуализации текстов SC-кода

% SCn-код (англ. Semantic Code natural) -- гипертекстовый вариант визуализации текстов SC-кода

% SCP (англ. Semantic Code Programming) -- графовый процедурный язык программирования, построенный на базе SC-кода

% SC-код (англ. Semantic Code) -- универсальный базовый способ смыслового представления знаний в виде семантических сетей с базовой теоретико-множественной интерпретацией

% SPARQL (англ. SPARQL Protocol and RDF Query Language) -- язык запросов к данным (является рекомендацией консорциума W3C и одной из технологий семантической паутины), представленным по модели RDF, а также протокол для передачи этих запросов и ответов на них

% SQL (англ. Structured Query Language — «язык структурированных запросов») —- универсальный язык запросов, применяемый для создания, модификации и управления данными в реляционных базах данных

% STRIPS (англ. Stanford Research Institute Problem Solver) -- планирующая система, использующая декларативно-процедуральное представление знаний в сочетании с эвристическим поиском, создана в 1971 г.

% W3C (англ. World Wide Web Consortium, W3C) -- Консорциум Всемирной паутины, организация, разрабатывающая и внедряющая технологические стандарты для Всемирной паутины

% ГРЗ -- гибридный решатель задач

% ИСС -- интеллектуальная справочная система

% НИЦ ЭВТ -- московский Научно-исследовательский центр электронной вычислительной техники

% ППР (Программа принятия решений) -- планирующая система для интеллектуального робота, созданная в 1977 г. под руководством В. П. Гладуна

% ПРИЗ (Пакет прикладных инженерных задач) -- система программирования, созданная под руководством Э. Х. Тыугу в 1970--1976 гг

% РЗ -- решатель задач

% УСК -- универсальный семантический код, разработанный В. В. Мартыновым

