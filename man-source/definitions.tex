%\usepackage[cp1251]{inputenc}
\usepackage[utf8]{inputenc}
\usepackage[T2A]{fontenc}
\usepackage[russian]{babel}

%\usepackage{pscyr}
%\renewcommand{\rmdefault}{ftm}

%для борьбы с переполнениями за счет разреженных слов в абзаце
\emergencystretch=25pt
%math
\usepackage{mathtools}
\usepackage{tabularx}
\usepackage{pdflscape}

\usepackage{afterpage}
\usepackage{geometry}
\usepackage{multirow}
\usepackage{enumitem}

\usepackage{amsmath,amssymb,amsfonts}
\usepackage{longtable,array}
\usepackage{graphicx,epsfig}
\usepackage[unicode,colorlinks=false,pagebackref=false, hidelinks]{hyperref}
\urlstyle{same}
%\usepackage{refcheck}% checks lost and useless labels, shows `keys' of \label in the margins
%фиксить картинки
%\usepackage{float}

\usepackage{textcomp}% For Celsium sign only

% Листинги с исходным кодом программ
\usepackage{fancyvrb}
\usepackage{listings}
\usepackage{totcount}
%\usepackage{totpages}
\usepackage{cite}

% Плавающие окружения. во многом лучше пакета float
\usepackage{floatrow}

\usepackage{tikz}
\newcolumntype{P}[1]{>{\centering\arraybackslash}p{#1}}
\newcolumntype{M}[1]{>{\centering\arraybackslash}m{#1}}

%for lists
\usepackage[ampersand]{easylist}
\ListProperties(Hide=100, Hang=false, Margin=0mm, Indent1=10.5mm, Indent2=15mm, Style*=-- ,
Style2*=$\bullet$ ,Style3*=$\circ$ ,Style4*=\tiny$\blacksquare$ )

\usepackage[ruled, vlined]{algorithm2e}
\makeatletter
\newenvironment{algo}[1][]
  {\renewcommand{\algorithmcfname}{Алгоритм}%
   \begin{algorithm}[#1]
   \long\def\@caption##1[##2]##3{%
     \par
     \begingroup\@parboxrestore
     \if@minipage\@setminipage\fi
     \normalsize \@makecaption{\AlCapSty{\AlCapFnt\algorithmcfname}}{\ignorespaces ##3}%
     \par\endgroup
   }}
  {\end{algorithm}}
\makeatother

% \SetKwData{KwDat}{Данные}
\SetKwInput{KwIn}{Исходные данные}
\SetKwInput{KwRes}{Результат}%
% \SetKwIF{Si}{SinonSi}{Sinon}{si}{alors}{sinon si}{sinon}{fin si}%
% \SetKwFor{Tq}{tant que}{faire}{fin tq}%

\newenvironment{easylistNum}{\begin{easylist}
\ListProperties(Hide1=0, Hang=false, Margin=0mm, Indent1=10.5mm, Indent2=15mm, Start1=1, Style*=, FinalMark={)})}{\ListProperties(Hide=100, Hang=false, Margin=0mm, Indent1=10.5mm, Indent2=15mm, Style*=-- ,
Style2*=$\bullet$ ,Style3*=$\circ$ ,Style4*=\tiny$\blacksquare$ )\end{easylist}}

\renewcommand\labelitemi{\textbf{--}}
\floatsetup[table]{capposition=top}

    % \lstset{
    %     language=Python,
    %     basicstyle=\ttfamily,
    %     keywordstyle=\bfseries,
    %     showspaces=false,
    %     showstringspaces=false,
    %     mathescape=true,
    %     aboveskip=0pt,
    %     belowskip=6pt
    % }
\lstdefinestyle{PythonStyle}{
  keywordstyle=\bf,
  belowcaptionskip=1\baselineskip,
  breaklines=true,
  language=Python,
  showstringspaces=false,
  basicstyle=\small\ttfamily,
  commentstyle=\itshape
}
% \lstdefinestyle{PythonStyle}{
% basicstyle=\footnotesize\ttfamily,
% language=Python,
% keywordstyle=\bfseries,
% showstringspaces=false,
% commentstyle={},
% texcl=true
% }

\graphicspath{{../}}

%\renewcommand{\cftchapleader}{\cftdotfill{\cftdotsep}}
