\newcommand{\actuality}{\textbf{Связь работы с научными программами (проектами), темами}}
\newcommand{\aim}{\textbf{Цель, задачи, объект и предмет исследования}}
%\newcommand{\tasks}{\textit{задачи исследования}}
\newcommand{\novelty}{\textbf{Научная новизна}}
\newcommand{\defpositions}{\textbf{Положения, выносимые на защиту}}
\newcommand{\influence}{\textbf{Научная и практическая значимость}}
\newcommand{\reliability}{\textbf{Степень достоверности}}
\newcommand{\contribution}{\textbf{Личный вклад соискателя ученой степени в результаты диссертации}}
\newcommand{\probation}{\textbf{Апробация диссертации и информация об использовании ее результатов}}
\newcommand{\publications}{\textbf{Опубликованность результатов диссертации}}

{\actuality}
\vspace{3mm}

Тема диссертации соответствует приоритетному направлению научно-технической деятельности согласно пункту 1 перечня приоритетных направлений научной, научно-технической и инновационной деятельности на 2021-2025 годы  (Указ Президента Республики Беларусь от 07 мая 2020 г. № 156).

Исследования по теме диссертационной работы проводились в рамках научных программ:
\begin{enumerate}[wide, labelindent=10mm]
\item НИР МОРБ <<Алгоритмы интеллектуального анализа и обработки больших объемов данных на основе нейронных сетей глубокого доверия>> (ГБ 15/203, № госрегистрации 20150743),
\item ГПНИ <<Информатика и космос, научное обеспечение безопасности и защиты от чрезвычайных ситуаций>> по заданию <<Нейросетевые методы обработки комплексной информации и принятия решений на основе интеллектуальных многоагентных систем>> (№ госрегистрации 20140547),
\item ГПНИ <<Информатика и космос, научное обеспечение безопасности и защиты от чрезвычайных ситуаций>> по заданию <<Методы и алгоритмы интеллектуальной обработки и анализа большого объема данных на основе нейронных сетей глубокого доверия>> (задание 1.6.05, № госрегистрации 20163595),
\item НИР <<Методы и алгоритмы построения интеллектуальных систем анализа и обработки данных>>, этап <<Разработка гибридных интеллектуальных систем на основе нейросимволического подхода>>,
\item НИР БРФФИ <<Модели и исследование 3-D оцифровки на основе фактических данных и анализа гетерогенных данных>> (№ Ф22КИ-046 от 05.11.2021 г., № госрегистрации 20220090).
\end{enumerate}

% Тема диссертации соответствует приоритетному направлению <<Информатика и космические исследования>> согласно пункту 5 перечня приоритетных направлений научных исследований Республики Беларусь на 2016–2020 гг. (Постановление Совета Министров Республики Беларусь от 12 марта 2015 г. № 190).

% Диссертационное исследование выполнено в рамках следующих НИР: <<Семантическая технология компонентного проектирования интеллектуальных решателей задач>> (грант Министерства образования № 12-3134 от 27.12.2011, № ГР 20121587); <<Методы и средства онтологического моделирования для семантических технологий проектирования интеллектуальных систем>> (БРФФИ № Ф15РМ-074 от 04.05.2015 г., № ГР 20151081); <<Формализация темпоральных рассуждений в интеллектуальных системах>> (\mbox{БРФФИ № Ф16Р-102} от 20.05.2016 г., № ГР 20164340); <<Разработка методов и средств поддержки принятия решений при выявлении информационных операций>> (БРФФИ № Ф16К-068 от 21.10.2016 г., № ГР 20164717).

\vspace{3mm}
\aim
\vspace{3mm}

\textit{Целью исследования} является разработка эффективных методов и алгоритмов для обучения глубоких нейронных сетей, используемых для решения задач компьютерного зрения, включающих распознавание маркировки продукта на конвейерной линии и обнаружение солнечных панелей на аэрофотоснимках.

Указанная цель определяет следующие \textit{задачи исследования}:
\begin{easylistNum}
	& Разработать метод неконтролируемого предобучения глубоких нейронных сетей, позволяющий повысить эффективность обучения моделей;
	& Разработать алгоритм редуцирования параметров глубоких нейронных сетей, позволяющий упростить структуру моделей;
	& Провести сравнительный анализ эффективности разработанных метода обучения и алгоритма редуцирования;
	& Разработать нейросетевые системы распознавания маркировки продукта на конвейерной линии и обнаружения солнечных панелей на аэрофотоснимках.
\end{easylistNum}

% \textit{Целью исследования} является разработка комплекса моделей, методики и средств построения и модификации гибридных решателей задач интеллектуальных систем.

% Указанная цель определяет следующие {\tasks}:
% \begin{enumerate}[wide, labelindent=10mm]
%   \item	Проанализировать существующие модели, методики и средства построения решателей задач, а также модели, методы и средства повышения эффективности и модифицируемости таких решателей, в частности, в направлении расширения множества решаемых ими задач.
%   \item	Разработать агентно-ориентированную модель гибридного решателя задач, обеспечивающую интеграцию различных моделей решения задач в рамках одной интеллектуальной системы при решении комплексных задач.
%   \item	Разработать формальную модель взаимодействия информационных процессов в общей семантической памяти, включающую классификацию таких процессов, средства их синхронизации, а также принципы взаимодействия агентов, выполняющих указанные процессы.
%   \item Разработать методику построения и модификации гибридных решателей задач, ориентированных на решение задач в семантической памяти, в основе которой лежит формальное описание деятельности разработчиков таких решателей.
%   \item Разработать средства автоматизации и информационной поддержки процесса построения и модификации решателей задач.
% \end{enumerate}

\textit{Объектом исследования} являются нейросетевые системы компьютерного зрения. \textit{Предметом исследования} выступают методы и алгоритмы обучения глубоких нейронных сетей и их применение к задачам компьютерного зрения.

\vspace{3mm}
\novelty
\vspace{3mm}

Научная новизна состоит в выявлении и доказательстве эквивалентности максимизации функции правдоподобия распределения входных данных, минимизации кросс-энтропийной функции ошибки и суммарной квадратичной ошибки при использовании линейных нейронов в пространстве синаптических связей ограниченной машины Больцмана.

Разработан метод обучения ограниченной машины Больцмана на основе доказанных эквивалентностей, применение которого для предобучения ГНС позволяет расширить класс обучаемых моделей и повысить обобщающую способность ГНС.

Разработан алгоритм редуцирования параметров нейросетевой модели, основывающийся на неконтролируемом предобучении, что позволяет уменьшить количество настраиваемых параметров модели без потери обобщающей способности.

Разработаны нейросетевые системы компьютерного зрения (система распознавания маркировки на производственной линии, система обнаружения солнечных панелей на аэрофотоснимках), которые основываются на применении предложенного метода для выполнения предобучения соответствующих нейросетевых моделей.

\vspace{3mm}
\defpositions
\vspace{3mm}

\begin{enumerate}[wide, labelindent=10mm]
\item Метод обучения ограниченной машины Больцмана, базирующийся на эквивалентности максимизации функции правдоподобия распределения входных данных в пространстве синаптических связей и минимизации суммарной квадратичной ошибки сети при использовании линейных нейронов, что позволяет расширить класс обучаемых моделей и повысить обобщающую способность глубоких нейронных сетей.
\item Алгоритм редуцирования параметров глубокой нейронной сети, базирующийся на неконтролируемом предобучении сети, что позволяет упростить ее архитектуру (путем сокращения числа настраиваемых параметров модели) без потери обобщающей способности.
\item Нейросетевые системы компьютерного зрения, базирующиеся на использовании предобученных моделей -- система распознавания (в реальном режиме времени) маркировки продукта на производственной линии и система обнаружения солнечных панелей на аэрофотоснимках (включая снимки низкого разрешения) с точностью до 87,46\%.

% \item Агентно-ориентированная модель гибридного решателя задач, рассматривающая каждый такой решатель как иерархическую систему агентов, управляемых ситуациями и событиями в общей семантической памяти, обеспечивающая модифицируемость таких решателей задач, а также возможность решения задач, требующих совместного использования различных методов решения задач.
% \item Формальная модель взаимодействия параллельных асинхронных информационных процессов в общей семантической памяти, определяющая принцип коммуникации агентов, выполняющих указанные процессы, включающая средства синхронизации процессов на основе механизма блокировок элементов семантической памяти и обеспечивающая возможность спецификации планируемых блокировок, а также выявления и устранения взаимоблокировок.
% \item Методика построения и модификации гибридных решателей задач, основанная на формальной онтологии деятельности разработчиков таких решателей и ориентированная на применение многократно используемых компонентов решателей, позволяющая снизить сроки разработки решателей как минимум на 31~\% за счет использования разработанных ранее компонентов.
% \item Средства автоматизации и информационной поддержки процесса построения и модификации гибридных решателей задач, включающие средства автоматизации процесса построения агентов обработки знаний и библиотеку многократно используемых компонентов таких решателей, которая в текущий момент позволяет сократить число агентов, разрабатываемых для каждого решателя, как минимум на 37~\%.
\end{enumerate}

%Личный вклад
\vspace{3mm}
\contribution
\vspace{3mm}

Основные положения диссертации получены автором лично. Из публикаций, сделанных в соавторстве, в содержание диссертации включены только те результаты, которые были получены лично соискателем. Соавтором основных публикаций автора является научный руководитель д.т.н., профессор В.А. Головко, который осуществлял определение целей и постановку задач исследований, выбор методов исследований, принимал участие в планировании работ и обсуждении результатов. 

\vspace{3mm}
\probation
\vspace{3mm}

Основные положения и результаты диссертационной работы докладывались и обсуждались на следующих международных конференциях: <<International Conference on Neural Networks and Artificial Intelligence>> (Брест, 2014); <<Информационное, программное и техническое обеспечение систем управления организационно-технологическими комплексами>> (Луцк, 2015); <<8th IEEE International Conference on Intelligent Data Acquisition and Advanced Computing Systems: Technology and Applications>> (Варшава, 2015); <<International Scientific-Practical Conference Problems of Infocommunications. Science and Technology (PIC S\&T)>> (2018); <<International Conference on Pattern Recognition and Information Processing (PRIP)>> (2019); <<Вычислительные методы, модели и образовательные технологии>> (Брест, 2013, 2014, 2015, 2016, 2019); <<Современные проблемы математики и вычислительной техники>> (Брест, 2015); <<Открытые семантические технологии проектирования интеллектуальных систем>> (Минск, 2018, 2019, 2020, 2021, 2023); <<11th IEEE International Conference on Intelligent Data Acquisition and Advanced Computing Systems: Technology and Applications (IDAACS)>> (Краков, 2021).

%Научная и практическая значимость
%\influence\ \ldots

%Степень достоверности
%\reliability\ полученных результатов обеспечивается \ldots \ Результаты находятся в соответствии с результатами, полученными другими авторами.

%Публикации
\vspace{3mm}
\publications
\vspace{3mm}

По теме диссертационной работы опубликовано 28 печатных работ (\textbf{9,48} авторских листа). Из них 11 статей в научных журналах (5,35 авторских листа), 5 статей в сборниках материалов научных конференций, включенных в системы международного цитирования (Scopus, Web of Science и IEEE Xplore digital, 0,48 авторских листа) и 12 статей в сборниках материалов научных конференций (\textbf{3,65} авторских листа).

% По материалам выполненных исследований опубликовано 37 научных ра-
% бот, в том числе 7 статей в рецензируемых изданиях. Без соавторства опубликовано 10 работ, из них 1 статья в рецензируемых изданиях.
% Общий объем публикаций по теме диссертации, соответствующий пункту
% 18 Положения о присуждении ученых степеней и присвоении ученых званий
% в Республике Беларусь, составляет около 18,88 авторского листа.

%По материалам диссертационного исследования опубликовано 37 научных работ, в том числе 7 статей в рецензируемых научных изданиях, соответствующих пункту 18 Положения о присуждении ученых степеней и присвоении ученых званий в Республике Беларусь, общим объемом около 3,5 авт. листа, 3 статьи в научных журналах, 27 статей в материалах республиканских и международных конференций. Без соавторства опубликовано 10 работ, из них 1 статья в рецензируемых изданиях. % Характеристика работы по структуре во введении и в автореферате не отличается (ГОСТ Р 7.0.11, пункты 5.3.1 и 9.2.1), потому её загружаем из одного и того же внешнего файла, предварительно задав форму выделения некоторым параметрам
%% регистрируем счётчики в системе totcounter
% \regtotcounter{totalcount@figure}
% \regtotcounter{totalcount@table}       % Если поставить в преамбуле то ошибка в числе таблиц
% \regtotcounter{TotPages}               % Если поставить в преамбуле то ошибка в числе страниц

%\regtotcounter{page}
%\newtotcounter{mainpage}

% \addtocounter{page}{-1}
%\addtocounter{mainpage}{-1}

\vspace{3mm}
\textbf{Структура и объем диссертации} 
\vspace{3mm}

Диссертация состоит из введения, общей характеристики работы, четырех глав с краткими выводами по каждой главе, заключения, библиографического списка, списка публикаций автора и приложений.

В \textbf{\textit{первой главе}} рассмотрено краткое введение в теорию искусственных нейронных сетей, дана классификация их типов. Определено понятие глубокой нейронной сети. Рассмотрены задачи обучения глубоких нейронных сетей и существующие методы обучения. \textbf{\textit{Вторая глава}} посвящена рассмотрению разработанного метода обучения ограниченной машины Больцмана и метода предобучения глубоких нейронных сетей на его основе. Также в этой главе предлагается алгоритм редуцирования параметров нейросетевых моделей. В \textbf{\textit{третьей главе}} приведены экспериментальные результаты, обосновывающие полученные теоретические результаты. В \textbf{\textit{четвертой главе}} рассмотрено практическое применение разработанных методов в интеллектуальных системах компьютерного зрения.

Общий объем диссертации составляет 116 страниц, из которых 91 страниц основного текста, 44 рисунка на 32 страницах, 17 таблиц на 9 страницах, библиография из 120 источников, включая 30 публикаций автора, приложения на 14 страницах.

% Диссертация состоит из введения, общей характеристики работы, четырех глав с краткими выводами по каждой главе, заключения, библиографического списка, списка публикаций автора и 17 приложений. Общий объём диссертации составляет 254 страницы, из которых 157 страниц основного текста, 51 рисунок на 26 страницах, 5 таблиц на 4 страницах, библиография из 167 источников, включая 37 публикаций автора, приложения на 50 страницах.
%\total{page} 
%\total{mainpage}

%\textcolor{red}{
%Диссертация состоит из введения, общей характеристики работы, четырех глав с краткими выводами по каждой главе, заключения, библиографического списка, списка публикаций автора и 11 приложений. Общий объём диссертации составляет \formbytotal{TotPages}{страниц}{у}{ы}{}, из которых 171 страниц основного текста, \formbytotal{totalcount@figure}{рисунк}{ом}{ами}{ами}, \formbytotal{totalcount@table}{таблиц}{ей}{ами}{ами}, библиография из 174 источников, включая 56 публикаций автора, приложения на 37 страницах.}

%\textcolor{red}{
%Диссертация состоит из введения, общей характеристики работы, четырех глав с краткими выводами по каждой главе, заключения, библиографического списка, списка публикаций автора и 7 приложений. В первой главе проведен анализ моделей, средств разработки баз знаний и методов их создания. Во второй главе предложена семантическая модель базы знаний. Третья глава содержит описание унифицированной семантической модели процесса создания баз знаний, а также унифицированной семантической модели средств автоматизации процесса создания баз знаний, включая унифицированную семантическую модель библиотеки многократно используемых компонентов баз знаний. В четвертой главе описана реализация разработанных моделей и средств, проведена их оценка.
%% на случай ошибок оставляю исходный кусок на месте, закомментированным
% Полный объём диссертации составляет  \ref*{TotPages}~страницу с~\totalfigures{totalcount@figure}~рисунками и~\totaltables{}~таблицами. Список литературы содержит \total{citenum}~наименований.}

%
%\textcolor{red}{
%Полный объём диссертации составляет
%\formbytotal{TotPages}{страниц}{у}{ы}{} 
%с~\formbytotal{totalcount@figure}{рисунк}{ом}{ами}{ами}
%и~\formbytotal{totalcount@table}{таблиц}{ей}{ами}{ами}. Список литературы %содержит  
%\formbytotal{citenum}{наименован}{ие}{ия}{ий}.}