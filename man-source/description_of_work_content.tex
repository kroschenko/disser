\newcommand{\actuality}{\textbf{Связь работы с научными программами (проектами), темами}}
\newcommand{\aim}{\textbf{Цель, задачи, объект и предмет исследования}}
%\newcommand{\tasks}{\textit{задачи исследования}}
\newcommand{\novelty}{\textbf{Научная новизна}}
\newcommand{\defpositions}{\textbf{Положения, выносимые на защиту}}
\newcommand{\influence}{\textbf{Научная и практическая значимость}}
\newcommand{\reliability}{\textbf{Степень достоверности}}
\newcommand{\contribution}{\textbf{Личный вклад соискателя ученой степени в результаты диссертации}}
\newcommand{\probation}{\textbf{Апробация диссертации и информация об использовании ее результатов}}
\newcommand{\publications}{\textbf{Опубликованность результатов диссертации}}

{\actuality}
\vspace{3mm}

Тема диссертации соответствует приоритетному направлению научно-технической деятельности согласно пункту 1 перечня приоритетных направлений научной, научно-технической и инновационной деятельности на 2021-2025 годы  (Указ Президента Республики Беларусь от 07 мая 2020 г. № 156).

Исследования по теме диссертационной работы проводились в рамках научных программ:
\begin{enumerate}[wide, labelindent=10mm]
\item НИР МОРБ <<Алгоритмы интеллектуального анализа и обработки больших объемов данных на основе нейронных сетей глубокого доверия>> (ГБ 15/203, № госрегистрации 20150743),
\item ГПНИ <<Информатика и космос, научное обеспечение безопасности и защиты от чрезвычайных ситуаций>> по заданию <<Нейросетевые методы обработки комплексной информации и принятия решений на основе интеллектуальных многоагентных систем>> (№ госрегистрации 20140547),
\item ГПНИ <<Информатика и космос, научное обеспечение безопасности и защиты от чрезвычайных ситуаций>> по заданию <<Методы и алгоритмы интеллектуальной обработки и анализа большого объема данных на основе нейронных сетей глубокого доверия>> (задание 1.6.05, № госрегистрации 20163595),
\item НИР <<Методы и алгоритмы построения интеллектуальных систем анализа и обработки данных>>, этап <<Разработка гибридных интеллектуальных систем на основе нейросимволического подхода>> (решение НТС УО <<Брестский государственный технический университет>> от 12.11.2021, протокол № 6, № 22202052022070),
\item НИР БРФФИ <<Модели и исследование 3-D оцифровки на основе фактических данных и анализа гетерогенных данных>> (№ Ф22КИ-046 от 05.11.2021 г., № госрегистрации 20220090).
\end{enumerate}

\vspace{3mm}
\aim
\vspace{3mm}

\textit{Целью исследования} является разработка эффективных методов и алгоритмов для обучения глубоких нейронных сетей, используемых для решения задач компьютерного зрения, включающих обнаружение солнечных панелей на аэрофотоснимках и распознавание маркировки продукта на конвейерной линии.

Указанная цель определяет следующие \textit{задачи исследования}:
\begin{easylistNum}
	& Разработать метод неконтролируемого предобучения глубоких нейронных сетей, позволяющий повысить эффективность обучения моделей;
	& Разработать алгоритм редуцирования параметров глубоких нейронных сетей, позволяющий упростить структуру моделей;
	& Провести сравнительный анализ эффективности разработанных метода обучения и алгоритма редуцирования;
	& Разработать нейросетевую систему компьютерного зрения для решения прикладных задач (обнаружения солнечных панелей на аэрофотоснимках и распознавания маркировки продукта на конвейерной линии) с применением предлагаемого метода обучения.
\end{easylistNum}

\textit{Объектом исследования} являются нейросетевые системы компьютерного зрения. \textit{Предметом исследования} выступают методы и алгоритмы обучения глубоких нейронных сетей и их применение к задачам компьютерного зрения.

\vspace{3mm}
\novelty
\vspace{3mm}

Научная новизна состоит в установлении эквивалентности задач максимизации функции правдоподобия распределения входных данных, минимизации кросс-энтропийной функции ошибки и суммарной квадратичной ошибки при использовании линейных нейронов в пространстве синаптических связей ограниченной машины Больцмана, что позволяет учитывать нелинейную природу нейронных элементов.

Разработан метод обучения ограниченной машины Больцмана на основе доказанных эквивалентностей, применение которого для предобучения ГНС позволяет расширить класс обучаемых моделей и повысить обобщающую способность ГНС.

Разработан алгоритм редуцирования параметров нейросетевой модели, основывающийся на неконтролируемом предобучении, что позволяет уменьшить количество настраиваемых параметров модели без потери обобщающей способности.

Разработана нейросетевая система компьютерного зрения, которая основана на предлагаемом методе предобучения нейросетевых моделей, применение которой позволяет улучшить качество решения задач классификации. Продемонстрирована эффективность системы на примере решения задач обнаружения солнечных панелей на аэрофотоснимках и распознавания маркировки продуктов на производственной линии.

\vspace{10mm}
\defpositions
\vspace{3mm}

\begin{enumerate}[wide, labelindent=10mm]
	\item Установление эквивалентности задач максимизации функции правдоподобия распределения входных данных, минимизации кросс-энтропийной функции ошибки и суммарной квадратичной ошибки при использовании линейных нейронов в пространстве синаптических связей ограниченной машины Больцмана, что позволяет учитывать нелинейную природу нейронных элементов.
	\item Метод обучения ограниченной машины Больцмана, базирующийся на эквивалентности задач максимизации функции правдоподобия распределения входных данных и минимизации суммарной квадратичной ошибки при использовании линейных нейронов в пространстве синаптических связей сети, что позволяет расширить класс обучаемых моделей и повысить обобщающую способность глубоких нейронных сетей.
	\item Алгоритм редуцирования параметров глубокой нейронной сети, базирующийся на неконтролируемом предобучении сети, что позволяет упростить ее архитектуру (путем сокращения числа настраиваемых параметров модели) без потери обобщающей способности.
	\item Нейросетевая система компьютерного зрения, основывающаяся на предлагаемом методе предобучения нейросетевых моделей, применение которой позволяет улучшить качество решения задач классификации. 
\end{enumerate}

%Личный вклад
\vspace{3mm}
\contribution
\vspace{3mm}

Основные положения диссертации получены соискателем лично. Соавтором основных публикаций автора является научный руководитель д.т.н., профессор В.А. Головко, который осуществлял определение целей и постановку задач исследований, выбор методов исследований, принимал участие в планировании работ и обсуждении результатов. В диссертационную работу не включены результаты, которые были получены другими соавторами или с другими соавторами. Материалы совместных публикаций использованы соискателем в объеме авторского вклада.

\vspace{3mm}
\probation
\vspace{3mm}

%Основные положения и результаты диссертационной работы докладывались и обсуждались на следующих международных конференциях: <<International Conference on Neural Networks and Artificial Intelligence>> (Брест, 2014); <<Информационное, программное и техническое обеспечение систем управления организационно-технологическими комплексами>> (Луцк, 2015); <<8th IEEE International Conference on Intelligent Data Acquisition and Advanced Computing Systems: Technology and Applications>> (Варшава, 2015); <<International Scientific-Practical Conference Problems of Infocommunications. Science and Technology (PIC S\&T)>> (2018); <<International Conference on Pattern Recognition and Information Processing (PRIP)>> (2019); <<Вычислительные методы, модели и образовательные технологии>> (Брест, 2013, 2014, 2015, 2016, 2019); <<Современные проблемы математики и вычислительной техники>> (Брест, 2015); <<Открытые семантические технологии проектирования интеллектуальных систем>> (Минск, 2015, 2018, 2019, 2020, 2021, 2023); <<11th IEEE International Conference on Intelligent Data Acquisition and Advanced Computing Systems: Technology and Applications (IDAACS)>> (Краков, 2021).
Основные положения и полученные результаты диссертационной работы докладывались и обсуждались на следующих конференциях: <<8th~International Conference on Neural Networks and Artificial Intelligence>> (Брест, 3-6 июня 2014 г.); <<8th~IEEE International Conference on Intelligent Data Acquisition and Advanced Computing Systems: Technology and Applications>> (Варшава, 24-26~сентября 2015~г.); <<International Scientific-Practical Conference Problems of Infocommunications. Science and Technology (PIC S\&T)>> (Харьков, 9-12~октября 2018~г.); <<14th~International Conference on Pattern Recognition and Information Processing (PRIP)>> (Минск, 21-23~мая 2019~г.); <<Вычислительные методы, модели и образовательные технологии>> (Брест, 22-23~октября 2013~г., 15-16~октября 2014~г., 22~октября 2015~г., 21~октября 2016~г., 18~октября 2019~г.); <<Современные проблемы математики и вычислительной техники>> (Брест, 19-21~ноября 2015~г.); <<Открытые семантические технологии проектирования интеллектуальных систем>> (Минск, 19-21~февраля 2015~г., 15-17~февраля 2018~г., 21-23~февраля 2019~г., 19-22~февраля 2020~г., 16-18~сентября 2021~г., 20-22~апреля 2023~г.); <<11th~IEEE International Conference on Intelligent Data Acquisition and Advanced Computing Systems: Technology and Applications (IDAACS)>> (Краков, 22-25~сентября 2021~г.); <<Международный конгресс по информатике: информационные системы и технологии (CSIST'2022)>> (Минск, 27-28~октября 2022~г.).

По результатам диссертации получено 3 акта о внедрении.
%Научная и практическая значимость
%\influence\ \ldots

%Степень достоверности
%\reliability\ полученных результатов обеспечивается \ldots \ Результаты находятся в соответствии с результатами, полученными другими авторами.

%Публикации
\vspace{3mm}
\publications
\vspace{3mm}

Основные результаты диссертационного исследования опубликованы в~30 научных работах, среди которых: 8 статей в научных изданиях в соответствии с пунктом 19 Положения о присуждении ученых степеней и присвоении ученых званий (общим объемом 3,87 авторского листа), 5 статей в других научных изданиях, 13 статей в сборниках материалов научных конференций и 4 тезисов.
 % Характеристика работы по структуре во введении и в автореферате не отличается (ГОСТ Р 7.0.11, пункты 5.3.1 и 9.2.1), потому её загружаем из одного и того же внешнего файла, предварительно задав форму выделения некоторым параметрам
%% регистрируем счётчики в системе totcounter
% \regtotcounter{totalcount@figure}
% \regtotcounter{totalcount@table}       % Если поставить в преамбуле то ошибка в числе таблиц
% \regtotcounter{TotPages}               % Если поставить в преамбуле то ошибка в числе страниц

%\regtotcounter{page}
%\newtotcounter{mainpage}

% \addtocounter{page}{-1}
%\addtocounter{mainpage}{-1}

\vspace{3mm}
\textbf{Структура и объем диссертации} 
\vspace{3mm}

Диссертация состоит из перечня сокращений и обозначений, введения, общей характеристики работы, четырех глав, заключения, списка использованных источников и двух приложений.

%В \textbf{\textit{первой главе}} рассмотрено краткое введение в теорию искусственных нейронных сетей, дана классификация их типов. Определено понятие глубокой нейронной сети. Рассмотрены задачи обучения глубоких нейронных сетей и существующие методы обучения. \textbf{\textit{Вторая глава}} посвящена рассмотрению разработанного метода обучения ограниченной машины Больцмана и метода предобучения глубоких нейронных сетей на его основе. Также в этой главе предлагается алгоритм редуцирования параметров нейросетевых моделей. В \textbf{\textit{третьей главе}} приведены экспериментальные результаты, обосновывающие полученные теоретические результаты. В \textbf{\textit{четвертой главе}} рассмотрено практическое применение разработанных методов в интеллектуальных системах компьютерного зрения.

Полный объем диссертации составляет 117 страниц, из которых 44 рисунка на 33 страницах, 17 таблиц на 9 страницах, 2 приложения на 13 страницах. Список использованных источников состоит из 110 наименований, включая 30 публикаций автора (на 5 страницах).

% Диссертация состоит из введения, общей характеристики работы, четырех глав с краткими выводами по каждой главе, заключения, библиографического списка, списка публикаций автора и 17 приложений. Общий объём диссертации составляет 254 страницы, из которых 157 страниц основного текста, 51 рисунок на 26 страницах, 5 таблиц на 4 страницах, библиография из 167 источников, включая 37 публикаций автора, приложения на 50 страницах.
%\total{page} 
%\total{mainpage}

%\textcolor{red}{
%Диссертация состоит из введения, общей характеристики работы, четырех глав с краткими выводами по каждой главе, заключения, библиографического списка, списка публикаций автора и 11 приложений. Общий объём диссертации составляет \formbytotal{TotPages}{страниц}{у}{ы}{}, из которых 171 страниц основного текста, \formbytotal{totalcount@figure}{рисунк}{ом}{ами}{ами}, \formbytotal{totalcount@table}{таблиц}{ей}{ами}{ами}, библиография из 174 источников, включая 56 публикаций автора, приложения на 37 страницах.}

%\textcolor{red}{
%Диссертация состоит из введения, общей характеристики работы, четырех глав с краткими выводами по каждой главе, заключения, библиографического списка, списка публикаций автора и 7 приложений. В первой главе проведен анализ моделей, средств разработки баз знаний и методов их создания. Во второй главе предложена семантическая модель базы знаний. Третья глава содержит описание унифицированной семантической модели процесса создания баз знаний, а также унифицированной семантической модели средств автоматизации процесса создания баз знаний, включая унифицированную семантическую модель библиотеки многократно используемых компонентов баз знаний. В четвертой главе описана реализация разработанных моделей и средств, проведена их оценка.
%% на случай ошибок оставляю исходный кусок на месте, закомментированным
% Полный объём диссертации составляет  \ref*{TotPages}~страницу с~\totalfigures{totalcount@figure}~рисунками и~\totaltables{}~таблицами. Список литературы содержит \total{citenum}~наименований.}

%
%\textcolor{red}{
%Полный объём диссертации составляет
%\formbytotal{TotPages}{страниц}{у}{ы}{} 
%с~\formbytotal{totalcount@figure}{рисунк}{ом}{ами}{ами}
%и~\formbytotal{totalcount@table}{таблиц}{ей}{ами}{ами}. Список литературы %содержит  
%\formbytotal{citenum}{наименован}{ие}{ия}{ий}.}