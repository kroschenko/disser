\begin{enumerate}[wide, labelindent=10mm]

\item Выявлена и доказана эквивалентность задач максимизации функции правдоподобия распределения входных данных, минимизации суммарной квадратичной ошибки сети при использовании линейных нейронов и минимизации кросс-энтропийной функции ошибки сети в пространстве синаптических связей ограниченной машины Больцмана. Из полученных теоретических результатов следует, что природа неконтролируемого обучения в RBM-сети является идентичной при использовании различных целевых функций (\cite{2-A}, \cite{3-A}, \cite{4-A}, \cite{5-A}, \cite{10-A}, \cite{12-A}, \cite{13-A});
\item Разработан метод неконтролируемого предобучения глубоких нейронных сетей, базирующийся на минимизации квадратичной ошибки сети в скрытом и видимом слоях RBM-машины, что позволяет учитывать нелинейную природу нейронных элементов. Разработанный метод применен для обучения глубоких полносвязных и сверточных архитектур нейронных сетей и протестирован на выборках MNIST, CIFAR-10, CIFAR-100. Показано, что предложенный метод обладает большей эффективностью, чем классический (\cite{1-A}, \cite{2-A}, \cite{3-A}, \cite{4-A}, \cite{5-A}, \cite{10-A}, \cite{12-A}, \cite{13-A}, \cite{17-A}, \cite{18-A}, \cite{19-A}, \cite{20-A}, \cite{21-A}, \cite{22-A});
\item Предложен алгоритм редуцирования параметров глубокой нейронной сети, базирующийся на неконтролируемом предобучении сети и позволяющий сократить количество настраиваемых параметров сети и упростить ее архитектуру без потери обобщающей способности. Проведены вычислительные эксперименты, доказывающие эффективность предложенного метода (\cite{11-A}, \cite{16-A}, \cite{30-A});
\item С использованием предложенного метода предобучения реализованы прикладные нейросетевые системы компьютерного зрения. 

Разработана нейросетевая система обнаружения солнечных панелей на аэрофотоснимках, позволяющая обнаруживать солнечные панели с точностью 87,46\% с возможностью использования фотографий низкого разрешения (\cite{9-A}, \cite{14-A}, \cite{15-A}).

Разработана нейросетевая система распознавания маркировки продукта на производственной линии, базирующаяся на интеграции различных моделей глубоких сверточных нейронных сетей. Предложенный конвейер представляет собой цепочку взаимодействующих моделей нейросетевых классификаторов и детекторов, которые решают отдельные подзадачи обнаружения и распознавания маркировки или ее части. Подобная архитектура позволяет добавлять новые типы маркировок благодаря простой модульной структуре. Архитектуры нейросетей, используемые для построения конвейера, позволяют осуществлять обработку изображения в реальном времени (\cite{6-A}, \cite{7-A}, \cite{8-A}, \cite{23-A}, \cite{24-A}, \cite{25-A}, \cite{26-A}, \cite{27-A}, \cite{28-A}, \cite{29-A}).

% \item Разработан нейросетевой алгоритм детекции и распознавания лиц на фото и видеоизображениях, базирующийся на использовании предобученных сверточных нейронных сетей и позволяющий пополнять базу знаний интеллектуальной системы. Применение подобного подхода позволило добиться простой расширяемости системы распознавания лиц и высокой точности распознавания лиц, уже включенных в базу [\cite{28-A}, \cite{29-A}];

% \item Введено понятие гибридного решателя задач, обоснована актуальность согласованного использования нескольких моделей решения задач при решении комплексных задач. Сформулирована проблема совместимости различных моделей решения задач, препятствующая созданию гибридных решателей и технологий их разработки. Предложен ряд принципов, лежащих в основе комплекса моделей, методики и средств, который предлагается разрабатывать как часть технологии OSTIS. В качестве основы для построения гибридных решателей задач предлагается использовать вариант реализации многоагентного подхода, при котором агенты взаимодействуют между собой исключительно путем спецификации информационных процессов, выполняемых агентами в семантической памяти. \cite{7-A,3-A,12-A,13-A,14-A,18-A,19-A,26-A,34-A,36-A,37-A}.

% \item Предложена агентно-ориентированная модель гибридного решателя задач, рассматривающая каждый такой решатель как иерархическую систему агентов, управляемых ситуациями и событиями в общей семантической памяти,
% обеспечивающая модифицируемость таких решателей задач, а также возможность решения задач, требующих совместного использования различных методов решения задач. Предложенная модель позволяет рассматривать разрабатываемый решатель задач на различных уровнях детализации, что обеспечивает возможность поэтапного проектирования решателей, а также их модифицируемость. На основе предложенной модели гибридного решателя задач построена агентно-ориентированная модель интерпретатора базового языка программирования, ориентированного на обработку знаний \cite{4-A,5-A,1-A,3-A,14-A,15-A,16-A,18-A,20-A,21-A,26-A,28-A,31-A}.

% \item Предложена модель взаимодействия параллельных асинхронных информационных процессов в общей семантической памяти, определяющая принцип коммуникации агентов, выполняющих указанные процессы, включающая средства синхронизации процессов на основе механизма блокировок элементов семантической памяти, и обеспечивающая возможность спецификации планируемых блокировок, а также выявления и устранения взаимоблокировок. Уточнено понятие агента, выполняющего преобразования в семантической памяти. Разработана классификация агентов и средства их спецификации \cite{4-A,5-A,1-A,3-A,24-A,25-A,26-A,28-A,30-A,33-A}.

% \item Разработана методика построения и модификации гибридных решателей задач, построенных на основе предложенной модели решателя. В основе методики лежит формальная онтология действий разработчиков таких решателей. Наличие такой методики позволяет автоматизировать процесс построения и модификации решателей и снизить требования к их разработчикам. Указанная методика предполагает поэтапное проектирование гибридных решателей с учетом повторного использования разработанных ранее компонентов и возможности независимой отладки и верификации компонентов решателя на нескольких уровнях. \cite{6-A,8-A,2-A,11-A,14-A,15-A,17-A,23-A,26-A,32-A,35-A}.

% \item Разработаны средства автоматизации и информационной поддержки процесса построения и модификации гибридных решателей задач, включающие в себя систему автоматизации процесса построения и модификации решателей и подсистему информационной поддержки разработчиков решателей в рамках метасистемы IMS. Решатели задач каждой из подсистем построены на основе предложенной модели решателя, что обеспечивает их модифицируемость. Разработана библиотека многократно используемых компонентов решателей задач, включающая набор компонентов, языковые средства их спецификации и средства автоматизации поиска компонентов на основе заданной спецификации \cite{4-A,6-A,8-A,15-A,17-A,22-A,26-A}.

% \item С использованием языка C реализована разработанная агентно-ориентированная модель интерпретатора базового языка программирования, ориентированного на обработку знаний. Все описанные в рамках диссертационной работы модели формализованы и включены в состав соответствующих разделов базы знаний разрабатываемой метасистемы IMS, таким образом обеспечена информационная поддержка разработчиков решателей.
% С использованием предложенных моделей и средств разработаны решатели задач для ряда прикладных интеллектуальных систем. В процессе построения указанных решателей произведено стартовое наполнение библиотеки многократно используемых sc-агентов, показана эффективность предложенной методики с учетом текущей версии библиотеки. 
% Сведения об использовании результатов диссертационной работы отражены в соответствующих актах внедрения, приведенных в приложении к диссертации \cite{8-A,9-A,10-A,13-A,20-A,27-A,29-A,32-A,36-A}.

\end{enumerate}