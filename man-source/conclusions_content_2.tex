
%Результаты исследований использованы для практической реализации интеллектуальной системы видеонаблюдения реального времени. 
Научные и практические результаты диссертационной работы использованы в научно-исследовательских работах, учебном процессе, а также в ряде прикладных систем.

Разработанные алгоритмы обнаружения и локализации солнечных панелей на аэрофотоснимках, методы предобучения нейросетевых моделей и инструментальные средства внедрены и используются в компании ООО~<<Intelligent Semantic Systems>> при разработке интеллектуальных систем в составе подсистем компьютерного зрения.

Предложенный метод предобучения глубоких нейронных сетей применялся при обучении модели классификатора кадров для нейросетевой системы распознавания маркировки продукта на производственной линии ОАО~<<Савушкин продукт>>.

Кроме того, научные и практические результаты диссертационной работы используются в учебном процессе учреждения образования <<Брестский государственный технический университет>>.

% Разработанные модели, средства и их программная реализация могут быть использованы при разработке решателей задач интеллектуальных систем различного назначения, как инструментальных, так и прикладных. 

Реализация программных модулей осуществлялась с использованием свободного ПО и может быть непосредственно применена при разработке отечественных проектов в области компьютерного зрения без необходимости приобретения дорогостоящих программных средств.

% Научные и практические результаты диссертационной работы используются в учебном процессе учреждения образования <<Белорусский государственный университет информатики и радиоэлектроники>>, в НИЛ 3.7 при выполнении работ по договору БРФФИ--РФФИ (№ ГР 20164340), а также на предприятиях ОАО <<Савушкин продукт>> при разработке решателя задач системы автоматизации рецептурного производства, ООО <<ФордэКонсалтинг>> при разработке программной системы обслуживания клиентов розничной торговли, ООО <<Кинросс-ресерч>> при разработке системы автоматизации и оптимизации строительного проектирования, ФГБУН Институт систем энергетики им. Л. А. Мелентьева СО РАН при разработке решателя в области энергетики.