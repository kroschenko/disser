\documentclass{thesisby}

%\usepackage[cp1251]{inputenc}
\usepackage[utf8]{inputenc}
\usepackage[T2A]{fontenc}
\usepackage[russian]{babel}

%\usepackage{pscyr}
%\renewcommand{\rmdefault}{ftm}

%для борьбы с переполнениями за счет разреженных слов в абзаце
\emergencystretch=25pt
%math
\usepackage{mathtools}
\usepackage{tabularx}
\usepackage{pdflscape}

\usepackage{afterpage}
\usepackage{geometry}
\usepackage{multirow}
\usepackage{enumitem}

\usepackage{amsmath,amssymb,amsfonts}
\usepackage{longtable,array}
\usepackage{graphicx,epsfig}
\usepackage[unicode,colorlinks=false,pagebackref=false]{hyperref}
%\usepackage{refcheck}% checks lost and useless labels, shows `keys' of \label in the margins
%фиксить картинки
%\usepackage{float}

\usepackage{textcomp}% For Celsium sign only

% Листинги с исходным кодом программ
\usepackage{fancyvrb}
\usepackage{listings}
\usepackage{totcount}
%\usepackage{totpages}

% Плавающие окружения. во многом лучше пакета float
\usepackage{floatrow}

\usepackage{tikz}
\newcolumntype{P}[1]{>{\centering\arraybackslash}p{#1}}
\newcolumntype{M}[1]{>{\centering\arraybackslash}m{#1}}

%for lists
\usepackage[ampersand]{easylist}
\ListProperties(Hide=100, Hang=false, Margin=0mm, Indent1=10.5mm, Indent2=15mm, Style*=-- ,
Style2*=$\bullet$ ,Style3*=$\circ$ ,Style4*=\tiny$\blacksquare$ )

\usepackage[ruled, vlined]{algorithm2e}
\makeatletter
\newenvironment{algo}[1][]
  {\renewcommand{\algorithmcfname}{Алгоритм}%
   \begin{algorithm}[#1]
   \long\def\@caption##1[##2]##3{%
     \par
     \begingroup\@parboxrestore
     \if@minipage\@setminipage\fi
     \normalsize \@makecaption{\AlCapSty{\AlCapFnt\algorithmcfname}}{\ignorespaces ##3}%
     \par\endgroup
   }}
  {\end{algorithm}}
\makeatother

% \SetKwData{KwDat}{Данные}
\SetKwInput{KwIn}{Исходные данные}
\SetKwInput{KwRes}{Результат}%
% \SetKwIF{Si}{SinonSi}{Sinon}{si}{alors}{sinon si}{sinon}{fin si}%
% \SetKwFor{Tq}{tant que}{faire}{fin tq}%

\newenvironment{easylistNum}{\begin{easylist}
\ListProperties(Hide1=0, Hang=false, Margin=0mm, Indent1=10.5mm, Indent2=15mm, Start1=1, Style*=, FinalMark={)})}{\ListProperties(Hide=100, Hang=false, Margin=0mm, Indent1=10.5mm, Indent2=15mm, Style*=-- ,
Style2*=$\bullet$ ,Style3*=$\circ$ ,Style4*=\tiny$\blacksquare$ )\end{easylist}}

\renewcommand\labelitemi{\textbf{--}}
\floatsetup[table]{capposition=top}

    % \lstset{
    %     language=Python,
    %     basicstyle=\ttfamily,
    %     keywordstyle=\bfseries,
    %     showspaces=false,
    %     showstringspaces=false,
    %     mathescape=true,
    %     aboveskip=0pt,
    %     belowskip=6pt
    % }
\lstdefinestyle{PythonStyle}{
  keywordstyle=\bf,
  belowcaptionskip=1\baselineskip,
  breaklines=true,
  language=Python,
  showstringspaces=false,
  basicstyle=\small\ttfamily,
  commentstyle=\itshape
}
% \lstdefinestyle{PythonStyle}{
% basicstyle=\footnotesize\ttfamily,
% language=Python,
% keywordstyle=\bfseries,
% showstringspaces=false,
% commentstyle={},
% texcl=true
% }

\graphicspath{{../}}

%\renewcommand{\cftchapleader}{\cftdotfill{\cftdotsep}}


\def\isabstract{1}

\begin{document}

\thispagestyle{empty}

\begin{center} 

\bfseries
{Учреждение образования}
\medskip

{БЕЛОРУССКИЙ ГОСУДАРСТВЕННЫЙ УНИВЕРСИТЕТ ИНФОРМАТИКИ И РАДИОЭЛЕКТРОНИКИ}

%\vspace*{\fill}

\vspace{2cm}
\end{center}

\noindent УДК 004.82:004.89 \\[5mm]

\begin{center}

{ШУНКЕВИЧ \\ Даниил Вячеславович}\\

%\vspace*{\fill}
\vspace{1cm}

{\bfseries АГЕНТНО-ОРИЕНТИРОВАННЫЕ РЕШАТЕЛИ ЗАДАЧ  ИНТЕЛЛЕКТУАЛЬНЫХ СИСТЕМ}\\ \vspace{2cm}

АВТОРЕФЕРАТ\\ диссертации на соискание ученой степени\\
кандидата технических наук\\
\medskip


%\vspace*{\fill}

    
\vspace{1cm}
по специальности 05.13.17 -- Теоретические основы информатики 

\vspace*{\fill}


Минск 2018
\end{center}


\newpage

\thispagestyle{empty} 

\noindent Работа выполнена в учреждении образования <<Белорусский государственный университет информатики и радиоэлектроники>>.

\vspace*{\fill} 

 

\noindent \hangindent=70mm \hangafter=1
     Научный руководитель \hspace{10mm} {\bf Голенков Владимир Васильевич}, доктор технических наук, профессор, заведующий кафедрой  интеллектуальных информационных технологий учреждения образования <<Белорусский государственный университет информатики и радиоэлектроники>>\\
            

\vspace*{\fill} \noindent \hangindent=70mm \hangafter=1 
    Официальные оппоненты: \hspace{10mm}{\bf Татур Михаил Михайлович}, доктор технических наук, профессор, профессор кафедры электронных вычислительных машин учреждения образования <<Белорусский государственный университет информатики и радиоэлектроники>> \\ [5pt]
    {\bf Ерофеев Александр Александрович}, кандидат технических наук, доцент, проректор по научной работе Белорусского государственного университета транспорта



\vspace*{\fill} \noindent \hangindent=66mm \hangafter=1 Оппонирующая
организация \hspace{4mm} Белорусский государственный университет


\vspace*{\fill} \noindent Защита состоится <<04>> октября~2018~г. в 14.00
на заседании совета по защите диссертаций Д 02.15.04 при учреждении образования <<Белорусский государственный университет информатики и радиоэлектроники>> по адресу: 220013, Минск, ул. П. Бровки, 6, корп. 1, ауд. 232, тел. 293-89-89, e-mail: dissovet@bsuir.by.


\vspace*{\fill} \noindent
С диссертацией можно ознакомиться в библиотеке учреждения образования <<Белорусский государственный университет информатики и радиоэлектроники>>

\vspace*{\fill} \noindent
Автореферат разослан~
<<\rule{10mm}{0.4pt}>>~августа~2018~г.
%~\rule{25mm}{0.4pt}~1999~г.

\vspace*{\fill}

\noindent Ученый секретарь совета\\ по защите диссертаций\\ доктор технических наук \hfill А. В. Сидоренко

\vspace*{\fill}

\vspace*{\fill}

\newpage
\setcounter{page}{1}

\centerline{\bf КРАТКОЕ ВВЕДЕНИЕ}

\parindent=1cm

\medskip
% В настоящее время все более актуальным становится использование интеллектуальных систем в самых различных областях. Одним из ключевых компонентов интеллектуальной системы, обеспечивающим возможность решать широкий круг задач, является решатель задач. Особенностью решателей задач интеллектуальных систем по сравнению с другими современными программными системами является необходимость решать задачи в условиях, когда сведения, необходимые для решения задачи, не локализованы явно в базе знаний интеллектуальной системы и должны быть найдены в процессе решения задачи на основании каких-либо критериев.

% \ifx\isabstract\undefined 
% Состав решателя задач каждой конкретной системы зависит от ее назначения, классов решаемых задач, предметной области и ряда других факторов.
% \fi

% \ifx\isabstract\undefined 
% В общем случае решатель задач обеспечивает возможность решения задач, связанных как с непосредственно основной функциональностью системы, так и с обеспечением эффективности работы такой системы, а также с обеспечением автоматизации развития самой этой системы. Решатель задач, обеспечивающий выполнение всех перечисленных функций, будем называть \textit{объединенным решателем задач} указанной интеллектуальной системы.
% \fi

% %Ключевой отличительной особенностью решателей задач от современных программных систем вообще является необходимость решать задачи в условиях, когда исходные данные для задачи не локализованы явно
% %Отличительными особенностями решателей задач является их ориентация на обработку знаний, хранящихся в базе знаний интеллектуальной системы, а также 

% Расширение областей применения интеллектуальных систем требует от таких систем возможности решения комплексных задач, решение каждой из которых предполагает совместное использование целого ряда различных моделей представления знаний и различных моделей решения задач. Кроме того, решение комплексных задач предполагает использование общих информационных ресурсов (в предельном случае -- всей базы знаний интеллектуальной системы) различными компонентами решателя, ориентированными на решение различных подзадач. Поскольку решатель комплексных задач осуществляет интеграцию различных моделей решения задач, будем называть его \textit{гибридным решателем задач}.

% Примерами комплексных задач являются:

% \begin{easylist}
% &	задачи понимания текстов естественного языка (как печатных, так и рукописных), понимания речевых сообщений, изображений\ifx\isabstract\undefined. В каждом из перечисленных случаев необходимо осуществить синтаксический анализ обрабатываемого файла (сигнала), устранить незначимые фрагменты, классифицировать значимые фрагменты, соотнести их с понятиями, известными системе, выявить те фрагменты, которые система распознать не в состоянии, устранить дублирование информации и т. д.\fi;
% &	задачи автоматизации адаптивного обучения школьников и студентов\ifx\isabstract\undefined, предполагающие, что система может самостоятельно решать различные задачи из некоторой предметной области, а также управлять процессом обучения, самостоятельно формировать задания для учащихся и контролировать их выполнение\fi;
% &	задачи планирования поведения интеллектуальных роботов\ifx\isabstract\undefined, предполагающие как понимание различного рода внешней информации, так и принятие различных решений с использованием как достоверных методов, так и правдоподобных\fi;
% &	задачи комплексной и гибкой автоматизации различных предприятий;
% &	и другие.
% \end{easylist}

% Использование различных моделей решения задач в рамках
% интеллектуальной системы предполагает декомпозицию комплексной задачи на подзадачи, которые могут быть решены с помощью одной из известных интеллектуальной системе моделей решения задач. Благодаря комбинации различных моделей решения задач, множество задач, решаемых гибридным решателем, будет значительно шире, чем объединение множеств задач, решаемых по отдельности всеми решателями задач, входящими в его состав\ifx\isabstract\undefined ~\cite{Tarasov2007}\fi.

% \ifx\isabstract\undefined
% Значительный вклад в разработку и исследование моделей, методов и средств решения задач в интеллектуальных системах внесли такие ученые, как E. W. Dijkstra, C. A. Hoare, P. Jackson, J. McCarthy, A. Newell, P. Norvig, G. Polya, R. Reiter, S. Russel, H. A. Simon, D. Waterman, M. Wooldridge, А. Н. Аверкин, И. З. Батыршин, А. Н. Борисов, В. Б. Борщев, В. Н. Вагин, Т. А. Гаврилова, Л. А. Гладков, В. А. Головко, А. Н. Горбань, В. И. Городецкий, В. В. Грибова, А. П. Еремеев, Ю. А. Загорулько, А. С. Клещев, А. В. Колесников, В. Е. Котов, О. П. Кузнецов, В. М. Курейчик, Д. В. Ландэ, Л. В. Массель, А. С. Нариньяни, Г. С. Осипов, Г. С. Плесневич, Э. В. Попов, Д. А. Поспелов, Г. В. Рыбина, П. О. Скобелев, В. Б. Тарасов, Э. Х. Тыугу, В. К. Финн, И. Б. Фоминых, В. Ф. Хорошевский, А. Е. Янковская и др.
% \fi

% Постоянная эволюция интеллектуальных систем и технологий их разработки делает актуальной не только проблему снижения сроков разработки гибридных решателей задач, но и проблему снижения трудоемкости внесения изменений в состав уже разработанных решателей без необходимости изменения архитектуры всей системы в целом. 

% Современные подходы к построению гибридных решателей задач, как правило, предполагают совмещение разнородных моделей решения задач без какой-либо единой основы, например, посредством специализированных программных интерфейсов между разными компонентами системы, что приводит к существенным накладным расходам при разработке такой системы и в особенности при ее модификации, в том числе при добавлении в систему новой модели решения задач. 

% Таким образом, несмотря на успехи в области разработки решателей задач интеллектуальных систем, остаются нерешенными проблемы, связанные с обеспечением:
% \begin{easylist}
% & совместимости различных частных решателей задач, т. е. возможности их согласованного использования при решении одной и той же комплексной задачи;
% & возможности без существенных накладных расходов модифицировать гибридный решатель непосредственно в процессе эксплуатации интеллектуальной системы, в том числе расширять число используемых моделей решения задач без каких-либо ограничений на вид этих моделей. Такое требование обусловлено тем, что при решении комплексной задачи априори может оказаться неизвестным, какие именно модели решения задач и виды знаний могут потребоваться.
% \end{easylist}

% Важным способом снижения трудоемкости процесса изменения функциональности интеллектуальных систем является накопление библиотек совместимых компонентов решателей, которые позволят значительно снизить как сроки разработки и модификации решателей, так и уровень профессиональных требований к их разработчикам.

% Кроме того, актуальной является проблема создания самих средств разработки гибридных решателей задач, обеспечивающих информационную поддержку и автоматизацию деятельности разработчиков.

Практические приложения компьютерного зрения с каждым годом становятся все более разнообразными. В современном мире компьютерное зрение используется повсеместно -- в сложных производственных и медицинских системах, в интеллектуальных системах интернета вещей и развлекательных приложениях для мобильных устройств.

В качестве основы при разработке таких систем все чаще находят применение глубокие нейросетевые модели. Данные модели показывают впечатляющие результаты при решении самых разнообразных задач компьютерного зрения -- распознавания, детекции и сегментации объектов на фото- и видеоизображениях, получения аннотаций для фотографий и генерации изображений по текстовому описанию. Глубокие нейронные сети, применяемые для решения подобных задач, содержат миллионы настраиваемых параметров и, для некоторых архитектур, десятки слоев нейронных элементов.

Обучение подобных <<тяжелых>> моделей с нуля является нетривиальной задачей. Оно часто сопряжено с риском переобучения, результатом которого является отличная приспособленность сети к данным из обучающей выборки, но плохая обобщающая способность, то есть неэффективность модели для данных, не использовавшихся при обучении. Чаще всего переобученность возникает при применении малой обучающей выборки. Другой проблемой является ресурсоемкость процесса обучения таких моделей, даже при применении современных технических средств.

Проблемы обучения глубоких нейронных сетей активно изучаются в зарубежных научных школах. В нашей стране такие исследования также проводятся. Однако, нужно отметить, что такие исследования часто носят эмпирический характер, поэтому разработка строгих математически обоснованных методов обучения остается важной задачей теории нейронных сетей.

В диссертационной работе разработаны подходы для неконтролируемого предобучения глубоких нейронных сетей и редуцирования параметров моделей. Предложены алгоритмы для решения практических задач теории компьютерного зрения -- обнаружения и локализации солнечных панелей на аэрофотоснимках, обнаружения и распознавания маркировки на поточных производственных линиях. Предложенные методы и алгоритмы позволят улучшить работу интеллектуальных систем, использующих полносвязные и сверточные нейросетевые модели.


\bigskip
\centerline{\bf ОБЩАЯ ХАРАКТЕРИСТИКА РАБОТЫ}
\medskip
\newcommand{\actuality}{\textbf{Связь работы с научными программами (проектами), темами}}
\newcommand{\aim}{\textbf{Цель, задачи, объект и предмет исследования}}
%\newcommand{\tasks}{\textit{задачи исследования}}
\newcommand{\novelty}{\textbf{Научная новизна}}
\newcommand{\defpositions}{\textbf{Положения, выносимые на защиту}}
\newcommand{\influence}{\textbf{Научная и практическая значимость}}
\newcommand{\reliability}{\textbf{Степень достоверности}}
\newcommand{\contribution}{\textbf{Личный вклад соискателя ученой степени в результаты диссертации}}
\newcommand{\probation}{\textbf{Апробация диссертации и информация об использовании ее результатов}}
\newcommand{\publications}{\textbf{Опубликованность результатов диссертации}}

{\actuality}
\vspace{3mm}

Тема диссертации соответствует приоритетному направлению научно-технической деятельности согласно пункту 1 перечня приоритетных направлений научной, научно-технической и инновационной деятельности на 2021-2025 годы  (Указ Президента Республики Беларусь от 07 мая 2020 г. № 156).

Исследования по теме диссертационной работы проводились в рамках научных программ:
\begin{enumerate}[wide, labelindent=10mm]
\item НИР МОРБ <<Алгоритмы интеллектуального анализа и обработки больших объемов данных на основе нейронных сетей глубокого доверия>> (ГБ 15/203, № госрегистрации 20150743),
\item ГПНИ <<Информатика и космос, научное обеспечение безопасности и защиты от чрезвычайных ситуаций>> по заданию <<Нейросетевые методы обработки комплексной информации и принятия решений на основе интеллектуальных многоагентных систем>> (№ госрегистрации 20140547),
\item ГПНИ <<Информатика и космос, научное обеспечение безопасности и защиты от чрезвычайных ситуаций>> по заданию <<Методы и алгоритмы интеллектуальной обработки и анализа большого объема данных на основе нейронных сетей глубокого доверия>> (задание 1.6.05, № госрегистрации 20163595),
\item НИР <<Методы и алгоритмы построения интеллектуальных систем анализа и обработки данных>>, этап <<Разработка гибридных интеллектуальных систем на основе нейросимволического подхода>> (решение НТС УО <<Брестский государственный технический университет>> от 12.11.2021, протокол № 6, № 22202052022070),
\item НИР БРФФИ <<Модели и исследование 3-D оцифровки на основе фактических данных и анализа гетерогенных данных>> (№ Ф22КИ-046 от 05.11.2021 г., № госрегистрации 20220090).
\end{enumerate}

\vspace{3mm}
\aim
\vspace{3mm}

\textit{Целью исследования} является разработка эффективных методов и алгоритмов для обучения глубоких нейронных сетей, используемых для решения задач компьютерного зрения, включающих обнаружение солнечных панелей на аэрофотоснимках и распознавание маркировки продукта на конвейерной линии.

Указанная цель определяет следующие \textit{задачи исследования}:
\begin{easylistNum}
	& Разработать метод неконтролируемого предобучения глубоких нейронных сетей, позволяющий повысить эффективность обучения моделей;
	& Разработать алгоритм редуцирования параметров глубоких нейронных сетей, позволяющий упростить структуру моделей;
	& Провести сравнительный анализ эффективности разработанных метода обучения и алгоритма редуцирования;
	& Разработать нейросетевую систему компьютерного зрения для решения прикладных задач (обнаружения солнечных панелей на аэрофотоснимках и распознавания маркировки продукта на конвейерной линии) с применением предлагаемого метода обучения.
\end{easylistNum}

\textit{Объектом исследования} являются нейросетевые системы компьютерного зрения. \textit{Предметом исследования} выступают методы и алгоритмы обучения глубоких нейронных сетей и их применение к задачам компьютерного зрения.

\vspace{3mm}
\novelty
\vspace{3mm}

Научная новизна состоит в установлении эквивалентности задач максимизации функции правдоподобия распределения входных данных, минимизации кросс-энтропийной функции ошибки и суммарной квадратичной ошибки при использовании линейных нейронов в пространстве синаптических связей ограниченной машины Больцмана, что позволяет учитывать нелинейную природу нейронных элементов.

Разработан метод обучения ограниченной машины Больцмана на основе доказанных эквивалентностей, применение которого для предобучения ГНС позволяет расширить класс обучаемых моделей и повысить обобщающую способность ГНС.

Разработан алгоритм редуцирования параметров нейросетевой модели, основывающийся на неконтролируемом предобучении, что позволяет уменьшить количество настраиваемых параметров модели без потери обобщающей способности.

Разработана нейросетевая система компьютерного зрения, которая основана на предлагаемом методе предобучения нейросетевых моделей, применение которой позволяет улучшить качество решения задач классификации. Продемонстрирована эффективность системы на примере решения задач обнаружения солнечных панелей на аэрофотоснимках и распознавания маркировки продуктов на производственной линии.

\vspace{10mm}
\defpositions
\vspace{3mm}

\begin{enumerate}[wide, labelindent=10mm]
	\item Установление эквивалентности задач максимизации функции правдоподобия распределения входных данных, минимизации кросс-энтропийной функции ошибки и суммарной квадратичной ошибки при использовании линейных нейронов в пространстве синаптических связей ограниченной машины Больцмана, что позволяет учитывать нелинейную природу нейронных элементов.
	\item Метод обучения ограниченной машины Больцмана, базирующийся на эквивалентности задач максимизации функции правдоподобия распределения входных данных и минимизации суммарной квадратичной ошибки при использовании линейных нейронов в пространстве синаптических связей сети, что позволяет расширить класс обучаемых моделей и повысить обобщающую способность глубоких нейронных сетей.
	\item Алгоритм редуцирования параметров глубокой нейронной сети, базирующийся на неконтролируемом предобучении сети, что позволяет упростить ее архитектуру (путем сокращения числа настраиваемых параметров модели) без потери обобщающей способности.
	\item Нейросетевая система компьютерного зрения, основывающаяся на предлагаемом методе предобучения нейросетевых моделей, применение которой позволяет улучшить качество решения задач классификации. 
\end{enumerate}

%Личный вклад
\vspace{3mm}
\contribution
\vspace{3mm}

Основные положения диссертации получены соискателем лично. Соавтором основных публикаций автора является научный руководитель д.т.н., профессор В.А. Головко, который осуществлял определение целей и постановку задач исследований, выбор методов исследований, принимал участие в планировании работ и обсуждении результатов. В диссертационную работу не включены результаты, которые были получены другими соавторами или с другими соавторами. Материалы совместных публикаций использованы соискателем в объеме авторского вклада.

\vspace{3mm}
\probation
\vspace{3mm}

%Основные положения и результаты диссертационной работы докладывались и обсуждались на следующих международных конференциях: <<International Conference on Neural Networks and Artificial Intelligence>> (Брест, 2014); <<Информационное, программное и техническое обеспечение систем управления организационно-технологическими комплексами>> (Луцк, 2015); <<8th IEEE International Conference on Intelligent Data Acquisition and Advanced Computing Systems: Technology and Applications>> (Варшава, 2015); <<International Scientific-Practical Conference Problems of Infocommunications. Science and Technology (PIC S\&T)>> (2018); <<International Conference on Pattern Recognition and Information Processing (PRIP)>> (2019); <<Вычислительные методы, модели и образовательные технологии>> (Брест, 2013, 2014, 2015, 2016, 2019); <<Современные проблемы математики и вычислительной техники>> (Брест, 2015); <<Открытые семантические технологии проектирования интеллектуальных систем>> (Минск, 2015, 2018, 2019, 2020, 2021, 2023); <<11th IEEE International Conference on Intelligent Data Acquisition and Advanced Computing Systems: Technology and Applications (IDAACS)>> (Краков, 2021).
Основные положения и полученные результаты диссертационной работы докладывались и обсуждались на следующих конференциях: <<8th~International Conference on Neural Networks and Artificial Intelligence>> (Брест, 3-6 июня 2014 г.); <<8th~IEEE International Conference on Intelligent Data Acquisition and Advanced Computing Systems: Technology and Applications>> (Варшава, 24-26~сентября 2015~г.); <<International Scientific-Practical Conference Problems of Infocommunications. Science and Technology (PIC S\&T)>> (Харьков, 9-12~октября 2018~г.); <<14th~International Conference on Pattern Recognition and Information Processing (PRIP)>> (Минск, 21-23~мая 2019~г.); <<Вычислительные методы, модели и образовательные технологии>> (Брест, 22-23~октября 2013~г., 15-16~октября 2014~г., 22~октября 2015~г., 21~октября 2016~г., 18~октября 2019~г.); <<Современные проблемы математики и вычислительной техники>> (Брест, 19-21~ноября 2015~г.); <<Открытые семантические технологии проектирования интеллектуальных систем>> (Минск, 19-21~февраля 2015~г., 15-17~февраля 2018~г., 21-23~февраля 2019~г., 19-22~февраля 2020~г., 16-18~сентября 2021~г., 20-22~апреля 2023~г.); <<11th~IEEE International Conference on Intelligent Data Acquisition and Advanced Computing Systems: Technology and Applications (IDAACS)>> (Краков, 22-25~сентября 2021~г.); <<Международный конгресс по информатике: информационные системы и технологии (CSIST'2022)>> (Минск, 27-28~октября 2022~г.).

По результатам диссертации получено 3 акта о внедрении.
%Научная и практическая значимость
%\influence\ \ldots

%Степень достоверности
%\reliability\ полученных результатов обеспечивается \ldots \ Результаты находятся в соответствии с результатами, полученными другими авторами.

%Публикации
\vspace{3mm}
\publications
\vspace{3mm}

Основные результаты диссертационного исследования опубликованы в~30 научных работах, среди которых: 8 статей в научных изданиях в соответствии с пунктом 19 Положения о присуждении ученых степеней и присвоении ученых званий (общим объемом 3,87 авторского листа), 5 статей в других научных изданиях, 13 статей в сборниках материалов научных конференций и 4 тезисов.
 % Характеристика работы по структуре во введении и в автореферате не отличается (ГОСТ Р 7.0.11, пункты 5.3.1 и 9.2.1), потому её загружаем из одного и того же внешнего файла, предварительно задав форму выделения некоторым параметрам
%% регистрируем счётчики в системе totcounter
% \regtotcounter{totalcount@figure}
% \regtotcounter{totalcount@table}       % Если поставить в преамбуле то ошибка в числе таблиц
% \regtotcounter{TotPages}               % Если поставить в преамбуле то ошибка в числе страниц

%\regtotcounter{page}
%\newtotcounter{mainpage}

% \addtocounter{page}{-1}
%\addtocounter{mainpage}{-1}

\vspace{3mm}
\textbf{Структура и объем диссертации} 
\vspace{3mm}

Диссертация состоит из перечня сокращений и обозначений, введения, общей характеристики работы, четырех глав, заключения, списка использованных источников и двух приложений.

%В \textbf{\textit{первой главе}} рассмотрено краткое введение в теорию искусственных нейронных сетей, дана классификация их типов. Определено понятие глубокой нейронной сети. Рассмотрены задачи обучения глубоких нейронных сетей и существующие методы обучения. \textbf{\textit{Вторая глава}} посвящена рассмотрению разработанного метода обучения ограниченной машины Больцмана и метода предобучения глубоких нейронных сетей на его основе. Также в этой главе предлагается алгоритм редуцирования параметров нейросетевых моделей. В \textbf{\textit{третьей главе}} приведены экспериментальные результаты, обосновывающие полученные теоретические результаты. В \textbf{\textit{четвертой главе}} рассмотрено практическое применение разработанных методов в интеллектуальных системах компьютерного зрения.

Полный объем диссертации составляет 117 страниц, из которых 44 рисунка на 33 страницах, 17 таблиц на 9 страницах, 2 приложения на 13 страницах. Список использованных источников состоит из 110 наименований, включая 30 публикаций автора (на 5 страницах).

% Диссертация состоит из введения, общей характеристики работы, четырех глав с краткими выводами по каждой главе, заключения, библиографического списка, списка публикаций автора и 17 приложений. Общий объём диссертации составляет 254 страницы, из которых 157 страниц основного текста, 51 рисунок на 26 страницах, 5 таблиц на 4 страницах, библиография из 167 источников, включая 37 публикаций автора, приложения на 50 страницах.
%\total{page} 
%\total{mainpage}

%\textcolor{red}{
%Диссертация состоит из введения, общей характеристики работы, четырех глав с краткими выводами по каждой главе, заключения, библиографического списка, списка публикаций автора и 11 приложений. Общий объём диссертации составляет \formbytotal{TotPages}{страниц}{у}{ы}{}, из которых 171 страниц основного текста, \formbytotal{totalcount@figure}{рисунк}{ом}{ами}{ами}, \formbytotal{totalcount@table}{таблиц}{ей}{ами}{ами}, библиография из 174 источников, включая 56 публикаций автора, приложения на 37 страницах.}

%\textcolor{red}{
%Диссертация состоит из введения, общей характеристики работы, четырех глав с краткими выводами по каждой главе, заключения, библиографического списка, списка публикаций автора и 7 приложений. В первой главе проведен анализ моделей, средств разработки баз знаний и методов их создания. Во второй главе предложена семантическая модель базы знаний. Третья глава содержит описание унифицированной семантической модели процесса создания баз знаний, а также унифицированной семантической модели средств автоматизации процесса создания баз знаний, включая унифицированную семантическую модель библиотеки многократно используемых компонентов баз знаний. В четвертой главе описана реализация разработанных моделей и средств, проведена их оценка.
%% на случай ошибок оставляю исходный кусок на месте, закомментированным
% Полный объём диссертации составляет  \ref*{TotPages}~страницу с~\totalfigures{totalcount@figure}~рисунками и~\totaltables{}~таблицами. Список литературы содержит \total{citenum}~наименований.}

%
%\textcolor{red}{
%Полный объём диссертации составляет
%\formbytotal{TotPages}{страниц}{у}{ы}{} 
%с~\formbytotal{totalcount@figure}{рисунк}{ом}{ами}{ами}
%и~\formbytotal{totalcount@table}{таблиц}{ей}{ами}{ами}. Список литературы %содержит  
%\formbytotal{citenum}{наименован}{ие}{ия}{ий}.}

\bigskip
\centerline{\bf ОСНОВНАЯ ЧАСТЬ}\medskip
Во \textbf{введении} обоснована актуальность темы диссертационной работы, дана краткая характеристика исследуемых вопросов, обозначены актуальные
задачи, решению которых посвящена диссертационная работа, дано определение понятию гибридного решателя задач.

В {\bf первой главе} обоснована актуальность согласованного использования нескольких моделей при решении комплексных задач в интеллектуальных системах. На основе анализа примеров такого рода задач сформулированы требования к гибридным решателям задач и технологиям их создания. Рассмотрены существующие подходы к построению решателей, выявлены их недостатки. Сформулирована проблема совместимости различных моделей решения задач на общей семантической основе, препятствующая созданию гибридных решателей и технологий, удовлетворяющих указанным требованиям.

В качестве основы для построения гибридных решателей задач предлагается использовать вариант реализации многоагентного подхода, при котором агенты взаимодействуют между собой путем спецификации информационных процессов, выполняемых агентами в семантической памяти.

Основные принципы предлагаемого подхода:
\begin{easylist}
& коммуникация агентов осуществляется по принципу <<доски объявлений>>, однако в отличие от классического подхода в роли сообщений выступают спецификации в общей семантической памяти выполняемых агентами процессов, направленных на решение каких-либо задач, а в роли среды коммуникации выступает сама семантическая память. Такой подход позволяет:

&& исключить необходимость разработки специализированного языка для обмена сообщениями;
&& обеспечить <<обезличенность>> общения, т. е. каждый из агентов в общем случае не знает, какие еще агенты есть в системе, кем сформулирован и кому адресован тот или иной запрос;
&& агентам, в том числе конечному пользователю, формулировать задачи в \textit{декларативном ключе}, т. е. не указывать для каждой задачи способ ее решения;
&& сделать средства коммуникации агентов и синхронизации их деятельности  более понятными разработчику и пользователю системы;

& в роли внешней среды для агентов выступает та же семантическая память, в которой формулируются задачи и посредством которой осуществляется взаимодействие агентов;
& спецификация каждого агента описывается средствами языка представления знаний в той же семантической памяти;
& синхронизация деятельности агентов осуществляется на уровне выполняемых ими процессов, направленных на решение задач в семантической памяти;
& каждый информационный процесс в любой момент времени имеет ассоциативный доступ к необходимым фрагментам базы знаний (с учетом механизма синхронизации);
& каждый из агентов обладает набором ключевых элементов (как правило, понятий), которые он использует в качестве отправных точек при ассоциативном поиске в рамках базы знаний.
\end{easylist}

Модель семантической памяти, используемая в данной работе, реализуется в виде хранилища семантических сетей (sc-хранилища), которое является частью платформы интерпретации семантических моделей (sc-моделей) интеллектуальных систем и обеспечивает возможность чтения и редактирования семантической сети, хранящейся в памяти. Платформа интерпретации кроме sc-хранилища включает также интерпретатор программ базового языка программирования, ориентированного на обработку унифицированных семантических сетей (языка SCP), программы которого также хранятся в семантической памяти. Таким образом, программа каждого агента, входящего в состав решателя, может быть реализована как платформенно-независимым образом (на языке SCP), так и платформенно-зависимым образом. Платформа интерпретации в общем случае может быть реализована различными способами, в том числе аппаратно.

Для реализации указанного подхода предлагается разработать агентно-ориентированную модель гибридного решателя задач интеллектуальных систем, а также формальную модель взаимодействия параллельных асинхронных информационных процессов, выполняемых агентами в общей семантической памяти.

Для снижения трудоемкости построения и модификации гибридных решателей задач и требований к их разработчикам предлагается разработать методику построения и модификации таких решателей, основанную на формальной онтологии деятельности разработчиков, и средства автоматизации и информационной поддержки процесса построения и модификации таких решателей, включающие библиотеку многократно используемых компонентов решателей. 

Предлагаемые в диссертационной работе модели, методика и средства предполагается разрабатывать как часть открытой семантической технологии проектирования интеллектуальных систем OSTIS.

%Графически структуру данной диссертационной работы можно изобразить как показано на рисунке \ref{fig:pic1_4}. Первый (левый) слой соответствует требованиям, предъявляемым к разрабатываемому комплексу моделей, методики и средств; второй (средний) – принципам, лежащим в основе предлагаемого подхода; третий (правый) – задачам, раскрывающим суть реализации принципов.
\iffalse
\begin{figure}[H]
  \centering
  \includegraphics[width=0.9\textwidth]{man-source/images/ch1/pic1_4_left.png}
  \caption{Структура диссертационной работы}
  \label{fig:pic1_4}
\end{figure}
\fi

В рамках \textbf{второй главы} предложена модель взаимодействия параллельных асинхронных информационных процессов в общей семантической памяти, определяющая принцип коммуникации агентов, выполняющих указанные процессы, включающая средства синхронизации процессов на основе механизма блокировок элементов семантической памяти и обеспечивающая возможность спецификации планируемых блокировок, а также выявления и устранения взаимоблокировок.

Графическая иллюстрация модели взаимодействия процессов показана на рисунке \ref{fig:pic_proc_model}. Как видно из рисунка, в соответствии с изложенными выше принципами агенты не обмениваются сообщениями напрямую, коммуникация между агентами осуществляется посредством спецификации выполняемых ими информационных процессов. В свою очередь, синхронизация выполнения параллельных информационных процессов осуществляется с использованием механизма блокировок элементов семантической памяти. Спецификация каждого такого процесса фиксируется в семантической памяти. Каждый агент также имеет соответствующую спецификацию, которая является частью базы знаний системы и содержит сведения об условиях инициирования агента, возможных результатах его работы, ключевых элементах и т. д.

\begin{figure}[H]
  \centering
  \includegraphics[width=0.8\textwidth]{man-source/images/ch2/pic_proc_model.pdf}
  \caption{Модель взаимодействия процессов в семантической памяти}
  \label{fig:pic_proc_model}
\end{figure}

Формально модель информационных процессов в семантической памяти задается следующим образом:

\begin{equation} 
\label{<eq2_1>} 
M_{IPM} = \{M_A, M_S, M_{SYNC}, M_{SCP}\},
\end{equation} 

\noindent \hangindent=21mm \hangafter=1
где $M_A$ – модель деятельности, выполняемой различными субъектами (агентами) в памяти компьютерной системы и за ее пределами;

\parindent=9mm \hangindent=23mm \hangafter=1
$M_S$ – модель субъекта (агента), осуществляющего преобразования в семантической памяти компьютерной системы;

\parindent=9mm \hangindent=31mm \hangafter=1
$M_{SYNC}$ – модель синхронизации выполнения процессов в семантической памяти компьютерной системы;

\parindent=9mm
$M_{SCP}$ – модель языка SCP.

\medskip
\parindent=1cm

С точки зрения принципов синхронизации были выделены три уровня агентов, работающих над общей семантической памятью (sc-агентов): \mbox{sc-агенты} интерпретации программ на языке SCP (scp-программ), программные \mbox{sc-агенты} и \mbox{sc-метаагенты}. Такой разделение позволяет четко разграничить платформенно-зависимую и платформенно-независимую часть решателя, а также обеспечить возможность выявления и устранения взаимоблокировок за счет использования sc-метаагентов.

Далее в рамках второй главы предложена агентно-ориентированная модель гибридного решателя задач, рассматривающая каждый такой решатель как иерархическую систему агентов, управляемых ситуациями и событиями в общей семантической памяти, обеспечивающая модифицируемость решателей, а также возможность решения задач, требующих совместного использования различных методов решения задач. Согласно данной модели, процесс решения интеллектуальной системой любой задачи предлагается делить на \textit{логически атомарные действия}, т. е. такие действия, выполнение которых не зависит от того, какие действия выполняются до или после них, и от того, частью каких более сложных действий они являются. Каждому классу логически атомарных действий ставится в соответствие агент решателя, таким образом обеспечивается независимость агентов друг от друга, что обеспечивает модифицируемость решателей и дает возможность накапливать и повторно использовать одни и те же агенты в составе разных решателей. Таким образом, sc-агент можно определить как компонент решателя, реагирующий на события в семантической памяти и способный выполнять действия, принадлежащие определенному классу \textit{логически атомарных действий}.

Формально семантическая модель гибридного решателя задач задается следующим образом:

\vspace{-4mm}

\begin{equation}
    \label{<eq2_6>} 
    M_{IPS} = \{AG_{NA}, AG_A, AG_R\},	
\end{equation}

\noindent \hangindent=29mm \hangafter=1
где $AG_{NA}$ – множество неатомарных абстрактных sc-агентов, входящих в состав решателя, т. е. коллективных абстрактных sc-агентов, которые декомпозируются на более простые;

\parindent=9mm \hangindent=25mm \hangafter=1
$AG_A$ – множество атомарных абстрактных sc-агентов, входящих в состав решателя, т. е. таких абстрактных sc-агентов, которые не имеют в своем составе других абстрактных sc-агентов;

% (за исключением агентов scp-интепретатора, в случае, если программа атомарного абстрактного sc-агента реализована на языке SCP)

\parindent=9mm \hangindent=25mm \hangafter=1
$AG_R$ – множество понятий, специфицирующих абстрактные sc-агенты в составе решателя, в том числе описывающих декомпозицию неатомарных агентов на атомарные.

\medskip
\parindent=1cm

Под абстрактным sc-агентом понимается класс функционально эквивалентных sc-агентов, для каждого из которых в общем случае уточняется способ его реализации (программа sc-агента). 

%Связь между предложенными моделями может быть проиллюстрирована как показано на рисунке \ref{fig:pic_ips}.

%В левой части рисунка \ref{fig:pic_ips} показан пример гибридного решателя, который декомпозируется на sc-агенты (атомарные и неатомарный), каждому из которых в конечном итоге соответствует некоторая scp-программа. В свою очередь, каждой из программ может соответствовать несколько процессов в семантической памяти, которые показаны в правой части рисунка. Каждый из процессов специфицируется в той же памяти, в частности, указываются соответствующие процессу блокировки.

Выделено несколько уровней детализации гибридного решателя задач (рисунок \ref{fig:pic_kpm_hier}): уровень агентов scp-интерпретатора, реализуемых на уровне платформы; уровень атомарных программных sc-агентов, реализуемых на уровне платформы; уровень атомарных программных sc-агентов, реализуемых на языке SCP; уровень неатомарных программных sc-агентов, иерархия которых в общем случае может иметь произвольное число уровней, самым верхним из которых является уровень самого гибридного решателя, который также трактуется как неатомарный sc-агент. При этом некоторые sc-агенты могут входить в состав нескольких неатомарных sc-агентов, в том числе расположенных на разных уровнях иерархии. Благодаря этому устраняется дублирование функционально эквивалентных агентов в составе решателя.

\begin{figure}[H]
    \centering
    \includegraphics[width=0.9\textwidth]{man-source/images/ch2/pic_kpm_hier.pdf}
    \caption{Иерархическая агентно-ориентированная модель решателя}
    \label{fig:pic_kpm_hier}
\end{figure}

\iffalse
\begin{figure}[H]
    \centering
    \includegraphics[width=\textwidth]{man-source/images/ch2/pic_ips.png}
    \caption{Детализированная схема обработки информации в интеллектуальной системе}
    \label{fig:pic_ips}
\end{figure}
\fi

\vspace{-3mm}

Как видно из рисунка, предложенный подход позволяет абстрагировать принципы построения решателя как системы sc-агентов от того, каким образом реализована программа того или иного атомарного абстрактного sc-агента в такой системе. Кроме того, иерархический подход к построению решателей позволяет обеспечить удобство проектирования и модификации решателя за счет возможности независимого проектирования и отладки компонентов решателя на разных уровнях иерархии, а также возможности рассматривать решатель на разных уровнях детализации. 

За счет иерархической структуры решателя и независимости его компонентов друг от друга предложенный подход к построению решателей задач дает возможность интегрировать различные модели решения задач в рамках одного решателя, а также легко модифицировать решатель, в том числе -- добавлять в его состав компоненты, реализующие новые, ранее не представленные в системе модели решения задач. В предложенном подходе добавление новой модели решения задач сводится к добавлению в иерархическую структуру решателя нового sc-агента (как правило, неатомарного) и добавлению в базу знаний спецификации новых видов знаний, используемых в рамках вводимой модели решения задач, например, средств представления в базе знаний искусственных нейронных сетей или логических утверждений.

В рамках \textbf{третьей главы} предложена методика построения и модификации гибридных решателей задач, построенных на основе предложенной модели. Предложенная методика предполагает проектирование иерархической структуры решателя методом <<сверху вниз>>. Наличие такой методики позволяет автоматизировать процесс построения и модификации гибридных решателей и снизить требования к их разработчикам. Указанная методика основана на формальной онтологии действий разработчиков решателей и предполагает поэтапное проектирование гибридных решателей и возможность независимой отладки и верификации компонентов решателя на нескольких уровнях. При этом предполагается на каждом из таких уровней использовать компонентный подход. На рисунке~\ref{fig:pic3_2} представлен перечень этапов, которые включает предлагаемая методика, с указанием последовательности их выполнения. Серым фоном отмечены те этапы, которые частично автоматизированы в рамках предложенных в работе средств.

\begin{figure}[H]
    \centering
    \includegraphics[width=0.8\textwidth]{man-source/images/ch3/pic3_2_small.png}
    \caption{Этапы методики построения и модификации гибридного решателя задач}
    \label{fig:pic3_2}
\end{figure}

\vspace{-3mm}

Разработаны средства автоматизации и информационной поддержки процесса построения и модификации гибридных решателей задач, включающие в себя систему автоматизации процесса построения и модификации решателей (САПМР) и подсистему информационной поддержки и консультационного обслуживания разработчиков решателей в рамках метасистемы IMS. Решатели задач каждой из подсистем построены на основе предложенной модели решателя, что обеспечивает их модифицируемость. Разработана библиотека многократно используемых компонентов решателей задач, включающая набор компонентов, языковые средства их спецификации и средства автоматизации поиска компонентов на основе заданной спецификации.

Рассматриваемая система поддержки построения и модификации решателей задач сама по себе также является системой, построенной по технологии OSTIS (ostis-системой), и имеет соответствующую структуру, которую графически можно изобразить следующим образом (рисунок~\ref{fig:pic3_3}):

\begin{figure}[H]
    \centering
    \includegraphics[width=0.8\textwidth]{man-source/images/ch3/pic3_3.pdf}
    \caption{Структура САПМР}
    \label{fig:pic3_3}
\end{figure}

\vspace{-3mm}

В рамках \textbf{четвертой главы} описана реализация разработанных ранее модели интерпретатора программ, ориентированных на обработку знаний, и модели системы автоматизации процесса построения и модификации гибридных решателей задач. Описана реализация решателей задач ряда прототипов интеллектуальных систем с использованием предложенных моделей и средств. В процессе построения указанных решателей произведено стартовое наполнение библиотеки многократно используемых компонентов. 

Достоинства предложенной модели гибридного решателя задач можно показать на примере разработанного с ее использованием решателя задач интеллектуальной справочной системы по геометрии Евклида (рисунок \ref{fig:pic_geom_solver}).

\begin{figure}[H]
    \centering
    \includegraphics[width=0.77\textwidth]{man-source/images/ch4/pic_geom_solver_dots.png}
    \caption{Структура модели гибридного решателя на примере геометрии}
    \label{fig:pic_geom_solver}
\end{figure}

\vspace{-3mm}

Как видно из рисунка, такая структура решателя позволяет при необходимости легко расширять функциональные возможности решателя на разных уровнях, не затрагивая при этом другие компоненты решателя. Так например, можно реализовать в рамках того же решателя другие стратегии решения задач, расширить число типов интерпретируемых логических правил и арифметических операций.

Была произведена оценка трудоемкости разработки решателей с учетом использования разработанных ранее компонентов. Для этого были выделены классы агентов по трудоемкости разработки (поисковый агент, усложненный поисковый агент, агент логического вывода, агент расчета операций), для каждого из классов экспериментальным путем были установлены нормы времени на разработку и отладку.

Расчет показал, что даже при использовании пессимистической оценки эффективности использование предложенной методики на сегодняшний день позволит сократить продолжительность разработки для каждого решателя как минимум на 30,92 \% за счет заимствования разработанных ранее агентов, что подтверждает эффективность применения данной методики. Средний процент заимствования агентов для существующих систем составляет 42,34 \%. 

Кроме того, был проведен анализ использования библиотеки многократно используемых компонентов решателей, разработанных с использованием предложенных моделей, методики и средств, по числу заимствованных агентов без учета их сложности. Было установлено, что текущая версия библиотеки многократно используемых компонентов позволяет сократить число агентов, разрабатываемых для каждой системы, не менее чем на 37~\%.

\bigskip
\centerline{\bf ЗАКЛЮЧЕНИЕ}
\smallskip
{\bf Основные научные результаты диссертации}
\smallskip

\begin{enumerate}[wide, labelindent=10mm]

\item Выявлена и доказана эквивалентность максимизации функции правдоподобия распределения входных данных, минимизации суммарной квадратичной ошибки сети при использовании линейных нейронов и минимизации кросс-энтропийной функции ошибки сети в пространстве синаптических связей ограниченной машины Больцмана. Из полученных теоретических результатов следует, что природа неконтролируемого обучения в RBM-сети является идентичной при использовании различных целевых функций (\cite{2-A}, \cite{3-A}, \cite{4-A}, \cite{5-A}, \cite{10-A}, \cite{12-A}, \cite{13-A});
\item Разработан метод неконтролируемого предобучения глубоких нейронных сетей, базирующийся на минимизации квадратичной ошибки сети в скрытом и видимом слоях RBM-машины, что позволяет учитывать нелинейную природу нейронных элементов. Разработанный метод применен для обучения глубоких полносвязных и сверточных архитектур нейронных сетей и протестирован на выборках MNIST, CIFAR-10, CIFAR-100. Показано, что предложенный метод обладает большей эффективностью, чем классический (\cite{1-A}, \cite{2-A}, \cite{3-A}, \cite{4-A}, \cite{5-A}, \cite{10-A}, \cite{12-A}, \cite{13-A}, \cite{17-A}, \cite{18-A}, \cite{19-A}, \cite{20-A}, \cite{21-A});
\item Предложен алгоритм редуцирования параметров глубокой нейронной сети, базирующийся на неконтролируемом предобучении сети и позволяющий сократить количество настраиваемых параметров сети и упростить ее архитектуру без потери обобщающей способности. Проведены вычислительные эксперименты, доказывающие эффективность предложенного метода (\cite{11-A}, \cite{16-A}, \cite{30-A});
\item Разработана нейросетевая система распознавания маркировки продукта на производственной линии, базирующаяся на интеграции различных сверточных нейронных сетей. Предложенный конвейер представляет собой цепочку взаимодействующих нейросетевых моделей, детекторов и классификаторов, каждая из которых решает отдельную задачу обнаружения и распознавания определенного типа маркировки либо ее части. Подобная архитектура позволила осуществлять распознавание различных типов маркировок с возможностью изменения и добавления новых типов благодаря простой модульной структуре. Типы нейросетей, используемых для построения конвейера, позволяют осуществлять обработку изображения в реальном времени (\cite{6-A}, \cite{7-A}, \cite{8-A}, \cite{22-A}, \cite{23-A}, \cite{24-A}, \cite{25-A}, \cite{26-A}, \cite{27-A});
% \item Разработан нейросетевой алгоритм детекции и распознавания лиц на фото и видеоизображениях, базирующийся на использовании предобученных сверточных нейронных сетей и позволяющий пополнять базу знаний интеллектуальной системы. Применение подобного подхода позволило добиться простой расширяемости системы распознавания лиц и высокой точности распознавания лиц, уже включенных в базу [\cite{28-A}, \cite{29-A}];
\item Разработана нейросетевая система обнаружения солнечных панелей на аэрофотоснимках, базирующаяся на применении предобученных сверточных нейронных сетей и позволяющая обнаруживать солнечные панели с точностью 87,46\% с возможностью использования фотографий низкого разрешения (\cite{9-A}, \cite{14-A}, \cite{15-A}).

% \item Введено понятие гибридного решателя задач, обоснована актуальность согласованного использования нескольких моделей решения задач при решении комплексных задач. Сформулирована проблема совместимости различных моделей решения задач, препятствующая созданию гибридных решателей и технологий их разработки. Предложен ряд принципов, лежащих в основе комплекса моделей, методики и средств, который предлагается разрабатывать как часть технологии OSTIS. В качестве основы для построения гибридных решателей задач предлагается использовать вариант реализации многоагентного подхода, при котором агенты взаимодействуют между собой исключительно путем спецификации информационных процессов, выполняемых агентами в семантической памяти. \cite{7-A,3-A,12-A,13-A,14-A,18-A,19-A,26-A,34-A,36-A,37-A}.

% \item Предложена агентно-ориентированная модель гибридного решателя задач, рассматривающая каждый такой решатель как иерархическую систему агентов, управляемых ситуациями и событиями в общей семантической памяти,
% обеспечивающая модифицируемость таких решателей задач, а также возможность решения задач, требующих совместного использования различных методов решения задач. Предложенная модель позволяет рассматривать разрабатываемый решатель задач на различных уровнях детализации, что обеспечивает возможность поэтапного проектирования решателей, а также их модифицируемость. На основе предложенной модели гибридного решателя задач построена агентно-ориентированная модель интерпретатора базового языка программирования, ориентированного на обработку знаний \cite{4-A,5-A,1-A,3-A,14-A,15-A,16-A,18-A,20-A,21-A,26-A,28-A,31-A}.

% \item Предложена модель взаимодействия параллельных асинхронных информационных процессов в общей семантической памяти, определяющая принцип коммуникации агентов, выполняющих указанные процессы, включающая средства синхронизации процессов на основе механизма блокировок элементов семантической памяти, и обеспечивающая возможность спецификации планируемых блокировок, а также выявления и устранения взаимоблокировок. Уточнено понятие агента, выполняющего преобразования в семантической памяти. Разработана классификация агентов и средства их спецификации \cite{4-A,5-A,1-A,3-A,24-A,25-A,26-A,28-A,30-A,33-A}.

% \item Разработана методика построения и модификации гибридных решателей задач, построенных на основе предложенной модели решателя. В основе методики лежит формальная онтология действий разработчиков таких решателей. Наличие такой методики позволяет автоматизировать процесс построения и модификации решателей и снизить требования к их разработчикам. Указанная методика предполагает поэтапное проектирование гибридных решателей с учетом повторного использования разработанных ранее компонентов и возможности независимой отладки и верификации компонентов решателя на нескольких уровнях. \cite{6-A,8-A,2-A,11-A,14-A,15-A,17-A,23-A,26-A,32-A,35-A}.

% \item Разработаны средства автоматизации и информационной поддержки процесса построения и модификации гибридных решателей задач, включающие в себя систему автоматизации процесса построения и модификации решателей и подсистему информационной поддержки разработчиков решателей в рамках метасистемы IMS. Решатели задач каждой из подсистем построены на основе предложенной модели решателя, что обеспечивает их модифицируемость. Разработана библиотека многократно используемых компонентов решателей задач, включающая набор компонентов, языковые средства их спецификации и средства автоматизации поиска компонентов на основе заданной спецификации \cite{4-A,6-A,8-A,15-A,17-A,22-A,26-A}.

% \item С использованием языка C реализована разработанная агентно-ориентированная модель интерпретатора базового языка программирования, ориентированного на обработку знаний. Все описанные в рамках диссертационной работы модели формализованы и включены в состав соответствующих разделов базы знаний разрабатываемой метасистемы IMS, таким образом обеспечена информационная поддержка разработчиков решателей.
% С использованием предложенных моделей и средств разработаны решатели задач для ряда прикладных интеллектуальных систем. В процессе построения указанных решателей произведено стартовое наполнение библиотеки многократно используемых sc-агентов, показана эффективность предложенной методики с учетом текущей версии библиотеки. 
% Сведения об использовании результатов диссертационной работы отражены в соответствующих актах внедрения, приведенных в приложении к диссертации \cite{8-A,9-A,10-A,13-A,20-A,27-A,29-A,32-A,36-A}.

\end{enumerate}

\smallskip {\bf Рекомендации по практическому использованию результатов}\medskip


%Результаты исследований использованы для практической реализации интеллектуальной системы видеонаблюдения реального времени. 
Научные и практические результаты диссертационной работы использованы в научно-исследовательских работах, учебном процессе, а также в ряде прикладных систем.

Разработанные алгоритмы обнаружения и локализации солнечных панелей на аэрофотоснимках, методы предобучения нейросетевых моделей и инструментальные средства внедрены и используются в компании ООО~<<Intelligent Semantic Systems>> при разработке интеллектуальных систем в составе подсистем компьютерного зрения.

Предложенный метод предобучения глубоких нейронных сетей применялся при обучении модели классификатора кадров для нейросетевой системы распознавания маркировки продукта на производственной линии ОАО~<<Савушкин продукт>>.

Кроме того, научные и практические результаты диссертационной работы используются в учебном процессе учреждения образования <<Брестский государственный технический университет>>.

% Разработанные модели, средства и их программная реализация могут быть использованы при разработке решателей задач интеллектуальных систем различного назначения, как инструментальных, так и прикладных. 

Реализация программных модулей осуществлялась с использованием свободного ПО и может быть непосредственно применена при разработке отечественных проектов в области компьютерного зрения без необходимости приобретения дорогостоящих программных средств.

% Научные и практические результаты диссертационной работы используются в учебном процессе учреждения образования <<Белорусский государственный университет информатики и радиоэлектроники>>, в НИЛ 3.7 при выполнении работ по договору БРФФИ--РФФИ (№ ГР 20164340), а также на предприятиях ОАО <<Савушкин продукт>> при разработке решателя задач системы автоматизации рецептурного производства, ООО <<ФордэКонсалтинг>> при разработке программной системы обслуживания клиентов розничной торговли, ООО <<Кинросс-ресерч>> при разработке системы автоматизации и оптимизации строительного проектирования, ФГБУН Институт систем энергетики им. Л. А. Мелентьева СО РАН при разработке решателя в области энергетики.

%\newpage

\def\selectlanguageifdefined#1{
\expandafter\ifx\csname date#1\endcsname\relax
\else\language\csname l@#1\endcsname\fi}

\bigskip
\centerline{\bf СПИСОК ПУБЛИКАЦИЙ СОИСКАТЕЛЯ ПО ТЕМЕ ДИССЕРТАЦИИ}

\vspace{1mm}
{\bf Статьи в рецензируемых научных журналах}
\vspace{2mm}

\begin{enumerate}[wide, labelindent=10mm]

%\section*{Статьи в рецензируемых научных журналах}
\ifx\isabstract\undefined 
\section* {Статьи в научных рецензируемых изданиях, включенных в перечень изданий, и в иностранных научных изданиях}
\fi

\bibitem{2-A}
\selectlanguageifdefined{russian}
Головко,~В.А. Персептроны и нейронные сети глубокого доверия : обучение и применение /~В.А.~Головко, А.А.~Крощенко
\newblock //~Вестник Брестского государственного технического университета. Физика, математика, информатика. ---
\newblock Брест,~2014. ---
\newblock №~5~(89). ---
\newblock {\cyr\CYRS.}~2--12.

\bibitem{4-A}
\selectlanguageifdefined{english}
Golovko,~V. The Nature of Unsupervised Learning in Deep Neural Networks : A New Understanding and Novel Approach /~V.~Golovko, A.~Kroshchanka, D.~Treadwell
\newblock //~Optical Memory and Neural Networks. ---
\newblock 2016. ---
\newblock Vol.~25, №~3. ---
\newblock P.~127--141.

\bibitem{5-A}
\selectlanguageifdefined{russian}
Головко,~В.А. Теория глубокого обучения : конвенциальный и новый подход /~В.А.~Головко, А.А.~Крощенко, М.В.~Хацкевич
\newblock //~Вестник Брестского государственного технического университета. Физика, математика, информатика. ---
\newblock Брест, 2016. ---
\newblock №~5~(101). ---
\newblock {\cyr\CYRS.}~7--16.

\bibitem{7-A}
\selectlanguageifdefined{russian}
Крощенко,~А.А. Реализация нейросетевой системы распознавания маркировки продукции /~А.А.~Крощенко, В.А.~Головко
\newblock //~Вестник Брестского государственного технического университета. Физика, математика, информатика. ---
\newblock Брест, 2019. ---
\newblock №~5~(118). ---
\newblock {\cyr\CYRS.}~9--12.

\bibitem{10-A}
\selectlanguageifdefined{english}
Golovko,~V.A.  Deep Neural Networks : Selected Aspects of Learning and Application /~V.A.~Golovko, A.A.~Kroshchanka, E.V.~Mikhno
\newblock //~Pattern Recognition and Image Analysis. ---
\newblock 2021. ---
\newblock Vol.~31, №~1. ---
\newblock P.~132--143.

\bibitem{29-A}
\selectlanguageifdefined{english}
Kroshchanka,~A. Neural network component of the product marking recognition system on the production line /~A.~Kroshchanka, D.~Ivaniuk
\newblock //~Open Semantic Technologies for Intelligent Systems : research papers collection
\newblock /~Belar. State Univ. of Informatics and Radioelectr.; ed. : V.V.~Golenkov (ed.-in-chief) [et al.]. ---
\newblock Minsk, 2021. ---
\newblock Iss.~5. ---
\newblock P.~219--224.

\bibitem{11-A}
\selectlanguageifdefined{english}
Kroshchanka,~A.A. Method for Reducing Neural-Network Models of Computer Vision /~A.A.~Kroshchanka, V.A.~Golovko, M.~Chodyka
\newblock //~Pattern Recognition and Image Analysis. ---
\newblock Berlin, Heidelberg : Springer-Verlag, 2022. ---
\newblock Vol.~32, №~2. ---
\newblock P.~294--300.

\bibitem{30-A}
\selectlanguageifdefined{english}
Kroshchanka,~A.A. Reduction of Neural Network Models in Intelligent Computer Systems of a New Generation /~A.~Kroshchanka
\newblock //~Open Semantic Technologies for Intelligent Systems (OSTIS-2023) : research papers collection
\newblock /~Belar. State Univ. of Informatics and Radioelectr.; ed. : V.V.~Golenkov (ed.-in-chief) [et al.]. ---
\newblock Minsk, 2023. ---
\newblock Iss. 7. ---
\newblock P.~127--132.

\ifx\isabstract\undefined 
\begin{center}
\vspace{3mm}
{\bf Статьи в других научных изданиях}
\vspace{3mm}
\end{center}
\else
\vspace{2mm}
{\bf Статьи в других научных изданиях}
\vspace{2mm}
\fi

\bibitem{3-A}
\selectlanguageifdefined{ukrainian}
Головко,~В.А. Метод обучения нейронной сети глубокого доверия и применение для визуализации данных /~В.А.~Головко, А.А.~Крощенко
\newblock //~Комп'ютерно-інтегровані технології: освіта, наука, виробництво. ---
\newblock \mbox{Луцьк},~2015. ---
\newblock №~19. ---
\newblock {\cyr\CYRS.}~6--12.

\bibitem{6-A}
\selectlanguageifdefined{russian}
Головко,~В.А. Интеграция искусственных нейронных сетей с базами знаний /~В.А.~Головко, В.В.~Голенков, В.П.~Ивашенко, В.В.~Таберко, Д.С.~Иванюк, А.А.~Крощенко, М.В.~Ковалёв
\newblock //~Онтология проектирования. ---
\newblock 2018. ---
\newblock Т.~8. ---
\newblock №~3~(29). ---
\newblock {\cyr\CYRS.}~366--386.

\bibitem{25-A}
\selectlanguageifdefined{english}
Golovko,~V. Principles of decision-making systems building based on the integration of neural networks and semantic models /~V.~Golovko, A.~Kroshchanka, V.~Ivashenko, M.~Kovalev, V.~Taberko, D.~Ivaniuk
\newblock //~Open Semantic Technologies for Intelligent Systems (OSTIS-2019) : research papers collection
\newblock /~Belar. State Univ. of Informatics and Radioelectr.; ed. : V.V.~Golenkov (ed.-in-chief) [et al.]. ---
\newblock Minsk, 2019. ---
\newblock Iss.~3. ---
\newblock P.~91--102.

\bibitem{27-A}
\selectlanguageifdefined{english}
Golovko,~V. Implementation of an intelligent decision support system to accompany the manufacturing process /~V.~Golovko, A.~Kroshchanka, M.~Kovalev, V.~Taberko, D.~Ivaniuk
\newblock //~Open Semantic Technologies for Intelligent Systems (OSTIS-2020) : research papers collection
\newblock /~Belar. State Univ. of Informatics and Radioelectr.; ed. : V.V.~Golenkov (ed.-in-chief) [et al.]. ---
\newblock Minsk, 2020. ---
\newblock Iss.~4. ---
\newblock P.~175--182.

\bibitem{9-A}
\selectlanguageifdefined{english}
Golovko,~V. Deep Convolutional Neural Network for Detection of Solar Panels /~V.~Golovko, A.~Kroshchanka, E.~Mikhno, M.~Komar, A.~Sachenko
\newblock //~Data-Centric Business and Applications. ICT Systems-Theory, Radio-Electronics, Information Technologies and Cybersecurity. ---
\newblock Cham: Springer International Publishing, 2021. ---
\newblock P.~371--389.

%\section*{Статьи в сборниках материалов научных конференций}
\ifx\isabstract\undefined 
\begin{center}
\vspace{3mm}
{\bf Статьи в сборниках материалов научных конференций}
\vspace{3mm}
\end{center}
\else
\vspace{2mm}
{\bf Статьи в сборниках материалов научных конференций}
\vspace{2mm}
\fi

\bibitem{1-A}
\selectlanguageifdefined{english}
Golovko,~V. A Learning Technique for Deep Belief Neural Networks /~V.~Golovko, A.~Kroshchanka, U.~Rubanau, S.~Jankowski
\newblock //~Neural Networks and Artificial Intelligence : proc. of the 8th Internat. Conf. ICNNAI 2014, Brest, Belarus, June 3-6, 2014
\newblock /~V.~Golovko, A.~Imada (eds.). ---
\newblock Springer, 2014. ---
\newblock Vol.~440. Communication in Computer and Information Science. ---
\newblock P.~136--146.

\bibitem{19-A}
\selectlanguageifdefined{russian}
Головко,~В.А. Применение нейронных сетей глубокого доверия для выделения семантически значимых признаков /~В.А.~Головко, А.А.~Крощенко
\newblock //~Открытые семантические технологии проектирования интеллектуальных систем : материалы V Междунар. науч.-технич. конф. OSTIS-2015, Минск, 19-21 февраля 2015 г.
\newblock /~УО <<Бел. гос. ун-т информатики и радиоэлектроники>>, ГУ <<Администрация Парка высоких технологий>>; редкол.: В.В.~Голенков (отв. ред.) [и др.]. ---
\newblock Минск, 2015. ---
\newblock P.~481--486.

\bibitem{12-A}
\selectlanguageifdefined{english}
Golovko,~V. A New Technique for Restricted Boltzmann Machine Learning [Electronic resource] /~V.~Golovko, A.~Kroshchanka, V.~Turchenko, S.~Jankowski, D.~Treadwell
\newblock //~Proceedings of the 8th IEEE International Conference of Intelligent Data Acquisition and Advanced Computing Systems~: Technology and Applications (IDAACS), Warsaw, Poland, Sept. 24-26, 2015
\newblock /~Research Institute for Intelligent Computer Systems, Ternopil National Economic University and V.M.~Glushkov Inst. of Cybernetics, National Academy for Sciences of Ukraine, Warsaw University of Technology. ---
\newblock P.~182--186. ---
\newblock Mode of access~: \url{https://ieeexplore.ieee.org/document/7340725}. ---
\newblock Date of access~: 05.06.2023.

\bibitem{20-A}
\selectlanguageifdefined{russian}
Крощенко,~А.А. Применение нейронных сетей глубокого доверия в интеллектуальном анализе данных /~А.А.~Крощенко
\newblock //~Современные проблемы математики и вычислительной техники : сб. материалов IX Респ. науч. конф. молодых ученых и студентов, Брест, 19-21 ноября 2015 г.
\newblock /~Мин. обр. Респ. Бел., УО <<Брест. гос. технич. ун-т>>; редкол.: В.С. Рубанов (гл. ред.) [и др.]. ---
\newblock Брест, 2015. ---
\newblock {\cyr\CYRS.}~12--14.

\bibitem{13-A}
\selectlanguageifdefined{english}
Golovko,~V. Theoretical Notes on Unsupervised Learning in Deep Neural Networks [Electronic resource] /~V.~Golovko, A.~Kroshchanka
\newblock //~Proceedings of the 8th Internat. Joint Conf. on Computational Intelligence (IJCCI 2016), Porto, Portugal, Nov. 9-11 2016. ---
\newblock P.~91--96. ---
\newblock Mode of access~: \url{https://scitepress.org/papers/2016/60843/60843.pdf}. ---
\newblock Date of access~: 05.06.2023.

\bibitem{14-A}
\selectlanguageifdefined{russian}
Golovko, V. Convolutional Neural Network Based Solar Photovoltaic Panel Detection in Satellite Photos [Electronic resource] /~V.~Golovko, S.~Bezobrazov, A.~Kroshchanka, A.~Sachenko, M.~Komar, A.~Karachka
\newblock //~The 9th IEEE Internat. Conf. on Intelligent Data Acquisition and Advanced Computing Systems: Technology and Applications (IDAACS), Bucharest, Romania, Sept. 21-23 2017. ---
\newblock P.~14--19. ---
\newblock Mode of access~: \url{https://ieeexplore.ieee.org/document/8094501}. ---
\newblock Date of access~: 05.06.2023.

\bibitem{23-A}
\selectlanguageifdefined{english}
Golovko,~V.A. Integration of artificial neural networks and knowledge bases /~V.A.~Golovko, A.A.~Kroshchanka, V.V.~Golenkov, V.P.~Ivashenko, M.V.~Kovalev, V.V.~Taberko, D.S.~Ivaniuk
\newblock //~Открытые семантические технологии проектирования интеллектуальных систем : материалы VIII Междунар. науч. технич. конф., Минск, 15-17 февраля 2018 г.
\newblock /~Мин. обр. Респ. Бел., УО <<Бел. гос. ун-т информатики и радиоэлектроники>>; редкол.: В.В. Голенков (гл. ред.) [и др.]. ---
\newblock Минск, 2018. ---
\newblock Вып.~2. ---
\newblock C.~133--146.

\bibitem{24-A}
\selectlanguageifdefined{russian}
Головко,~В.А. Нейросетевые модели глубокого обучения для решения задач распознавания объектов на изображении /~В.А.~Головко, А.А.~Крощенко
\newblock //~Вычислительные методы, модели и образовательные технологии : сб. материалов VII Междунар. науч.-практич. конф., Брест, 19 октября 2018 г.
\newblock /~Брест. гос. ун-т имени А.С.~Пушкина; под общ. ред. А.А.~Козинского. ---
\newblock Брест, 2018. ---
\newblock {\cyr\CYRS.}~3--5. 

\bibitem{15-A}
\selectlanguageifdefined{russian}
Golovko, V. Development of Solar Panels Detector [Electronic~resource] /~V.~Golovko, A.~Kroshchanka, S.~Bezobrazov, A.~Sachenko, M.~Komar, O.~Novosad
\newblock //~2018 International Scientific-Practical Conference <<Problems of Infocommunications. Science and Technology (PIC S\&T 2018)>>, Kharkiv, Ukraine, 9-12 October 2018
\newblock /~Institute of Electrical and Electronics Engineers, Inc. ---
\newblock Mode of access~: \url{https://www.semanticscholar.org/paper/Development-of-Solar-Panels-Detector-Golovko-Kroshchanka/914c0b8c159c64100609fc8455636b1e3f8568cb}. ---
%International Scientific-Practical Conference Problems of Infocommunications. Science and Technology (PIC S\&T). ---
\newblock Date of access~: 05.06.2023.
%\newblock IEEE, 2018. ---
%\newblock P.~761--764.

\bibitem{8-A}
\selectlanguageifdefined{english}
Golovko,~V. Brands and caps labeling recognition in images using deep learning [Electronic~resource] /~V.~Golovko, A.~Kroshchanka, E.~Mikhno
\newblock //~Pattern Recognition and Information Processing : revised selected papers of the 14th Internat. Conf. PRIP 2019, Minsk, May~21-23 2019. ---
\newblock Eds. : S.V.~Ablameyko, V.V.~Krasnoproshin, M.M.~Lukashevich. ---
\newblock P.~35--51. ---
\newblock Mode of access~: \url{https://www.researchgate.net/publication/337459978_Brands_and_Caps_Labeling_Recognition_in_Images_Using_Deep_Learning}. ---
\newblock Date of access~: 05.06.2023.

\bibitem{26-A}
\selectlanguageifdefined{russian}
Головко,~В.А. Обнаружение и распознавание маркировки продукции с помощью нейросетевых алгоритмов /~В.А.~Головко, А.А.~Крощенко
\newblock //~Вычислительные методы, модели и образовательные технологии: сб. материалов VII Междунар. науч.-практич. конф., Брест, 18 октября 2019 г.
\newblock /~Брест. гос. ун-т имени А.С.~Пушкина; под общ. ред. А.А.~Козинского. ---
\newblock Брест, 2019. ---
\newblock {\cyr\CYRS.}~3--6. 

\bibitem{28-A}
\selectlanguageifdefined{english}
Golovko,~V. Neuro-Symbolic Artificial Intelligence: Application for Control the Quality of Product Labeling [Electronic~resource] /~V.~Golovko, A.~Kroshchanka, M.~Kovalev, V.~Taberko, D.~Ivaniuk
\newblock //~Open Semantic Technologies for Intelligent Systems : revised selected papers of the 10th Intern. Conf. (OSTIS 2020), Minsk, Febr. 19-22, 2020
\newblock /~Eds. : V. Golenkov, V. Krasnoproshin, V. Golovko, E. Azarov. ---
\newblock P.~81--101. ---
\newblock Mode of access~: \url{https://libeldoc.bsuir.by/bitstream/123456789/42395/1/Golovko_Neuro_Symbolic.pdf}. ---
\newblock Date of access~: 05.06.2023.

\bibitem{16-A}
\selectlanguageifdefined{russian}
Kroshchanka,~A. The Reduction of Fully Connected Neural Network Parameters Using the Pre-training Technique /~A.~Kroshchanka, V.~Golovko
\newblock //~Proceedings of the 11th IEEE Intern. Conf. on Intelligent Data Acquisition and Advanced Computing Systems : Technology and Applications (IDAACS), Cracow, Poland, September 22-25, 2021
\newblock /~Cracow University of Technology [et al.]. ---
\newblock Vol.~2. ---
\newblock P.~937--941. ---
\newblock Mode of access~: \url{https://www.researchgate.net/publication/357613109_The_Reduction_of_Fully_Connected_Neural_Network_Parameters_Using_the_Pre-training_Technique}. ---
\newblock Date of access~: 05.06.2023.

% \bibitem{28-A}
% \selectlanguageifdefined{english}
% Kroshchanka, A. Semantic analysis of the video stream based on neuro-symbolic artificial intelligence~/ A. Kroshchanka, E. Mikhno, M. Kovalev, V. Zahariev, A. Zagorskij~// 
%  Open Semantic Technologies for Intelligent Systems (OSTIS-2021). ---
% \newblock Minsk : BSUIR, 2021. ---
% \newblock №~5. ---
% \newblock P.~193--204.

% \bibitem{29-A}
% \selectlanguageifdefined{russian}
% Головко, В. А. Гибридная интеллектуальная система оценки эмоционального состояния пользователя~/ В. А. Головко, А. А. Крощенко~// 
%  Современные проблемы математики и вычислительной техники : сборник материалов ХII Республиканской научной конференции молодых ученых и студентов, Брест, 18–19 ноября 2021 г.~/ Министерство образования Республики Беларусь, Брестский государственный технический университет ; редкол.: В. А. Головко (гл. ред.) [и др.]. ---
% \newblock Брест : БрГТУ, 2021. ---
% \newblock {\cyr\CYRS.}~10--13.

% \bibitem{24-A}
% \selectlanguageifdefined{russian}
% Формальное семантическое описание целенаправленной деятельности различного вида субъектов~/ Д. В. Шункевич,  А. В. Губаревич, М.~Н.~Святкина, О. Л. Моросин~// 
%  Открытые семантические технологии проектирования интеллектуальных систем (OSTIS-2016) : материалы VI Междунар. науч.-техн. конф., Минск, 18–20 февр. 2016 г.~/ Белорус. гос. ун-т информатики и радиоэлектроники ; редкол.: В. В. Голенков (отв. ред.) [и др.]. ---
% \newblock Минск, 2016. ---
% \newblock {\cyr\CYRS.}~125--136.

% \bibitem{25-A}
% \selectlanguageifdefined{russian}
% Шункевич, Д. В. Взаимодействие асинхронных параллельных процессов обработки знаний в общей семантической памяти~/ Д. В. Шункевич~// 
%  Открытые семантические технологии проектирования интеллектуальных систем (OSTIS-2016) : материалы VI Междунар. науч.-техн. конф., Минск, 18–20 февр. 2016 г.~/ Белорус. гос. ун-т информатики и радиоэлектроники ; редкол.: В. В. Голенков (отв. ред.) [и др.]. ---
% \newblock Минск, 2016. ---
% \newblock {\cyr\CYRS.}~137--144.

% \bibitem{26-A}
% \selectlanguageifdefined{english}
% Shunkevich, D. Ontology-based design of knowledge processing machines~/ D. Shunkevich~// 
%   Открытые семантические технологии проектирования интеллектуальных систем : материалы междунар. науч.-техн. конф., Минск, 16–18 февр. 2017 г.~/ Белорус. гос. ун-т информатики и радиоэлектроники ; редкол.: В. В. Голенков (отв. ред.) [и др.]. ---
% \newblock Минск, 2017. --- Вып. 1. ---
% \newblock P.~73--94.

% \bibitem{27-A}
% \selectlanguageifdefined{english}
% Ontology-based design of batch manufacturing enterprises~/ V.~Golenkov, K.~Rusetski, D.~Shunkevich, I.~Davydenko, V.~Zakharov, V.~Ivashenko, D.~Koronchik, V.~Taberko, D.~Ivanyuk~// 
%   Открытые семантические технологии проектирования интеллектуальных систем : материалы междунар. науч.-техн. конф., Минск, 16–18 февр. 2017 г.~/ Белорус. гос. ун-т информатики и радиоэлектроники ; редкол.: В. В. Голенков (отв. ред.) [и др.]. ---
% \newblock Минск, 2017. --- Вып.~1. ---
% \newblock С.~265--280.

% \bibitem{28-A}
% \selectlanguageifdefined{russian}
% Семантическая модель представления и обработки баз знаний~/ В.~В.~Голенков, Н. А. Гулякина, И. Т. Давыденко, Д. В. Шункевич~// 
%  Аналитика и управление данными в областях с интенсивным использованием данных : сб. науч. тр. XIX Междунар. конф. DAMDID/RCDL’2017, Москва, 10–13 окт. 2017 г. / Федер. исслед. центр <<Информатика и управление>> Рос. акад. наук ; под ред. Л. А. Калиниченко [и др.]. ---
% \newblock М., 2017. ---
% \newblock {\cyr\CYRS.}~412--419.

%\section*{Статьи в сборниках материалов научных конференций}
\ifx\isabstract\undefined 
\begin{center}
\vspace{3mm}
{\bf Тезисы}
\vspace{3mm}
\end{center}
\else
\vspace{2mm}
%\newpage
{\bf Тезисы}
\vspace{2mm}
\fi

\bibitem{17-A}
\selectlanguageifdefined{russian}
Крощенко,~А.А. Методы глубокого обучения нейронных сетей /~А.А.~Крощенко
\newblock //~Вычислительные методы, модели и образовательные технологии : сб. матер. региональной науч.-практич. конф., Брест, 22-23 окт. 2013 г.
\newblock /~Брест. гос. ун-т им. А.С.~Пушкина; под общ. ред. О.В.~Матысика. ---
\newblock Брест, 2013. ---
\newblock {\cyr\CYRS.}~21--22.

\bibitem{18-A}
\selectlanguageifdefined{russian}
Головко,~В.А. Об одном методе обучения нейронных сетей глубокого доверия /~В.А.~Головко, А.А.~Крощенко
\newblock //~Вычислительные методы, модели и образовательные технологии : сб. матер. Междунар. науч.-практич. конф., Брест, 15-16 окт. 2014 г.
\newblock /~Брест. гос. ун-т им. А.С.~Пушкина; под общ. ред. О.В.~Матысика. ---
\newblock Брест, 2014. ---
\newblock {\cyr\CYRS.}~98--99.

\bibitem{21-A}
\selectlanguageifdefined{russian}
Головко,~В.А. Применение нейронных сетей глубокого доверия в интеллектуальном анализе данных /~В.А.~Головко, А.А.~Крощенко
\newblock //~Вычислительные методы, модели и образовательные технологии : сб. матер. Междунар. науч.-практич. конф., Брест, 22 окт. 2015 г.
\newblock /~Брест. гос. ун-т им. А.С.~Пушкина; под общ. ред. О.В.~Матысика. ---
\newblock Брест, 2015. ---
\newblock {\cyr\CYRS.}~97--98.

\bibitem{22-A}
\selectlanguageifdefined{russian}
Крощенко,~А.А. Применение глубокой нейронной сети для решения задачи распознавания образов /~А.А.~Крощенко
\newblock //~Вычислительные методы, модели и образовательные технологии : сб. матер. Междунар. науч.-практич. конф., Брест, 21 окт. 2016 г.
\newblock /~Брест. гос. ун-т им. А.С.~Пушкина; под общ. ред. О.В.~Матысика. ---
\newblock Брест, 2016. ---
\newblock {\cyr\CYRS.}~132--133.

% \bibitem{29-A}
% \selectlanguageifdefined{russian}
% Интеллектуальный решатель задач по геометрии~/ Д. В. Шункевич, С. С.~Заливако, О. Ю.~Савельева, С. С.~Старцев~// 
%   Информационные системы и технологии IST’2010 : материалы VI Междунар. конф., Минск, 24–25 нояб. 2010 г. / Белорус. гос. ун-т [и др.] ; редкол.: А. Н. Курбацкий (отв. ред.) [и др.]. ---
% \newblock Минск, 2010. ---
% \newblock {\cyr\CYRS.}~482--485.

% \bibitem{30-A}
% \selectlanguageifdefined{russian}
% Шункевич, Д. В.  Принципы проектирования и интерпретации программ, ориентированных на обработку семантических сетей~/ Д. В. Шункевич~// 
%  Информационные технологии и системы-2013 (ИТС 2013) : материалы междунар. науч. конф., Минск, 23 окт. 2013 г.~/ Белорус. гос. ун-т информатики и радиоэлектроники ; редкол.: \mbox{Л. Ю.} Шилин (гл. ред) [и др.]. ---
% \newblock Минск, 2013. ---
% \newblock {\cyr\CYRS.}~116--117.

% \bibitem{31-A}
% \selectlanguageifdefined{russian}
% Шункевич, Д. В.  Базовые понятия технологии компонентного проектирования машин обработки знаний систем дистанционного обучения~/ \mbox{Д. В. Шункевич}~// 
%   Дистанционное обучение -- образовательная среда XXI века : материалы VIII междунар. науч.-метод. конф., Минск, 5–6 дек. 2013г. / Белорус. гос. ун-т информатики и радиоэлектроники ; редкол.: Б. В. Никульшин [и др.]. ---
% \newblock Минск, 2013. ---
% \newblock {\cyr\CYRS.}~186--187.

% \bibitem{32-A}
% \selectlanguageifdefined{russian}
% Якимчик, С. В.  Принципы построения решателей задач в прикладных интеллектуальных системах~/ С. В. Якимчик, Д. В. Шункевич~// 
%  Информационные технологии и системы-2014 (ИТС 2014) : материалы междунар. науч. конф., Минск, 29 окт. 2014 г.~/ Белорус. гос. ун-т информатики и радиоэлектроники ; редкол.: \mbox{Л. Ю.} Шилин (гл. ред.) [и др.]. ---
% \newblock Минск, 2014. ---
% \newblock {\cyr\CYRS.}~160--161.

% \bibitem{33-A}
% \selectlanguageifdefined{russian}
% Губаревич, А. В.  Онтология деятельности интеллектуальных агентов над общей памятью~/ А. В. Губаревич, М. Н. Святкина Д. В. Шункевич~// 
%  Информационные технологии и системы-2015 (ИТС 2015) : материалы междунар. науч. конф., Минск, 28 окт. 2015 г.~/ Белорус. гос. ун-т информатики и радиоэлектроники ; редкол.: \mbox{Л. Ю.} Шилин (гл. ред.) [и др.]. ---
% \newblock Минск, 2015. ---
% \newblock {\cyr\CYRS.}~138--139.

% \bibitem{34-A}
% \selectlanguageifdefined{russian}
% Онтологическое моделирование для реализации семантических технологий создания интеллектуальной системы управления жизненным циклом~/ А. В. Федотова, И. Т. Давыденко, М. Н. Святкина, Д. В. Шункевич~// 
%     Информационная безопасность регионов России (ИБРР–2015) : IX С.-Петерб. межрегион. конф., Санкт-Петербург, 28–30 окт. 2015 г. : материалы конф. / \mbox{С.-Петерб.} О-во информатики, вычисл. техники, систем связи и упр ; редкол.: Б. Я. Советов [и др.].  ---
% \newblock СПб., 2015. ---
% \newblock {\cyr\CYRS.}~336.

% \bibitem{35-A}
% \selectlanguageifdefined{russian}
% Шункевич, Д. В.  Унифицированная семантическая модель процесса проектирования машин обработки знаний~/ Д. В. Шункевич, И. Б. Фоминых, О. Л. Моросин~// 
%  Информационные технологии и системы-2016 (ИТС 2016) : материалы междунар. науч. конф., Минск, 26 окт. 2016 г.~/ Белорус. гос. ун-т информатики и радиоэлектроники ; редкол.: \mbox{Л. Ю.} Шилин (гл. ред.) [и др.]. ---
% \newblock Минск, 2016. ---
% \newblock {\cyr\CYRS.}~172--173.

% \bibitem{36-A}
% \selectlanguageifdefined{russian}
% Шункевич, Д. В.  Представление базовых конструкций языка семантических сетей с теоретико-множественной интерпретацией средствами платформы Neo4j~/ Д. В. Шункевич, О. С. Родионова~// 
%  Информационные технологии и системы-2016 (ИТС 2016) : материалы междунар. науч. конф., Минск, 26 окт. 2016 г.~/ Белорус. гос. ун-т информатики и радиоэлектроники ; редкол.: \mbox{Л. Ю.} Шилин (гл. ред.) [и др.]. ---
% \newblock Минск, 2016. ---
% \newblock {\cyr\CYRS.}~174--175.

% \bibitem{37-A}
% \selectlanguageifdefined{russian}
% Голенков, В. В. Принципы построения машин обработки баз знаний~/ В. В. Голенков, Д. В. Шункевич~// 
%  Информационные технологии и системы-2017 (ИТС 2017) : материалы междунар. науч. конф., Минск, 25 окт. 2017 г.~/ Белорус. гос. ун-т информатики и радиоэлектроники ; редкол.: \mbox{Л. Ю.} Шилин (гл. ред.) [и др.]. ---
% \newblock Минск, 2017. ---
% \newblock {\cyr\CYRS.}~134--135.

\end{enumerate}

\newpage
\begin{center}
\bf РЕЗЮМЕ
\\[1mm]\rm Шункевич Даниил Вячеславович\\[1mm] \bf
АГЕНТНО-ОРИЕНТИРОВАННЫЕ РЕШАТЕЛИ ЗАДАЧ ИНТЕЛЛЕКТУАЛЬНЫХ СИСТЕМ
 \end{center}

{\bf Ключевые слова}: решатель задач, база знаний, интеллектуальная система, семантическая модель, семантическая память, многоагентная система.

\textbf{Целью исследования} является разработка комплекса моделей, методики и средств построения и модификации решателей задач интеллектуальных систем.

\textbf{Полученные результаты и их новизна}:
в работе исследованы существующие подходы к построению решателей задач интеллектуальных систем. Разработана агентно-ориентированная модель гибридного решателя задач, рассматривающая каждый такой решатель как коллектив агентов, работающих над общей семантической памятью, позволяющая реализовать и интегрировать различные модели решения задач в рамках гибридного решателя, а также обеспечить модифицируемость таких решателей. Разработана формальная модель взаимодействия параллельных асинхронных информационных процессов в общей семантической памяти, включающая средства их синхронизации на основе механизма блокировок элементов семантической памяти, а также принцип взаимодействия агентов, выполняющих указанные процессы, обеспечивающая возможность спецификации планируемых блокировок, выявления и устранения взаимоблокировок. Разработана методика построения и модификации гибридных решателей задач, основанная на формальной онтологии деятельности разработчиков таких решателей и ориентированная на применение библиотеки многократно используемых компонентов решателей задач. Разработаны средства автоматизации и информационной поддержки процесса построения и модификации гибридных решателей задач, включающие библиотеку многократно используемых компонентов таких решателей и средства автоматизации процесса построения агентов обработки знаний. Указанные средства разработаны с использованием предложенной модели решателя задач.

\textbf{Рекомендации к использованию и область применения}:
разработанные модели, средства и их программная реализация могут быть использованы при разработке решателей задач интеллектуальных систем различного назначения. Полученные в диссертации результаты могут быть использованы в учебном процессе высших учебных заведений для обучения специалистов по специальностям, где предусмотрено чтение курсов, связанных с моделями решения задач в интеллектуальных системах.

\newpage
\begin{center}
\bf РЭЗЮМЭ\\[1mm]\rm Шункевiч Данiiл Вячаслававiч\\[1mm] \bf
АГЕНТНА-АРЫЕНТАВАНЫЯ РАШАЦЕЛI ЗАДАЧ ІНТЭЛЕКТУАЛЬНЫХ СІСТЭМ
\end{center}

{\bf Ключавыя словы}: рашацель задач, база веда\u{у}, інтэлектуальная сістэма, семантычная мадэль, семантычная памяць, шматагентная сістэма.

\textbf{Мэтай работы} з'яўляецца распрацоўка комплексу мадэляў, метаду і сродкаў пабудовы і мадыфікацыі рашацеляў задач інтэлектуальных сістэм.

\textbf{Атрыманыя вынiкi i iх навiзна}:
у працы даследаваны існуючыя падыходы да пабудовы рашацеляў задач інтэлектуальных сістэм. Распрацавана агентна-арыентаваная мадэль гiбрыднага рашацеля задач, якая разглядае кожны такі рашацель як калектыў агентаў, якія працуюць над агульнай семантычнай памяццю, і дазваляе рэалізаваць і інтэграваць розныя мадэлі рашэння задач у рамках гiбрыднага рашацеля, а таксама забяспечыць мадыфікуемасць такiх рашацеляў. Распрацавана фармальная мадэль узаемадзеяння паралельных асінхронных інфармацыйных працэсаў у агульнай семантычнай памяці, якая ўключае сродкі іх сінхранізацыі на аснове механізму блакіровак элементаў семантычнай памяці, а таксама прынцып узаемадзеяння агентаў, якія выконваюць азначаныя працэсы, і забяспечвае магчымасць спецыфікацыі плануемых блакіровак, выяўлення і ліквідацыі ўзаемаблакіровак. Распрацавана методыка пабудовы і мадыфікацыі гiбрыдных рашацеляў задач, заснаваная на фармальнай анталогіі дзейнасці распрацоўшчыкаў такіх рашацеляў і арыентаваная на ўжыванне бібліятэкі шматразова выкарыстоўваемых кампанентаў рашацеляў задач. Распрацаваны сродкі аўтаматызацыі і інфармацыйнай падтрымкі працэсу пабудовы і мадыфікацыі гiбрыдных рашацеляў задач, якія ўключаюць бібліятэку шматразова выкарыстоўваемых кампанентаў азначаных рашацеляў і сродкі аўтаматызацыі працэсу пабудовы агентаў апрацоўкi ведаў. Азначаныя сродкі распрацаваны з выкарыстаннем прапанаванай мадэлі рашацеля задач.

\textbf{Рэкамендацыi па выкарыстаннi i вобласць ужывання}:
распрацаваныя мадэлі, сродкі і іх праграмная рэалізацыя могуць быць выкарыстаныя падчас распрацоўкі рашацеляў задач інтэлектуальных сістэм разнастайнага прызначэння. Атрыманыя ў дысертацыі вынікі могуць быць выкарыстаныя ў навучальным працэсе вышэйшых навучальных устаноў для навучання спецыялістаў па спецыяльнасцях, дзе прадугледжваецца чытанне курсаў, звязаных з мадэлямі рашэння задач у інтэлектуальных сістэмах.

\newpage
\begin{center}
\bf SUMMARY\\[1mm]\rm Shunkevich Daniil Viacheslavovich\\[1mm] \bf
AGENT-ORIENTED PROBLEM SOLVERS \\OF INTELLIGENT SYSTEMS
\end{center}

{\bf Key words:} problem solver, knowledge base, intelligent system, semantic model, semantic memory, multi-agent system.

\textbf{The purpose of research } is to develop a complex of models, methods and tools for constructing and modifying the problem solvers of intelligent systems.

\textbf{The obtained results and their novelty}:
in work the existing approaches to the construction of problem solvers of intelligent systems are investigated. An agent-oriented model of the hybrid solver was developed, which considers each such solver as a team of agents working over the common semantic memory, which allows implementing and integrating of various problem solving models within a hybrid solver, and also ensuring the modifiability of such solvers. A formal model of the interaction of parallel asynchronous information processes in the common semantic memory was developed, including the means of their synchronization based on the mechanism of blocking the elements of semantic memory, as well as the interaction mechanism of agents performing these processes, providing the possibility of specifying planned locks, identifying and eliminating deadlocks. A method for hybrid problem solvers constructing and modifying was developed, based on the formal ontology of the activity of such solvers developers and oriented to the use of the library of problem solvers reusable components. The means of automation and information support for the process of hybrid problem solvers constructing and modifying were developed, including a library of reusable components of such solvers and means for automating the process of knowledge-processing agents constructing. These tools are developed using the proposed solver model.

\textbf{Recommendations on the use and field of application:}
the developed models, tools and their software implementation can be used in the development of problem solvers for intelligent systems for various purposes. The results obtained in the thesis can be used in the educational process of higher education institutions for training specialists in specialties, which provides the reading of courses related to problem solving models in intelligent systems.

\newpage
\thispagestyle{empty}

\vspace* {0.5cm}

\begin{center}
\textit{Научное издание}

\vspace*{\fill}

\textbf{Шункевич} Даниил Вячеславович\\[25mm]
 {\large\bf АГЕНТНО-ОРИЕНТИРОВАННЫЕ РЕШАТЕЛИ ЗАДАЧ ИНТЕЛЛЕКТУАЛЬНЫХ СИСТЕМ\\}
 
\vspace* {2cm}

 
АВТОРЕФЕРАТ
 
диссертации на соискание ученой степени кандидата технических наук

\vspace* {1cm}

по специальности 05.13.17 — Теоретические основы информатики

\end{center}

\vspace*{\fill}\vspace*{\fill} \vspace*{\fill} \vspace*{\fill}
\begin{tabbing}
XXXXXXXXXXXXXXXXXXXXXXXXXXXXXX\= \kill
\end{tabbing}

\begin{center}

{\small \mbox{Подписано в печать ~~~~.06.2018 г. Формат $60\times84\ \  1/16$. Бумага офсетная. Гарнитура «Таймс».}\\
Отпечатано на ризографе. Усл. печ. 1,63. Уч.-изд. л. 1,5. Тираж 60 экз. Заказ ~~~~.

\bigskip
%Учетн. изд. л. 1.2. Тираж 60 экз. ~ Заказ № \hspace{8mm}.\\

Издатель и полиграфическое исполнение: учреждение образования \\
«Белорусский государственный университет информатики и радиоэлектроники». \\
Свидетельство о государственной регистрации издателя, изготовителя, распространителя печатных изданий № 1/238 от 24.03.2014,\\ 
№ 2/113 от 07.04.2014, № 3/615 от 07.04.2014. \\
ЛП № 02330/264 от 14.04.2014.\\
220013, Минск, П. Бровки, 6}
\end{center}
\end{document}
