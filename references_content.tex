%\section*{Статьи в рецензируемых научных журналах}
\ifx\isabstract\undefined 
\section* {Статьи в рецензируемых научных журналах}
\fi

\bibitem{1-A}
\selectlanguageifdefined{english}
Golovko, V. Learning Technique for Deep Belief Neural Networks~/ V. Golovko, A. Kroshchanka, U. Rubanau, S. Jankowski~//
  Springer. ---
\newblock 2014. ---
\newblock Vol.~440. Communication in Computer and Information Science ---
\newblock P.~136--146.

\bibitem{2-A}
\selectlanguageifdefined{russian}
Головко, В. А. Персептроны и нейронные сети глубокого доверия: обучение и применение~/ В. А. Головко, А. А. Крощенко~//
  Вестник Брестского государственного технического университета. ---
\newblock Брест,~2014. ---
\newblock Т.~5. ---
\newblock {\cyr\CYRS.}~2--12.

\bibitem{3-A}
\selectlanguageifdefined{russian}
Головко, В. А. Метод обучения нейронной сети глубокого доверия и применение для визуализации данных~/ В. А. Головко, А. А. Крощенко~//
  Комп'ютерно-інтегровані технології: освіта, наука, виробництво. ---
\newblock Луцк,~2015. ---
\newblock Вып.~19. ---
\newblock {\cyr\CYRS.}~6--12.

\bibitem{4-A}
\selectlanguageifdefined{english}
Golovko, V. The nature of unsupervised learning in deep neural networks: A new understanding and novel approach~/ V. Golovko, A. Kroshchanka, D. Treadwell~//
  Optical Memory and Neural Networks. ---
\newblock Pleiades Publishing, 2016. ---
\newblock P.~127--141.

\bibitem{5-A}
\selectlanguageifdefined{russian}
Головко, В. А. Теория глубокого обучения: конвенциальный и новый подход~/ В. А. Головко, А. А. Крощенко, М. В. Хацкевич~//
  Вестник Брестского государственного технического университета. ---
\newblock Брест, 2016. ---
\newblock №~5. ---
\newblock {\cyr\CYRS.}~7--16.

\bibitem{6-A}
\selectlanguageifdefined{russian}
Головко, В. А. Интеграция искусственных нейронных сетей с базами знаний~/ В. А. Головко, В. В. Голенков, В. П. Ивашенко, В. В. Таберко, Д. С. Шаток, А. А. Крощенко, М. В. Ковалёв ~//
  Онтология проектирования. ---
\newblock EBSCO Publishing, 2018. ---
\newblock Т.~8. ---
\newblock №~3(29). ---
\newblock {\cyr\CYRS.}~366--386.

\bibitem{7-A}
\selectlanguageifdefined{russian}
Крощенко, А. А. Реализация нейросетевой системы распознавания маркировки продукции~/ А. А. Крощенко, В. А. Головко~//
  Вестник Брестского государственного технического университета. ---
\newblock Брест, 2019. ---
\newblock №~5. ---
\newblock {\cyr\CYRS.}~9--12.

\bibitem{8-A}
\selectlanguageifdefined{english}
Golovko, V. Brands and caps labeling recognition in images using deep learning~/ V. Golovko, A. Kroshchanka, E. Mikhno~//
  International Conference on Pattern Recognition and Information Processing. ---
\newblock Springer, Cham, 2019. ---
\newblock P.~35--51.

\bibitem{9-A}
\selectlanguageifdefined{english}
Golovko, V. Deep convolutional neural network for detection of solar panels~/ V. Golovko, A. Kroshchanka, E. Mikhno, M. Komar, A. Sachenko~//
  Data-Centric Business and Applications. ---
\newblock Springer, Cham, 2021. ---
\newblock P.~371--389.

\bibitem{10-A}
\selectlanguageifdefined{english}
Golovko,  V. A.  Deep Neural Networks: Selected Aspects of Learning and Application~/ V. A. Golovko, A. A. Kroshchanka, E. V. Mikhno~//
  Pattern Recognition and Image Analysis. ---
\newblock Pleiades Publishing, 2021. ---
\newblock Vol.~31. ---
\newblock №~1. ---
\newblock P.~132--143.

\bibitem{11-A}
\selectlanguageifdefined{english}
Kroshchanka, A. A. Method for Reducing Neural-Network Models of Computer Vision~/ A. A. Kroshchanka, V. A. Golovko, M. Chodyka~//
  Pattern Recognition and Image Analysis. ---
\newblock Pleiades Publishing, 2022. ---
\newblock Vol.~32. ---
\newblock №~2. ---
\newblock P.~294--300.

\ifx\isabstract\undefined 
\begin{center}
\vspace{3mm}
\newpage
{\bf Статьи в сборниках материалов научных конференций, включенных в системы международного цитирования\\ (Scopus, Web of Science и IEEE Xplore digital)}
\vspace{3mm}
\end{center}
\else
\vspace{2mm}
{\bf Статьи в сборниках материалов научных конференций, включенных в системы международного цитирования\\ (Scopus, Web of Science и IEEE Xplore digital)}
\vspace{2mm}
\fi

\bibitem{12-A}
\selectlanguageifdefined{english}
Golovko, V. A New Technique for Restricted Boltzmann Machine Learning~/ A. Kroshchanka, V. Turchenko, S. Jankowski, D. Treadwell~//
  Proceedings of the 8th IEEE International Conference IDAACS-2015, Warsaw 24-26 September 2015. ---
\newblock Warsaw, 2015. ---
\newblock P.~182--186.

\bibitem{13-A}
\selectlanguageifdefined{english}
Golovko, V. Theoretical Notes on Unsupervised Learning in Deep Neural Networks~/ V. Golovko, A. Kroshchanka~//
  Proceedings of the 8th International Joint Conference on Computational Intelligence (IJCCI 2016). ---
\newblock SCITEPRESS, 2016. ---
\newblock P.~91--96.

\bibitem{14-A}
\selectlanguageifdefined{russian}
Golovko, V. Convolutional neural network based solar photovoltaic panel detection in satellite photos~/ V. Golovko, S. Bezobrazov, A. Kroshchanka, A. Sachenko, M. Komar, A. Karachka~//
  9th IEEE International Conference on Intelligent Data Acquisition and Advanced Computing Systems: Technology and Applications (IDAACS). ---
\newblock IEEE, 2017. ---
\newblock Vol.~1. ---
\newblock P.~14--19.

\bibitem{15-A}
\selectlanguageifdefined{russian}
Golovko, V. Development of solar panels detector~/ V. Golovko, A. Kroshchanka, S. Bezobrazov, A. Sachenko, M. Komar, O. Novosad~//
  International Scientific-Practical Conference Problems of Infocommunications. Science and Technology (PIC S\&T). ---
\newblock IEEE, 2018. ---
\newblock P.~761--764.

\bibitem{16-A}
\selectlanguageifdefined{russian}
Kroshchanka, A. The Reduction of Fully Connected Neural Network Parameters Using the Pre-training Technique~/ A. Kroshchanka, V. Golovko~//
  11th IEEE International Conference on Intelligent Data Acquisition and Advanced Computing Systems: Technology and Applications (IDAACS). ---
\newblock IEEE, 2021. ---
\newblock Vol.~2. ---
\newblock P.~937--941.

%\section*{Статьи в сборниках материалов научных конференций}
\ifx\isabstract\undefined 
\begin{center}
\vspace{3mm}
{\bf Статьи в сборниках материалов научных конференций}
\vspace{3mm}
\end{center}
\else
\vspace{2mm}
{\bf Статьи в сборниках материалов научных конференций}
\vspace{2mm}
\fi

\bibitem{17-A}
\selectlanguageifdefined{russian}
Крощенко, А. А. Методы глубокого обучения нейронных сетей~/ А.А. Крощенко~// 
 Материалы конференции <<Вычислительные методы, модели и образовательные технологии>>. ---
\newblock Брест : БрГУ, 2013. ---
\newblock {\cyr\CYRS.}~21--22.

\bibitem{18-A}
\selectlanguageifdefined{russian}
Головко, В. А. Об одном методе обучения нейронных сетей глубокого доверия~/ В. А. Головко, А. А. Крощенко~// 
 Вычислительные методы, модели и образовательные технологии : сборник материалов международной научно-практической конференции. ---
\newblock Брест : БрГУ, 2014. ---
\newblock {\cyr\CYRS.}~98--99.

\bibitem{19-A}
\selectlanguageifdefined{russian}
Крощенко, А. А. Применение нейронных сетей глубокого доверия в интеллектуальном анализе данных~/ А. А. Крощенко~// 
 сборник материалов IX Республиканской научной конференции молодых ученых и студентов <<Современные проблемы математики и вычислительной техники>>. ---
\newblock Брест : БрГТУ, 2015. ---
\newblock {\cyr\CYRS.}~12--14.

\bibitem{20-A}
\selectlanguageifdefined{russian}
Головко, В. А.  Применение нейронных сетей глубокого доверия в интеллектуальном анализе данных~/ В. А. Головко, А. А. Крощенко~// 
 Вычислительные методы, модели и образовательные технологии : сборник материалов международной научно-практической конференции. ---
\newblock Брест : БрГУ, 2015.---
\newblock {\cyr\CYRS.}~97--98.

\bibitem{21-A}
\selectlanguageifdefined{russian}
Крощенко, А. А. Применение глубокой нейронной сети для решения задачи распознавания образов~/ А. А. Крощенко~// 
 Вычислительные методы, модели и образовательные технологии : сборник материалов международной научно-практической конференции. ---
\newblock Брест : БрГУ, 2016.---
\newblock {\cyr\CYRS.}~132--133.

\bibitem{22-A}
\selectlanguageifdefined{english}
Golovko, V. A. Integration of artificial neural networks and knowledge bases~/ V. A. Golovko, A. A. Kroshchanka, V. V. Golenkov, V. P. Ivashenko, M. V. Kovalev, V. V. Taberko, D. S. Ivaniuk~// 
 Open Semantic Technologies for Intelligent Systems (OSTIS-2018). ---
\newblock Minsk : BSUIR, 2018. ---
\newblock P.~133--146.

\bibitem{23-A}
\selectlanguageifdefined{english}
Golovko, V. Principles of decision-making systems building based on the integration of neural networks and semantic models~/ V. Golovko, A. Kroshchanka, V. Ivashenko, M. Kovalev, V. Taberko, D. Ivaniuk~//
 Open Semantic Technologies for Intelligent Systems (OSTIS-2019). ---
\newblock Minsk : BSUIR, 2019. ---
\newblock P.~91--102.

\bibitem{24-A}
\selectlanguageifdefined{russian}
Головко, В. А. Обнаружение и распознавание маркировки продукции с помощью нейросетевых алгоритмов~/ В. А. Головко, А. А. Крощенко~//
 Вычислительные методы, модели и образовательные технологии : сборник материалов VII международной научно-практической конференции. ---
\newblock Брест : БрГУ, 2019. ---
\newblock {\cyr\CYRS.}~3--6. 

\bibitem{25-A}
\selectlanguageifdefined{english}
Golovko, V. Implementation of an intelligent decision support system to accompany the manufacturing process~/ V. Golovko, A. Kroshchanka, M. Kovalev, V. Taberko, D. Ivaniuk~// 
 Open Semantic Technologies for Intelligent Systems (OSTIS-2020). ---
\newblock Minsk : BSUIR, 2020. ---
\newblock P.~175--182.

\bibitem{26-A}
\selectlanguageifdefined{english}
Golovko, V. Neuro-Symbolic Artificial Intelligence: Application for Control the Quality of Product Labeling~/ V. Golovko, A. Kroshchanka, M. Kovalev, V. Taberko, D. Ivaniuk~// 
 International Conference on Open Semantic Technologies for Intelligent Systems. ---
\newblock Springer, Cham, 2020.---
\newblock P.~81--101.

\bibitem{27-A}
\selectlanguageifdefined{english}
Kroshchanka, A. Neural network component of the product marking recognition system on the production line~/ A. Kroshchanka, D. Ivaniuk~// 
 Open Semantic Technologies for Intelligent Systems (OSTIS-2021). ---
\newblock Minsk : BSUIR, 2021. -
\newblock P.~219--224.

% \bibitem{28-A}
% \selectlanguageifdefined{english}
% Kroshchanka, A. Semantic analysis of the video stream based on neuro-symbolic artificial intelligence~/ A. Kroshchanka, E. Mikhno, M. Kovalev, V. Zahariev, A. Zagorskij~// 
%  Open Semantic Technologies for Intelligent Systems (OSTIS-2021). ---
% \newblock Minsk : BSUIR, 2021. ---
% \newblock №~5. ---
% \newblock P.~193--204.

% \bibitem{29-A}
% \selectlanguageifdefined{russian}
% Головко, В. А. Гибридная интеллектуальная система оценки эмоционального состояния пользователя~/ В. А. Головко, А. А. Крощенко~// 
%  Современные проблемы математики и вычислительной техники : сборник материалов ХII Республиканской научной конференции молодых ученых и студентов, Брест, 18–19 ноября 2021 г.~/ Министерство образования Республики Беларусь, Брестский государственный технический университет ; редкол.: В. А. Головко (гл. ред.) [и др.]. ---
% \newblock Брест : БрГТУ, 2021. ---
% \newblock {\cyr\CYRS.}~10--13.

\bibitem{30-A}
\selectlanguageifdefined{english}
Kroshchanka, A. Reduction of neural network models in intelligent computer systems of a new generation~/ A. Kroshchanka~// 
 Open Semantic Technologies for Intelligent Systems (OSTIS-2023). ---
\newblock Minsk : BSUIR, 2023. -
\newblock P.~127--132.

% \bibitem{24-A}
% \selectlanguageifdefined{russian}
% Формальное семантическое описание целенаправленной деятельности различного вида субъектов~/ Д. В. Шункевич,  А. В. Губаревич, М.~Н.~Святкина, О. Л. Моросин~// 
%  Открытые семантические технологии проектирования интеллектуальных систем (OSTIS-2016) : материалы VI Междунар. науч.-техн. конф., Минск, 18–20 февр. 2016 г.~/ Белорус. гос. ун-т информатики и радиоэлектроники ; редкол.: В. В. Голенков (отв. ред.) [и др.]. ---
% \newblock Минск, 2016. ---
% \newblock {\cyr\CYRS.}~125--136.

% \bibitem{25-A}
% \selectlanguageifdefined{russian}
% Шункевич, Д. В. Взаимодействие асинхронных параллельных процессов обработки знаний в общей семантической памяти~/ Д. В. Шункевич~// 
%  Открытые семантические технологии проектирования интеллектуальных систем (OSTIS-2016) : материалы VI Междунар. науч.-техн. конф., Минск, 18–20 февр. 2016 г.~/ Белорус. гос. ун-т информатики и радиоэлектроники ; редкол.: В. В. Голенков (отв. ред.) [и др.]. ---
% \newblock Минск, 2016. ---
% \newblock {\cyr\CYRS.}~137--144.

% \bibitem{26-A}
% \selectlanguageifdefined{english}
% Shunkevich, D. Ontology-based design of knowledge processing machines~/ D. Shunkevich~// 
%   Открытые семантические технологии проектирования интеллектуальных систем : материалы междунар. науч.-техн. конф., Минск, 16–18 февр. 2017 г.~/ Белорус. гос. ун-т информатики и радиоэлектроники ; редкол.: В. В. Голенков (отв. ред.) [и др.]. ---
% \newblock Минск, 2017. --- Вып. 1. ---
% \newblock P.~73--94.

% \bibitem{27-A}
% \selectlanguageifdefined{english}
% Ontology-based design of batch manufacturing enterprises~/ V.~Golenkov, K.~Rusetski, D.~Shunkevich, I.~Davydenko, V.~Zakharov, V.~Ivashenko, D.~Koronchik, V.~Taberko, D.~Ivanyuk~// 
%   Открытые семантические технологии проектирования интеллектуальных систем : материалы междунар. науч.-техн. конф., Минск, 16–18 февр. 2017 г.~/ Белорус. гос. ун-т информатики и радиоэлектроники ; редкол.: В. В. Голенков (отв. ред.) [и др.]. ---
% \newblock Минск, 2017. --- Вып.~1. ---
% \newblock С.~265--280.

% \bibitem{28-A}
% \selectlanguageifdefined{russian}
% Семантическая модель представления и обработки баз знаний~/ В.~В.~Голенков, Н. А. Гулякина, И. Т. Давыденко, Д. В. Шункевич~// 
%  Аналитика и управление данными в областях с интенсивным использованием данных : сб. науч. тр. XIX Междунар. конф. DAMDID/RCDL’2017, Москва, 10–13 окт. 2017 г. / Федер. исслед. центр <<Информатика и управление>> Рос. акад. наук ; под ред. Л. А. Калиниченко [и др.]. ---
% \newblock М., 2017. ---
% \newblock {\cyr\CYRS.}~412--419.

% %\section*{Статьи в сборниках материалов научных конференций}
% \ifx\isabstract\undefined 
% \begin{center}
% \vspace{3mm}
% {\bf Тезисы докладов в сборниках материалов научных конференций}
% \vspace{3mm}
% \end{center}
% \else
% \vspace{2mm}
% %\newpage
% {\bf Тезисы докладов в сборниках материалов научных конференций}
% \vspace{2mm}
% \fi

% \bibitem{29-A}
% \selectlanguageifdefined{russian}
% Интеллектуальный решатель задач по геометрии~/ Д. В. Шункевич, С. С.~Заливако, О. Ю.~Савельева, С. С.~Старцев~// 
%   Информационные системы и технологии IST’2010 : материалы VI Междунар. конф., Минск, 24–25 нояб. 2010 г. / Белорус. гос. ун-т [и др.] ; редкол.: А. Н. Курбацкий (отв. ред.) [и др.]. ---
% \newblock Минск, 2010. ---
% \newblock {\cyr\CYRS.}~482--485.

% \bibitem{30-A}
% \selectlanguageifdefined{russian}
% Шункевич, Д. В.  Принципы проектирования и интерпретации программ, ориентированных на обработку семантических сетей~/ Д. В. Шункевич~// 
%  Информационные технологии и системы-2013 (ИТС 2013) : материалы междунар. науч. конф., Минск, 23 окт. 2013 г.~/ Белорус. гос. ун-т информатики и радиоэлектроники ; редкол.: \mbox{Л. Ю.} Шилин (гл. ред) [и др.]. ---
% \newblock Минск, 2013. ---
% \newblock {\cyr\CYRS.}~116--117.

% \bibitem{31-A}
% \selectlanguageifdefined{russian}
% Шункевич, Д. В.  Базовые понятия технологии компонентного проектирования машин обработки знаний систем дистанционного обучения~/ \mbox{Д. В. Шункевич}~// 
%   Дистанционное обучение -- образовательная среда XXI века : материалы VIII междунар. науч.-метод. конф., Минск, 5–6 дек. 2013г. / Белорус. гос. ун-т информатики и радиоэлектроники ; редкол.: Б. В. Никульшин [и др.]. ---
% \newblock Минск, 2013. ---
% \newblock {\cyr\CYRS.}~186--187.

% \bibitem{32-A}
% \selectlanguageifdefined{russian}
% Якимчик, С. В.  Принципы построения решателей задач в прикладных интеллектуальных системах~/ С. В. Якимчик, Д. В. Шункевич~// 
%  Информационные технологии и системы-2014 (ИТС 2014) : материалы междунар. науч. конф., Минск, 29 окт. 2014 г.~/ Белорус. гос. ун-т информатики и радиоэлектроники ; редкол.: \mbox{Л. Ю.} Шилин (гл. ред.) [и др.]. ---
% \newblock Минск, 2014. ---
% \newblock {\cyr\CYRS.}~160--161.

% \bibitem{33-A}
% \selectlanguageifdefined{russian}
% Губаревич, А. В.  Онтология деятельности интеллектуальных агентов над общей памятью~/ А. В. Губаревич, М. Н. Святкина Д. В. Шункевич~// 
%  Информационные технологии и системы-2015 (ИТС 2015) : материалы междунар. науч. конф., Минск, 28 окт. 2015 г.~/ Белорус. гос. ун-т информатики и радиоэлектроники ; редкол.: \mbox{Л. Ю.} Шилин (гл. ред.) [и др.]. ---
% \newblock Минск, 2015. ---
% \newblock {\cyr\CYRS.}~138--139.

% \bibitem{34-A}
% \selectlanguageifdefined{russian}
% Онтологическое моделирование для реализации семантических технологий создания интеллектуальной системы управления жизненным циклом~/ А. В. Федотова, И. Т. Давыденко, М. Н. Святкина, Д. В. Шункевич~// 
%     Информационная безопасность регионов России (ИБРР–2015) : IX С.-Петерб. межрегион. конф., Санкт-Петербург, 28–30 окт. 2015 г. : материалы конф. / \mbox{С.-Петерб.} О-во информатики, вычисл. техники, систем связи и упр ; редкол.: Б. Я. Советов [и др.].  ---
% \newblock СПб., 2015. ---
% \newblock {\cyr\CYRS.}~336.

% \bibitem{35-A}
% \selectlanguageifdefined{russian}
% Шункевич, Д. В.  Унифицированная семантическая модель процесса проектирования машин обработки знаний~/ Д. В. Шункевич, И. Б. Фоминых, О. Л. Моросин~// 
%  Информационные технологии и системы-2016 (ИТС 2016) : материалы междунар. науч. конф., Минск, 26 окт. 2016 г.~/ Белорус. гос. ун-т информатики и радиоэлектроники ; редкол.: \mbox{Л. Ю.} Шилин (гл. ред.) [и др.]. ---
% \newblock Минск, 2016. ---
% \newblock {\cyr\CYRS.}~172--173.

% \bibitem{36-A}
% \selectlanguageifdefined{russian}
% Шункевич, Д. В.  Представление базовых конструкций языка семантических сетей с теоретико-множественной интерпретацией средствами платформы Neo4j~/ Д. В. Шункевич, О. С. Родионова~// 
%  Информационные технологии и системы-2016 (ИТС 2016) : материалы междунар. науч. конф., Минск, 26 окт. 2016 г.~/ Белорус. гос. ун-т информатики и радиоэлектроники ; редкол.: \mbox{Л. Ю.} Шилин (гл. ред.) [и др.]. ---
% \newblock Минск, 2016. ---
% \newblock {\cyr\CYRS.}~174--175.

% \bibitem{37-A}
% \selectlanguageifdefined{russian}
% Голенков, В. В. Принципы построения машин обработки баз знаний~/ В. В. Голенков, Д. В. Шункевич~// 
%  Информационные технологии и системы-2017 (ИТС 2017) : материалы междунар. науч. конф., Минск, 25 окт. 2017 г.~/ Белорус. гос. ун-т информатики и радиоэлектроники ; редкол.: \mbox{Л. Ю.} Шилин (гл. ред.) [и др.]. ---
% \newblock Минск, 2017. ---
% \newblock {\cyr\CYRS.}~134--135.