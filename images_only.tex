\documentclass{thesisby}

%\usepackage[cp1251]{inputenc}
\usepackage[utf8]{inputenc}
\usepackage[T2A]{fontenc}
\usepackage[russian]{babel}

%\usepackage{pscyr}
%\renewcommand{\rmdefault}{ftm}

%для борьбы с переполнениями за счет разреженных слов в абзаце
\emergencystretch=25pt
%math
\usepackage{mathtools}
\usepackage{tabularx}
\usepackage{pdflscape}

\usepackage{afterpage}
\usepackage{geometry}
\usepackage{multirow}
\usepackage{enumitem}

\usepackage{amsmath,amssymb,amsfonts}
\usepackage{longtable,array}
\usepackage{graphicx,epsfig}
\usepackage[unicode,colorlinks=false,pagebackref=false, hidelinks]{hyperref}
\urlstyle{same}
%\usepackage{refcheck}% checks lost and useless labels, shows `keys' of \label in the margins
%фиксить картинки
%\usepackage{float}

\usepackage{textcomp}% For Celsium sign only

% Листинги с исходным кодом программ
\usepackage{fancyvrb}
\usepackage{listings}
\usepackage{totcount}
%\usepackage{totpages}
\usepackage{cite}

% Плавающие окружения. во многом лучше пакета float
\usepackage{floatrow}

\usepackage{tikz}
\newcolumntype{P}[1]{>{\centering\arraybackslash}p{#1}}
\newcolumntype{M}[1]{>{\centering\arraybackslash}m{#1}}

%for lists
\usepackage[ampersand]{easylist}
\ListProperties(Hide=100, Hang=false, Margin=0mm, Indent1=10.5mm, Indent2=15mm, Style*=-- ,
Style2*=$\bullet$ ,Style3*=$\circ$ ,Style4*=\tiny$\blacksquare$ )

\usepackage[ruled, vlined]{algorithm2e}
\makeatletter
\newenvironment{algo}[1][]
  {\renewcommand{\algorithmcfname}{Алгоритм}%
   \begin{algorithm}[#1]
   \long\def\@caption##1[##2]##3{%
     \par
     \begingroup\@parboxrestore
     \if@minipage\@setminipage\fi
     \normalsize \@makecaption{\AlCapSty{\AlCapFnt\algorithmcfname}}{\ignorespaces ##3}%
     \par\endgroup
   }}
  {\end{algorithm}}
\makeatother

% \SetKwData{KwDat}{Данные}
\SetKwInput{KwIn}{Исходные данные}
\SetKwInput{KwRes}{Результат}%
% \SetKwIF{Si}{SinonSi}{Sinon}{si}{alors}{sinon si}{sinon}{fin si}%
% \SetKwFor{Tq}{tant que}{faire}{fin tq}%

\newenvironment{easylistNum}{\begin{easylist}
\ListProperties(Hide1=0, Hang=false, Margin=0mm, Indent1=10.5mm, Indent2=15mm, Start1=1, Style*=, FinalMark={)})}{\ListProperties(Hide=100, Hang=false, Margin=0mm, Indent1=10.5mm, Indent2=15mm, Style*=-- ,
Style2*=$\bullet$ ,Style3*=$\circ$ ,Style4*=\tiny$\blacksquare$ )\end{easylist}}

\renewcommand\labelitemi{\textbf{--}}
\floatsetup[table]{capposition=top}

    % \lstset{
    %     language=Python,
    %     basicstyle=\ttfamily,
    %     keywordstyle=\bfseries,
    %     showspaces=false,
    %     showstringspaces=false,
    %     mathescape=true,
    %     aboveskip=0pt,
    %     belowskip=6pt
    % }
\lstdefinestyle{PythonStyle}{
  keywordstyle=\bf,
  belowcaptionskip=1\baselineskip,
  breaklines=true,
  language=Python,
  showstringspaces=false,
  basicstyle=\small\ttfamily,
  commentstyle=\itshape
}
% \lstdefinestyle{PythonStyle}{
% basicstyle=\footnotesize\ttfamily,
% language=Python,
% keywordstyle=\bfseries,
% showstringspaces=false,
% commentstyle={},
% texcl=true
% }

\graphicspath{{../}}

%\renewcommand{\cftchapleader}{\cftdotfill{\cftdotsep}}


\begin{document}

\begin{figure}[H]
  \center
  \includegraphics[width=\linewidth]{man-source/images/ch1/pic1_1.pdf}
  \caption{Упрощенная схема системы комплексной автоматизации производства} 
    \label{fig:pic1_1}
\end{figure}


\begin{figure}[H]
  \center
  \includegraphics[width=0.65\linewidth]{man-source/images/ch1/pic1_2.pdf}
  \caption{Упрощенная схема подсистемы обработки нештатных ситуаций} 
    \label{fig:pic1_2}
\end{figure}


\begin{figure}[H]
  \center
  \includegraphics[width=0.9\linewidth]{man-source/images/ch1/pic1_3.png}
  \caption{Архитектура ostis-системы} 
    \label{fig:pic1_3}
\end{figure}


\begin{figure}[H]
  \centering
  \includegraphics[width=\textwidth]{man-source/images/ch1/pic1_4.pdf}
  \caption{Структура диссертационной работы}
  \label{fig:pic1_4}
\end{figure}


\begin{figure}[H]
  \centering
  \includegraphics[width=0.8\textwidth]{man-source/images/ch1/pic1_5.png}
  \caption{Взаимосвязь предметных областей}
  \label{fig:pic1_5}
\end{figure}


\begin{figure}[H]
  \centering
  \includegraphics[width=\textwidth]{man-source/images/ch2/pic_kpp.jpg}
  \caption{Организация обработки информации в ostis-системе}
  \label{fig:pic_kpp}
\end{figure}


\begin{figure}[H]
  \centering
  \includegraphics[width=\textwidth]{man-source/images/ch2/pic2_1.pdf}
  \caption{Структура главы 2}
  \label{fig:pic2_1}
\end{figure}


\begin{figure}[H]
  \centering
  \includegraphics[width=0.7\textwidth]{man-source/images/ch2/pic_proc_model.png}
  \caption{Модель взаимодействия процессов в семантической памяти}
  \label{fig:pic_proc_model}
\end{figure}


\begin{figure}[H]
  \centering
  \includegraphics[width=0.8\textwidth]{man-source/images/ch2/pic_laa.png}
  \caption{Декомпозиция логически атомарного действия на поддействия}
  \label{fig:pic_laa}
\end{figure}


\begin{figure}[H]
  \centering
  \includegraphics[width=0.85\textwidth]{man-source/images/ch2/pic_probl_ex.png}
  \caption{Пример представления задачи}
  \label{fig:pic_probl_ex}
\end{figure}


\begin{figure}[H]
    \centering
    \includegraphics[width=\textwidth]{man-source/images/ch2/pic2_2.png}
    \caption{Пример описания блокировок в семантической памяти}
    \label{fig:pic2_2}
\end{figure}


\begin{figure}[H]
    \centering
    \includegraphics[width=\textwidth]{man-source/images/ch2/pic2_3.png}
    \caption{Блокировки пользователя}
    \label{fig:pic2_3}
\end{figure}


\begin{figure}[H]
    \centering
    \includegraphics[width=\textwidth]{man-source/images/ch2/pic_ips.png}
    \caption{Детализированная схема обработки информации в ostis-системе}
    \label{fig:pic_ips}
\end{figure}


\begin{figure}[H]
    \centering
    \includegraphics[width=\textwidth]{man-source/images/ch2/pic_kpm_hier.png}
    \caption{Иерархическая агентно-ориентированная модель решателя}
    \label{fig:pic2_4}
\end{figure}


\begin{figure}[H]
    \centering
    \includegraphics{man-source/images/ch2/pic2_5.png}
    \caption{Спецификация действия интерпретации scp-программы}
    \label{fig:pic2_5}
\end{figure}


\begin{figure}[H]
    \centering
    \includegraphics[width=\textwidth]{man-source/images/ch3/pic3_1.pdf}
    \caption{Структура главы 3}
    \label{fig:pic3_1}
\end{figure}


\begin{figure}[H]
    \centering
    \includegraphics[width=\textwidth]{man-source/images/ch3/pic3_2.png}
    \caption{Этапы процесса построения и модификации гибридных решателей задач}
    \label{fig:pic3_2}
\end{figure}


\begin{figure}[H]
    \centering
    \includegraphics[width=\textwidth]{man-source/images/ch3/pic3_3.pdf}
    \caption{Структура системы автоматизации процесса построения и модификации решателей задач}
    \label{fig:pic3_3}
\end{figure}


\begin{figure}[H]
    \centering
    \includegraphics{man-source/images/ch4/pic4_1.png}
    \caption{Добавление точки останова}
    \label{fig:pic4_1}
\end{figure}


\begin{figure}[H]
    \centering
    \includegraphics{man-source/images/ch4/pic4_2.png}
    \caption{Поиск остановленных операторов}
    \label{fig:pic4_2}
\end{figure}


\begin{figure}[H]
    \centering
    \includegraphics{man-source/images/ch4/pic4_3.png}
    \caption{Оператор, на котором остановилось выполнение отлаживаемого процесса}
    \label{fig:pic4_3}
\end{figure}


\begin{figure}[H]
    \centering
    \includegraphics{man-source/images/ch4/pic4_4.png}
    \caption{Запрос семантической окрестности оператора}
    \label{fig:pic4_4}
\end{figure}


\begin{figure}[H]
    \centering
    \includegraphics{man-source/images/ch4/pic4_5.png}
    \caption{Семантическая окрестность оператора}
    \label{fig:pic4_5}
\end{figure}


\begin{figure}[H]
    \centering
    \includegraphics[width=0.9\textwidth]{man-source/images/ch4/pic_geom_solver_dots.png}
    \caption{Структура решателя задач по геометрии}
    \label{fig:pic_geom_solver}
\end{figure}


\begin{figure}[H]
    \centering
    \includegraphics{man-source/images/ch4/pic4_6.png}
    \caption{Результат работы агента поиска непосредственных связей между объектами}
    \label{fig:pic4_6}
\end{figure}


\begin{figure}[H]
    \centering
    \includegraphics{man-source/images/ch4/pic4_7.png}
    \caption{Результат работы агента поиска понятий, через которые определяется заданное понятие}
    \label{fig:pic4_7}
\end{figure}


\begin{figure}[H]
    \centering
    \includegraphics{man-source/images/ch4/pic4_8.png}
    \caption{Результат работы агента поиска понятий, которые определяются на основе заданного}
    \label{fig:pic4_8}
\end{figure}


\begin{figure}[H]
    \centering
    \includegraphics{man-source/images/ch4/pic4_9.png}
    \caption{Образец поиска}
    \label{fig:pic4_9}
\end{figure}


\begin{figure}[H]
    \centering
    \includegraphics{man-source/images/ch4/pic4_10.png}
    \caption{Результат работы агента поиска всех конструкций, изоморфных заданному образцу}
    \label{fig:pic4_10}
\end{figure}


\begin{figure}[H]
    \centering
    \includegraphics{man-source/images/ch4/pic4_11.png}
    \caption{Результат работы агента поиска sc-текста доказательства для заданного утверждения}
    \label{fig:pic4_11}
\end{figure}


\begin{figure}[H]
    \centering
    \includegraphics{man-source/images/ch4/pic4_12.png}
    \caption{Результат работы агента поиска отношений, заданных на понятии}
    \label{fig:pic4_12}
\end{figure}


\begin{figure}[H]
    \centering
    \includegraphics{man-source/images/ch4/pic4_13.png}
    \caption{Результат работы агента поиска утверждений об объекте}
    \label{fig:pic4_13}
\end{figure}


\begin{figure}[H]
    \centering
    \includegraphics{images/ch4/geom_cond.png}
    \caption{Условие геометрической задачи}
    \label{fig:pic_geom_cond}
\end{figure}


\begin{figure}[H]
    \centering
    \includegraphics[width=\textwidth]{images/ch4/geom_cond_1.png}
    \caption{Формальная запись условия геометрической задачи (часть 1)}
    \label{fig:pic_geom_cond_1}
\end{figure}


\begin{figure}[H]
    \centering
    \includegraphics[width=\textwidth]{images/ch4/geom_cond_2.png}
    \caption{Формальная запись условия геометрической задачи (часть 2)}
    \label{fig:pic_geom_cond_2}
\end{figure}


\begin{figure}[H]
    \centering
    \includegraphics{images/ch4/geom_step1.png}
    \caption{Результат шага 2}
    \label{fig:pic_geom_step_1}
\end{figure}


\begin{figure}[H]
    \centering
    \includegraphics{images/ch4/geom_step2.png}
    \caption{Результат шага 3}
    \label{fig:pic_geom_step_2}
\end{figure}


\begin{figure}[H]
    \centering
    \includegraphics{images/ch4/geom_step3.png}
    \caption{Результат шага 4}
    \label{fig:pic_geom_step_3}
\end{figure}


\begin{figure}[H]
    \centering
    \includegraphics{images/ch4/geom_step4.png}
    \caption{Результат шага 5}
    \label{fig:pic_geom_step_4}
\end{figure}


\begin{figure}[H]
    \centering
    \includegraphics{images/ch4/geom_step5.png}
    \caption{Результат шага 6}
    \label{fig:pic_geom_step_5}
\end{figure}


\begin{figure}[H]
    \centering
    \includegraphics[scale=0.75]{man-source/images/ch4/pic4_14.png}
    \caption{Исходный граф}
    \label{fig:pic4_14}
\end{figure}


\begin{figure}[H]
    \centering
    \includegraphics[scale=0.75]{man-source/images/ch4/pic4_15.png}
    \caption{Результат работы агента}
    \label{fig:pic4_15}
\end{figure}


\begin{figure}[H]
    \centering
    \includegraphics[scale=0.75]{man-source/images/ch4/pic4_16.png}
    \caption{Исходный граф}
    \label{fig:pic4_16}
\end{figure}


\begin{figure}[H]
    \centering
    \includegraphics[scale=0.75]{man-source/images/ch4/pic4_17.png}
    \caption{Результат работы агента}
    \label{fig:pic4_17}
\end{figure}


\begin{figure}[H]
    \centering
    \includegraphics[scale=0.75]{man-source/images/ch4/pic4_18.png}
    \caption{Минимальный остов исходного графа}
    \label{fig:pic4_18}
\end{figure}


\begin{figure}[H]
    \centering
    \includegraphics[scale=0.75]{man-source/images/ch4/pic4_19.png}
    \caption{Множество мостов исходного графа}
    \label{fig:pic4_19}
\end{figure}


\begin{figure}[H]
    \centering
    \includegraphics[scale=0.75]{man-source/images/ch4/pic4_20.png}
    \caption{Исходный граф}
    \label{fig:pic4_20}
\end{figure}


\begin{figure}[H]
    \centering
    \includegraphics[scale=0.75]{man-source/images/ch4/pic4_21.png}
    \caption{Результат работы агента}
    \label{fig:pic4_21}
\end{figure}


\begin{figure}[H]
    \centering
    \includegraphics[scale=0.75]{man-source/images/ch4/pic4_22.png}
    \caption{Исходный граф}
    \label{fig:pic4_22}
\end{figure}


\begin{figure}[H]
    \centering
    \includegraphics[scale=0.75]{man-source/images/ch4/pic4_23.png}
    \caption{Результат работы агента}
    \label{fig:pic4_23}
\end{figure}


\begin{figure}[H]
    \centering
    \includegraphics{man-source/images/ch4/pic4_24.png}
    \caption{Схема наполнения емкости}
    \label{fig:pic4_24}
\end{figure}


\end{document}